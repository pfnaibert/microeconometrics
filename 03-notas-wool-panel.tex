\documentclass[11pt, oneside, a4paper, article]{article}
% \documentclass[11pt, oneside, a4paper, article, ms]{memoir}
% \documentclass[11pt,oneside,a4paper]{letter}
\usepackage[a4paper, margin=1in]{geometry}   % MARGENS

%%%%%%%%%%%%%%%%%%%%%%%%%%%%%%%%%%%%%%%%%%%%%%%%%%%%%%
% Pacotes básicos 
\usepackage[T1]{fontenc}		% Selecao de codigos de fonte.
\usepackage[utf8]{inputenc}		% Codificacao do documento (conversão automática dos acentos)

%%%%%%%%%%%%%%%%%%%%%%%%%%%%%%%%%%%%%%%%%%%%%%%%%%%%%%
% BABEL
% \usepackage{babel}
\usepackage[english]{babel}
% \usepackage[brazilian]{babel}
% \selectlanguage{english}

%%%%%%%%%%%%%%%%%%%%%%%%%%%%%%%%%%%%%%%%%%%%%%%%%%%%%%
% AMS
\usepackage{amsfonts, amsmath, amssymb, amsthm}
\usepackage{mathtools, latexsym}
% \usepackage{mathabx}

%%%%%%%%%%%%%%%%%%%%%%%%%%%%%%%%%%%%%%%%%%%%%%%%%%%%%%
% Pacotes de citações
\usepackage{natbib}	% Citações}

%%%%%%%%%%%%%%%%%%%%%%%%%%%%%%%%%%%%%%%%%%%%%%%%%%%%%%
\usepackage{indentfirst}		% Indenta o primeiro parágrafo de cada seção.
\usepackage{color}				% Controle das cores
\usepackage{graphicx}			% Inclusão de gráficos
\usepackage{microtype} 			% para melhorias de justificação
\usepackage{lscape}				% Gira a página em 90 graus
%\usepackage{listings}			% Formatação para inserir códigos
\usepackage[normalem]{ulem}
\usepackage[all]{xy}
\usepackage{xcolor}
\usepackage{ragged2e}           % formatação texto
\usepackage{bm}                 % bold symbols 
\usepackage[colorlinks, citecolor=blue, urlcolor=blue, linkcolor=red]{hyperref} % referencias dentro do texo (*QUEBRA MEMOIR*)
\usepackage{url}                % URL
\usepackage{graphicx}			% Inclusão de gráficos
\usepackage{subcaption}			% Faz subfiguras
%\usepackage{subfigure}         % subfigures
% \usepackage{setspace}         % Espaçamento
\usepackage{enumitem}
% \usepackage{enumerate}
\usepackage[sharp]{easylist}
% \usepackage{titlesec}		    % Customização de seçoes
\usepackage{titling}          % personalized other things
\usepackage{fancyhdr}         % personalized page style
\usepackage{lipsum}           % dummy text
% \usepackage{exercise}       % exercises
\usepackage[pagewise]{lineno}           % show line numbers
% \linenumbers

%%%%%%%%%%%%%%%%%%%%%%%%%%%%%%%%%%%%%%%%%%%%%%%%%%%%%%
%%%%%%%%%%%%%%%%%%%%%%%%%%%%%%%%%%%%%%%%%%%%%%%%%%%%%%%%%%%%%%
% MACROS
% 2017/10/13 Sex 11:34:37 
%%%%%%%%%%%%%%%%%%%%%%%%%%%%%%%%%%%%%%%%%%%%%%%%%%%%%%%%%%%%%%%%%

%%%%%%%%%%%%%%%%%%%%%%%%%%%%%%%%%%%%%%%%%%%%%%%%%%%%%%
% pagestyle
\rhead{\thepage}
\lhead{\theauthor}
\chead{\thetitle}
\renewcommand{\headrulewidth}{.5 pt} % no rule 
\rfoot{}
\lfoot{}
\cfoot{}

%%%%%%%%%%%%%%%%%%%%%%%%%%%%%%%%%%%%%%%%%%%%%%%%%%%%%%%%%%%%%%%%%
% Maketitle customization
\renewcommand{\maketitle}{%
\begin{center}
{\Large\uppercase{\thetitle}}\\
\vspace{1 em}
{\large\textsc{By \theauthor}} \\
\vspace{1 em}
\thedate
\end{center}
}

%%%%%%%%%%%%%%%%%%%%%%%%%%%%%%%%%%%%%%%%%%%%%%%%%%%%%%%%%%%%%%%%%
% Defs, theorems, etc
% \usepackage{amsthm}

\theoremstyle{definition}
\newtheorem{defn}{Definition}[section]

\newtheorem{teo}{Theorem}[section]
% \newtheorem{cor}{Corollary}[theorem]
\newtheorem{cor}{Corollary}[section]
% \newtheorem{lem}{Lemma}[theorem]
\newtheorem{lem}{Lemma}[section]

\theoremstyle{remark}
\newtheorem*{remark}{Remark}

\renewcommand{\qedsymbol}{\blacksquare}

%%%%%%%%%%%%%%%%%%%%%%%%%%%%%%%%%%%%%%%%%%%%%%%%%%%%%%%%%%%%%%%%%
% Stats

%%%%%%%%%%%%%%%%%%%%%%%%%%%%%%%%%%%%%%%%%%%%%%%%%%%%%%%%%%%%%%%%%
% operators
\newcommand{\ssr}{\ensuremath{\text{SSR}}}
\newcommand{\tss}{\ensuremath{\text{TSS}}}

\newcommand{\E}{\ensuremath{\text{E}}}
\newcommand{\Var}{\ensuremath{\text{Var}}}
\newcommand{\Cov}{\ensuremath{\text{Cov}}}
\newcommand{\Corr}{\ensuremath{\text{Corr}}}
\newcommand{\Avar}{\ensuremath{\text{Avar}}}
\newcommand{\se}{\ensuremath{\text{se}}}

\newcommand{\Avarhat}{\ensuremath{\widehat{\text{Avar}}}}
\newcommand{\Varhat}{\ensuremath{\widehat{\text{Var}}}}

\newcommand{\p}{\ensuremath{\text{P}}}

\newcommand{\B}{\ensuremath{\text{B}}}
% \newcommand{\H}{\ensuremath{\text{H}}}
% \renewcommand{\L}{\ensuremath{\text{L}}}

\newcommand{\rank}{\ensuremath{\text{rank}}}
\newcommand{\posto}{\ensuremath{\text{posto}}}
\newcommand{\MSE}{\ensuremath{\text{MSE}}}

\newcommand{\chisq}{\ensuremath{\chi^{2}}}

\newcommand{\nor}{\ensuremath{\mathcal{N}}}

\newcommand{\proj}{\ensuremath{\text{proj}}}


%%%%%%%%%%%%%%%%%%%%%%%%%%%%%%%%%%%%%%%%%%%%%%%%%%%%%%%%%%%%%%%%%
% Lims
\newcommand{\plim}{\ensuremath{\text{plim} \,}}

\newcommand{\arrowp}{\ensuremath{ \overset{p}{\longrightarrow} } }
\newcommand{\arrowd}{\ensuremath{ \overset{d}{\longrightarrow} } }
\newcommand{\asim}{\ensuremath{\overset{a}{\sim}}}

\newcommand{\bigO}{\ensuremath{\text{O}}}
\newcommand{\bigOp}{\ensuremath{\text{O}_{p}}}

\newcommand{\lito}{\ensuremath{\text{o}}}
\newcommand{\litop}{\ensuremath{\text{o}_{p}}}

%%%%%%%%%%%%%%%%%%%%%%%%%%%%%%%%%%%%%%%%%%%%%%%%%%%%%%%%%%%%%%%%%
% greeks
\newcommand{\alphavec}{\ensuremath{\mbs{\alpha}}}

\newcommand{\sigmahat}{\ensuremath{\hat{\sigma}}}
\newcommand{\Sigmahat}{\ensuremath{\widehat{\Sigma}}}

\newcommand{\Sigmainv}{\ensuremath{\Sigma^{-1}}}
\newcommand{\Sigmainvhat}{\ensuremath{\widehat{\Sigma^{-1}}}}

\newcommand{\Omegahat}{\ensuremath{\widehat{\Omega}}}

\newcommand{\mubold}{\ensuremath{\mbs{\mu}}}
\newcommand{\muvec}{\ensuremath{\mbs{\mu}}}

\newcommand{\betahat}{\ensuremath{\hat{\beta}}}
\newcommand{\betahatbold}{\ensuremath{\widehat{\mbs{\beta}}}}

\newcommand{\betavec}{\ensuremath{\mbs{\beta}}}
\newcommand{\betavechat}{\ensuremath{\widehat{\mbs{\beta}}}}

\newcommand{\gammavec}{\ensuremath{\mbs{\gamma}}}
\newcommand{\deltavec}{\ensuremath{\mbs{\delta}}}
\newcommand{\pivec}{\ensuremath{\mbs{\pi}}}
\newcommand{\pimat}{\ensuremath{\mbs{\Pi}}}

\newcommand{\thetabold}{\ensuremath{\mbs{\theta}}}
\newcommand{\Thetabold}{\ensuremath{\mbs{\Theta}}}
\newcommand{\thetahat}{\ensuremath{\hat{\theta}}}
\newcommand{\thetaboldhat}{\ensuremath{\hat{\mbs{\theta}}}}

\newcommand{\thetatil}{\ensuremath{\tilde{\theta}}}
\newcommand{\thetatilbold}{\ensuremath{\tilde{\mbs{\theta}}}}

%%%%%%%%%%%%%%%%%%%%%%%%%%%%%%%%%%%%%%%%%%%%%%%%%%%%%%%%%%%%%%%%%
% Vectors

\newcommand{\onevec}{\ensuremath{\mbs{1}}}
\newcommand{\zerovec}{\ensuremath{\mbs{0}}}


\newcommand{\bvec}{\ensuremath{\mbf{b}}}
\newcommand{\cvec}{\ensuremath{\mbf{c}}}
\newcommand{\fvec}{\ensuremath{\mbf{f}}}
\newcommand{\gvec}{\ensuremath{\mbf{g}}}
\newcommand{\hvec}{\ensuremath{\mbf{h}}}
\newcommand{\rvec}{\ensuremath{\mbf{r}}}
\newcommand{\uvec}{\ensuremath{\mbs{u}}}
\newcommand{\vvec}{\ensuremath{\mbf{v}}}
\newcommand{\vhat}{\ensuremath{\hat{v}}}
\newcommand{\xvec}{\ensuremath{\mbs{x}}}
\newcommand{\yvec}{\ensuremath{\mbs{y}}}
\newcommand{\zvec}{\ensuremath{\mbf{z}}}
\newcommand{\wvec}{\ensuremath{\mbf{w}}}

\newcommand{\uhat}{\ensuremath{\hat{u}}}
\newcommand{\yhat}{\ensuremath{\hat{y}}}

\newcommand{\uvechat}{\ensuremath{\widehat{\mbs{u}}}}
\newcommand{\yvechat}{\ensuremath{\widehat{\mbs{y}}}}

%%%%%%%%%%%%%%%%%%%%%%%%%%%%%%%%%%%%%%%%%%%%%%%%%%%%%%%%%%%%%%%%%
% Matrices

\newcommand{\Amat}{\ensuremath{\mbf{A}}}
\newcommand{\Bmat}{\ensuremath{\mbf{B}}}
\newcommand{\Cmat}{\ensuremath{\mbf{C}}}
\newcommand{\Dmat}{\ensuremath{\mbf{D}}}
\newcommand{\Imat}{\ensuremath{\mbf{I}}}
\newcommand{\Mmat}{\ensuremath{\mbf{M}}}
\newcommand{\Mzero}{\ensuremath{\mbf{M}^{0}}}

\newcommand{\Pmat}{\ensuremath{\mbf{P}}}
\newcommand{\Qmat}{\ensuremath{\mbf{Q}}}
\newcommand{\Rmat}{\ensuremath{\mbf{R}}}
\newcommand{\Vmat}{\ensuremath{\mbf{V}}}
\newcommand{\Xmat}{\ensuremath{\mbf{X}}}
\newcommand{\Ymat}{\ensuremath{\mbf{y}}}
\newcommand{\Zmat}{\ensuremath{\mbf{Z}}}
\newcommand{\Wmat}{\ensuremath{\mbf{Z}}}

\newcommand{\Ahat}{\ensuremath{\widehat{A}}}
\newcommand{\Bhat}{\ensuremath{\widehat{B}}}
\newcommand{\Dhat}{\ensuremath{\widehat{D}}}
\newcommand{\Vhat}{\ensuremath{\widehat{V}}}

\newcommand{\Amathat}{\ensuremath{\widehat{\mbf{A}}}}
\newcommand{\Bmathat}{\ensuremath{\widehat{\mbf{B}}}}
\newcommand{\Vmathat}{\ensuremath{\widehat{\mbf{V}}}}

\newcommand{\Ahatbold}{\ensuremath{\widehat{\mbf{A}}}}
\newcommand{\Bhatbold}{\ensuremath{\widehat{\mbf{B}}}}
\newcommand{\Chatbold}{\ensuremath{\widehat{\mbf{C}}}}
\newcommand{\Dhatbold}{\ensuremath{\widehat{\mbf{D}}}}
\newcommand{\Vhatbold}{\ensuremath{\widehat{\mbf{V}}}}

\renewcommand{\LG}{\mathcal{L}}
\renewcommand{\l}{\ell}

%%%%%%%%%%%%%%%%%%%%%%%%%%%%%%%%%%%%%%%%%%%%%%%%%%%%%%%%%%%%%%%%%

%%%%%%%%%%%%%%%%%%%%%%%%%%%%%%%%%%%%%%%%%%%%%%%%%%%%%%%%%%%%%%%%%
% inner product
% produto interno na forma <x,y>
\DeclarePairedDelimiterX\innerp[2]{\langle}{\rangle}{#1,#2}
%   \DeclarePairedDelimiterX\floor[1]{\lfloor}{\rfloor}{#1}
\DeclareMathOperator{\Tr}{Tr}
\DeclareMathOperator*{\argmin}{\arg\!\min}
\DeclareMathOperator*{\argmax}{\arg\!\max}

%%%%%%%%%%%%%%%%%%%%%%%%%%%%%%%%%%%%%%%%%%%%%%%%%%%%%%%%%%%%%%%%%
% sequencias
%  fazer $\{x_n\}$ e $\{x\}_n^N$ mais rápido.
\renewcommand{\seq}[1]{\{#1\}}

%%%%%%%%%%%%%%%%%%%%%%%%%%%%%%%%%%%%%%%%%%%%%%%%%%%%%%%%%%%%%%%%%
% Atalhos para notação de conjuntos dos números reais, naturais, etc
% \renewcommand{\R}{\mathbb{R}}

\newcommand{\model}{\ensuremath{\mathcal{M}}}
\newcommand{\RS}{\ensuremath{R^2}}
\newcommand{\N}{\ensuremath{\mathbb{N}}}
\newcommand{\Q}{\ensuremath{\mathbb{Q}}}
\newcommand{\R}{\ensuremath{\mathbb{R}}} 
\newcommand{\Rn}{\ensuremath{\mathbb{R}^n}} 
\newcommand{\Rm}{\ensuremath{\mathbb{R}^m}} 
\newcommand{\Rp}{\ensuremath{\mathbb{R}^p}} 
\newcommand{\I}{\ensuremath{\mathbb{R}\setminus\mathbb{Q}}}
\newcommand{\md}{\ensuremath{\mathcal{D}}}

%%%%%%%%%%%%%%%%%%%%%%%%%%%%%%%%%%%%%%%%%%%%%%%%%%%%%%%%%%%%%%%
% atalhos para outros operadores matemáticos
% \DeclareMathOperator{\L}{\mathcal{L}}
\DeclareMathOperator{\eu}{\mathrm{e}}
\DeclareMathOperator{\de}{\mathrm{d}}
\DeclareMathOperator{\up}{\uparrow}
\DeclareMathOperator{\dn}{\downarrow}

%%%%%%%%%%%%%%%%%%%%%%%%%%%%%%%%%%%%%%%%%%%%%%%%%%%%%%%%%%%%%%%
% Derivadas
\newcommand{\dpf}[2]{\frac{d#1}{d#2}}
\newcommand{\dsf}[2]{\frac{d^2#1}{d#2^2}}
\newcommand{\opt}[1]{{#1}^{*}}

%%%%%%%%%%%%%%%%%%%%%%%%%%%%%%%%%%%%%%%%%%%%%%%%%%%%%%%%%%%%%%%
% Derivadas Parciais
\newcommand{\pdf}[2]{\frac{\partial#1}{\partial#2}}
\newcommand{\psdf}[2]{\frac{\partial^2#1}{\partial#2^2}}
\newcommand{\pcdf}[3]{\frac{\partial^2#1}{\partial#2 \partial#3}}

%%%%%%%%%%%%%%%%%%%%%%%%%%%%%%%%%%%%%%%%%%%%%%%%%%%%%%%%%%%%%%%
% função valor
\newcommand{\vth}{V_{\theta}}

%%%%%%%%%%%%%%%%%%%%%%%%%%%%%%%%%%%%%%%%%%%%%%%%%%%%%%%%%%%%%%%
% Atalhos para cores:

\newcommand{\red}[1]{\textcolor{red}{#1}}
\newcommand{\blue}[1]{\textcolor{blue}{#1}}
\newcommand{\green}[1]{\textcolor{green}{#1}}
\newcommand{\magenta}[1]{\textcolor{magenta}{#1}}
% \newcommand{\hl}[1]{\colorbox{yellow}{#1}}

%%%%%%%%%%%%%%%%%%%%%%%%%%%%%%%%%%%%%%%%%%%%%%%%%%%%%%%%%%%%%%%
% GREEK LETTERS
\newcommand{\ga}{\ensuremath{\alpha}} 
\newcommand{\gab}{\ensuremath{\boldsymbol{\alpha}}} 
\newcommand{\gb}{\ensuremath{\beta}} 
\newcommand{\gbb}{\ensuremath{\boldsymbol{\beta}}} 
\newcommand{\G}{\ensuremath{\Gamma}} 
\newcommand{\gg}{\ensuremath{\gamma}} 
\newcommand{\D}{\ensuremath{\Delta}} 
\newcommand{\gl}{\ensuremath{\lambda}} 
\newcommand{\gL}{\ensuremath{\Lambda}} 
\newcommand{\gib}{\ensuremath{\boldsymbol{\iota}}} 
\newcommand{\gi}{\ensuremath{\iota}}
\newcommand{\err}{\ensuremath{\varepsilon}} 

%%%%%%%%%%%%%%%%%%%%%%%%%%%%%%%%%%%%%%%%%%%%%%%%%%%%%%%%%%%%%%%
% 
\newcommand{\bm}[1]{\boldsymbol{#1}}
\newcommand{\mbs}[1]{\boldsymbol{#1}}
\newcommand{\mbf}[1]{\mathbf{#1}}

%%%%%%%%%%%%%%%%%%%%%%%%%%%%%%%%%%%%%%%%%%%%%%%%%%%%%%%%%%%%%%%
% Não faço a mínima ideia o que esse comandos fazem
% \renewcommand{\cancel}{\color{red}}
% \newcommand{\encircle}[1]{\tikz[baseline= (X.base)]\node(X)[draw, shape=circle, inner sep=0]{\strut#1};}
%%%%%%%%%%%%%%%%%%%%%%%%%%%%%%%%%%%%%%%%%%%%%%%%%%%%%%%%%%%%%%%




% \pagestyle{headings}
\pagestyle{fancy}
\numberwithin{equation}{section}
% \setcounter{equation}{0}

%%%%%%%%%%%%%%%%%%%%%%%%%%%%%%%%%%%%%%%%%%%%%%%%%%%%%%
% Begin
\begin{document}

%%%%%%%%%%%%%%%%%%%%%%%%%%%%%%%%%%%%%%%%%%%%%%%%%%%%%%
% METADATA
\title{Microeconometrics: Lecture Notes \cite{wool-2010}}
\author{Paulo F. Naibert}
% \date{25/06/2020}
% \date{\today}

%%%%%%%%%%%%%%%%%%%%%%%%%%%%%%%%%%%%%%%%%%%%%%%%%%%%%%
\thispagestyle{empty}
\pagenumbering{gobble}

\begin{center}
% \textbf{UNIVERSIDADE FEDERAL DO RIO GRANDE DO SUL}
% \\
% \textbf{PROGRAMA DE PÓS-GRADUAÇÃO EM ECONOMIA}
% \\
\textbf{Microeconometrics}

\vfill
\textbf{\thetitle}

\vfill
\textbf{Autor: Paulo Ferreira Naibert } 


\end{center}

\vfill

\begin{center}
\textbf{Porto Alegre \\ 04/07/2020 \\ Revisão: \today}
\end{center}

%%%%%%%%%%%%%%%%%%%%%%%%%%%%%%%%%%%%%%%%%%%%%%%%%%%%%%
\clearpage
\pagenumbering{arabic}

%%%%%%%%%%%%%%%%%%%%%%%%%%%%%%%%%%%%%%%%%%%%%%%%%%%%%%

\begin{center}
\textbf{Wooldridge, Jeffrey M. 2010. \textit{Econometric Analysis of Cross Section and Panel Data.} 2ED. Boston, Massachussetts: MIT Press.}
\end{center}

% \renewcommand{\labelenumi}{\Roman{enumi}}
% \renewcommand{\labelenumii}{\arabic{enumii}}

\begin{enumerate}[noitemsep, label=\Roman* ]

\item Introduction and Background
\begin{enumerate}[noitemsep, leftmargin = 1em, label=\arabic*.]
\item  Introduction
\item  Conditional Expectations and Related Concepts
\item  Basic Asymptotic Theory
\end{enumerate}

\item Linear Models
\begin{enumerate}[noitemsep, leftmargin = 1em, label=\arabic*.]
\addtocounter{enumii}{3}
\item  The Single-Equation Linear Model and OLS Estimation
\item  Instrumental Variables Estimation of Single-Equation Linear Models
\item  Additional Single-Equation Topics
\item  Estimating Systems of Equations by OLS and GLS
\item  System Estimation by Instrumental Variables
\item  Simulteneous Equations Models
\item  Basic Linear Unobserved Effects Panel Data Models
\item  More Topics in Linear Unobserved Effects Models
\end{enumerate}

\item General Approaches to Nonlinear Estimation
\begin{enumerate}[noitemsep, leftmargin = 1em, label=\arabic*.]
\addtocounter{enumii}{11}
\item  M-Estimation
\item  Maximum Likelihood Methods
\item  Generalized Method of Moments and Minimum Distance Estimation
\end{enumerate}

\item Nonlinear Models and Related Topics
\begin{enumerate}[noitemsep, leftmargin = 1em, label=\arabic*.]
\addtocounter{enumii}{14}
\item  Discrete Response Models
\item  Corner Solution Outcomes and Censored Regression Molds
\item  Sample Selection, Attrition, and Stratified Sampling
\item  Estimating Average Treatment Effects
\item  Count Data and Related Models
\item  Duration Analysis
\end{enumerate}
\end{enumerate}

%%%%%%%%%%%%%%%%%%%%%%%%%%%%%%%%%%%%%%%%%%%%%%%%%%%%%%
\clearpage
\section{Introduction}
\noindent
\citet[C.1, pp. 3--9]{wool-2010} 

%%%%%%%%%%%%%%%%%%%%%%%%%%%%%%%%%%%%%%%%%%%%%%%%%%%%%%
\begin{enumerate}[noitemsep, label*=1.\arabic*.]
\item  Causal Relationships and Ceteris Paribus Analysis

\item  The Stochastic Setting and Asymptotic Analysis
\begin{enumerate}[noitemsep, leftmargin=1 em, label*=\arabic*.]
\item Data Structures
\item Asymptotic Analysis
\end{enumerate}

\item  The Stochastic Setting and Asymptotic Analysis
\item  Some Examples
\item  Why Not Fixed Explanatory Analysis
\end{enumerate}

%%%%%%%%%%%%%%%%%%%%%%%%%%%%%%%%%%%%%%%%%%%%%%%%%%%%%%
\clearpage
\section{Conditional Expectations and Related Concepts}
\noindent
\citet[C.2, pp. 13--34]{wool-2010} 

%%%%%%%%%%%%%%%%%%%%%%%%%%%%%%%%%%%%%%%%%%%%%%%%%%%%%%
\begin{enumerate}[noitemsep, label*=2.\arabic*.]
\item  The Role of Conditional Expectations in Econometrics

\item  Features of Conditional Expectations
\begin{enumerate}[noitemsep, leftmargin=1 em, label*=\arabic*.]
\item Definition and Examples
\item Partial Effects, Elasticities, and Semielasticities
\item The Error Form of Models of Conditional Expectations
\item Some Properties of Conditional Expectations
\item Average Partial Effects
\end{enumerate}
\item  Linear Projections

\item[A]  Appendices

\begin{enumerate}[noitemsep]
\item[A.1] Properties of Conditional Expectations
\item[A.2] Properties of Conditional Variances
\item[A.3] Properties of Linear Projections
\end{enumerate}
\end{enumerate}

%%%%%%%%%%%%%%%%%%%%%%%%%%%%%%%%%%%%%%%%%%%%%%%%%%%%%%
\subsection{The Role of Conditional Expectations in Econometrics}

%%%%%%%%%%%%%%%%%%%%%%%%%%%%%%%%%%%%%%%%%%%%%%%%%%%%%%
\subsection{Features of Conditional Expectations}

%%%%%%%%%%%%%%%%%%%%%%%%%%%%%%%%%%%%%%%%%%%%%%%%%%%%%%
\subsubsection{Definition and Examples}

%%%%%%%%%%%%%%%%%%%%%%%%%%%%%%%%%%%%%%%%%%%%%%%%%%%%%%
\subsubsection{Partial Effects, Elasticities, and Semielasticities}

%%%%%%%%%%%%%%%%%%%%%%%%%%%%%%%%%%%%%%%%%%%%%%%%%%%%%%
\subsubsection{The Error Form of Models of Conditional Expectations}

%%%%%%%%%%%%%%%%%%%%%%%%%%%%%%%%%%%%%%%%%%%%%%%%%%%%%%
\subsubsection{Some Properties of Conditional Expectations}

%%%%%%%%%%%%%%%%%%%%%%%%%%%%%%%%%%%%%%%%%%%%%%%%%%%%%%
\subsubsection{Average Partial Effects}

%%%%%%%%%%%%%%%%%%%%%%%%%%%%%%%%%%%%%%%%%%%%%%%%%%%%%%
\subsection{Linear Projections}

%%%%%%%%%%%%%%%%%%%%%%%%%%%%%%%%%%%%%%%%%%%%%%%%%%%%%%
\subsection{Appendices}

%%%%%%%%%%%%%%%%%%%%%%%%%%%%%%%%%%%%%%%%%%%%%%%%%%%%%%
\subsubsection{Properties of Conditional Expectations}

%%%%%%%%%%%%%%%%%%%%%%%%%%%%%%%%%%%%%%%%%%%%%%%%%%%%%%
\subsubsection{Properties of Conditional Variances}

%%%%%%%%%%%%%%%%%%%%%%%%%%%%%%%%%%%%%%%%%%%%%%%%%%%%%%
\subsubsection{Properties of Linear Projections}

%%%%%%%%%%%%%%%%%%%%%%%%%%%%%%%%%%%%%%%%%%%%%%%%%%%%%%
\clearpage
\section{Basic Aymptotic Theory}

\noindent
\citet[C.3, p.35--45]{wool-2010}

%%%%%%%%%%%%%%%%%%%%%%%%%%%%%%%%%%%%%%%%%%%%%%%%%%%%%%
\begin{enumerate}[noitemsep, label*=3.\arabic*.]
\item  Convergence of Deterministic Sequences
\item  Convergence in Probability and Bounded in Probability
\item  Convergence in Distribution
\item  Limit Theorems for Random Samples

\item  Limiting Behavior of Estimators and Test Statistics
\begin{enumerate}[noitemsep, leftmargin=1 em, label*=\arabic*.]
\item Asymptotic Properties of Estimators
\item Asymptotic Properties of Test Statistics
\end{enumerate}
\end{enumerate}

\blue{White (1984)}.

%%%%%%%%%%%%%%%%%%%%%%%%%%%%%%%%%%%%%%%%%%%%%%%%%%%%%%
\subsection{Convergence of Deterministic Sequences}

%%%%%%%%%%%%%%%%%%%%%%%%%%%%%%%%%%%%%%%%%%%%%%%%%%%%%%
\begin{defn}
\item
\begin{enumerate}
\item 
A sequence of nonrandom numbers 
$\seq{a_{N}; N = 1, 2, \dots}$
\textbf{converges} to $a$ (has limit $a$) if for all $\err > 0$ there exists $N_{\err}$ such that if $N > N_{\err}$, then $|a_{N} - a| < \err$.
We write
$a_{N} \to a \text{ as } N \to +\infty$

\item
A  sequence 
$\seq{a_{N}; N = 1, 2, \dots}$
is \textbf{bounded} if and only if there is some $b < \infty$ such that $|a_{N}| < b$ for all $N = 1, 2, \dots$.
Otherwise, we say that $\seq{a_{N}}$ is \textbf{unbounded}.
\end{enumerate}
\end{defn}
%%%%%%%%%%%%%%%%%%%%%%%%%%%%%%%%%%%%%%%%%%%%%%%%%%%%%%

%%%%%%%%%%%%%%%%%%%%%%%%%%%%%%%%%%%%%%%%%%%%%%%%%%%%%%
\begin{defn}
\item
\begin{enumerate}
\item 
A sequence $\seq{a_{N}}$ is $\bigO( N^{\lambda})$ (\textit{at most of order} $N^{\lambda}$) if 
$N^{-\lambda} a_{N}$ 
is bounded.
When $\lambda = 0$, $\seq{a_{N}}$ is bounded, and we also write $a_{N} = \bigO(1)$ \textit{(big oh one)}.

\item
A  sequence 
$\seq{a_{N}}$ is $\lito(N^{\lambda})$ if  $N^{-\lambda} a_{N} \to 0$.
When $\lamda = 0$, $a_{N}$ converges to zero, and we also write
$a_{N} = \lito(1)$ \textit{(little oh one)}.

\end{enumerate}
\end{defn}
%%%%%%%%%%%%%%%%%%%%%%%%%%%%%%%%%%%%%%%%%%%%%%%%%%%%%%

%%%%%%%%%%%%%%%%%%%%%%%%%%%%%%%%%%%%%%%%%%%%%%%%%%%%%%
\subsection{Convergence in Probability and Bounded in Probability}

%%%%%%%%%%%%%%%%%%%%%%%%%%%%%%%%%%%%%%%%%%%%%%%%%%%%%%
\begin{defn}
\item
\begin{enumerate}
\item 
A sequence of random variables
$\seq{x_{N}; N = 1, 2, \dots}$
\textbf{converges in probability} to the constant $a$ if for all $\err > 0$, 
\begin{equation*}
	\p \left[ |x_{N} - a| > \err \right] \to 0 \quad \text{as } N \to \infty.
\end{equation*}
We write $x_{N} \arrowp a$  and say that $a$ is the \textbf{probability limit} ($\plim$) of $x_{N}$; $\plim x_{N} = a$.

\item
	In the special case where $a =0$ ($\plim x_{N} = 0$), we also say that $\seq{x_{N}}$ is $\litop(1)$ (\textit{little oh p one}).
	We also write $x_{N} = \litop(1)$ or
	$x_{N} \arrowp 0$ or
	
\item
	A sequence of random variables $\seq{x_{N}}$ is \textbf{bounded in probability} if and only if for every $\err > 0$, there exists a $b_{\err} < \infty$ and an integer $N_{\err}$ such that
\begin{equation*}
\p \left[ | x_{N} | \geq b_{\err} \right]  < \err 
\quad \text{for all } N \geq N_{\err}.
\end{equation*}
We write $x_{N} = \bigOp(1)$ ($x_{N}$ is \textit{big oh p one}).

\end{enumerate}
\end{defn}
%%%%%%%%%%%%%%%%%%%%%%%%%%%%%%%%%%%%%%%%%%%%%%%%%%%%%%

\begin{lem}
If $x_{N} \arrowp a$, then $x_{N} = \bigOp(1)$.
This lemma also holds for vectors and matrices.
\end{lem}

%%%%%%%%%%%%%%%%%%%%%%%%%%%%%%%%%%%%%%%%%%%%%%%%%%%%%%
\subsection{Convergence in Distribution}

%%%%%%%%%%%%%%%%%%%%%%%%%%%%%%%%%%%%%%%%%%%%%%%%%%%%%%
\subsection{Limit Theorems for Random Samples}

%%%%%%%%%%%%%%%%%%%%%%%%%%%%%%%%%%%%%%%%%%%%%%%%%%%%%%
\subsection{Limiting Behavior of Estimators and Test Statistics}

%%%%%%%%%%%%%%%%%%%%%%%%%%%%%%%%%%%%%%%%%%%%%%%%%%%%%%
\subsubsection{Asymptotic Properties of Estimators}

%%%%%%%%%%%%%%%%%%%%%%%%%%%%%%%%%%%%%%%%%%%%%%%%%%%%%%
\subsubsection{Asymptotic Properties of Test Statistics}

%%%%%%%%%%%%%%%%%%%%%%%%%%%%%%%%%%%%%%%%%%%%%%%%%%%%%%
\clearpage
\section{The Single-Equation Linear Model and OLS Estimation}

\noindent
\citet[C.4 p.49--76]{wool-2010}

%%%%%%%%%%%%%%%%%%%%%%%%%%%%%%%%%%%%%%%%%%%%%%%%%%%%%%
\begin{enumerate}[noitemsep, label*=4.\arabic*]
\item  Overview of the Single-Equation Linear Model

\item  Asymptotic Porperties of OLS
	\begin{enumerate}[noitemsep, leftmargin=1 em, label*=.\arabic*.]
\item Consitency
\item Asymptotic Inference Using OLS
\item Heteroskedasticity-Robust Inference
\item Lagrange Multiplier (Score) Tests
\end{enumerate}

\item  OLS Solutions to the Omitted Variables Problem
\begin{enumerate}[noitemsep, leftmargin=1 em, label*=\arabic*.]
\item OLS Ignoring the Omitted Variables
\item The Porxy Variable--OLS Solution
\item Models with Interactions in Unobservables
\end{enumerate}

\item  Properties of OLS under Measurement Error
\begin{enumerate}[noitemsep, leftmargin=1 em, label*=\arabic*.]
\item Measurement Error in the Dependent Variable
\item Measurement Error in an Explanatory Variable
\end{enumerate}
\end{enumerate}

%%%%%%%%%%%%%%%%%%%%%%%%%%%%%%%%%%%%%%%%%%%%%%%%%%%%%%
\subsection{Modelo de equações lineares} 
\noindent
\citet[Sec. 4.1 -- Overview of the Single-Equation Linear Model; p.49]{wool-2010}

O modelo populacional que estudamos é linear em seus parâmetros,

\vspace{-1 em}
\begin{align} \label{ols:mod}
y &= \beta_{0} + \beta_{1} x_{1} + \dots + \beta_{K} x_{K} + u
\end{align}
\vspace{-1 em}

\noindent
onde

\begin{itemize} \itemsep0pt
\item $y, x_{1}, \dots, x_{K}$  são escalares aleatórios e observáveis (i.e., conseguimos observá-los em uma amostra aleatória da população);

\item $u$ é o \textit{random disturbance} não observável, ou erro; 

\item $\beta_{0}, \beta_{1}, \dots, \beta_{K}$ são parâmetros (constantes) que gostaríamos de estimar.
\end{itemize}

%%%%%%%%%%%%%%%%%%%%%%%%%%%%%%%%%%%%%%%%%%%%%%%%%%%%%%
\paragraph{Structural Model}
\blue{Goldberger (1972)}
\footnote{Goldberger (1972), ``Structural Equation Methods in the Social Sciences,'' \textit{Econometrica} 40, 979--1001.}
defines a \textbf{structural model} as one representing a causal relationship, as opposed to a relationship that simply captures statistical associations.
A structural equation can be obtained from an economic model, or it can be obtained through informal reasoning.
Someteimes the structural model is directly estimable.
Other times times we must combine auxiliary assumptions about other variables with algebraic manipulations to arrive at an \textbf{estimable model}.
In addtion, we will often have reasons to estimate \textbf{nonstructural equations}, sometimes as a precursor to estimating a structural equation.

%%%%%%%%%%%%%%%%%%%%%%%%%%%%%%%%%%%%%%%%%%%%%%%%%%%%%%
\paragraph{Zero Mean Error}
The key condition for OLS to consitently estimate the $\beta_{j}$ (assuming we have availabe a random sample from the population) is that the error (in the population) has mean zero and is uncorrelated with each of the regressors:

\vspace{-1 em}
\begin{align} \label{zeromeanerror}
	\E( u ) = 0 \, , \quad \Cov(x_{j}, u) = 0 \, , \quad j = 1, \dots, K.
\end{align}

The zero-mean assumptions is for free when an intercept is included.
It is the zero covariance of $u$ with each $x_{j}$ that is important.
From \red{XX} we know that equation \eqref{ols:mod} and assumption \eqref{zeromeanerror} is equivalent to defining the\textbf{linear projection} of $y$ onto 
$(1, x_{1}, \dots, x_{K})$ as
$\beta_{0} + \beta_{1} x_{1} + \dots \beta_{K} x_{K}$.

Sufficient for assumption \eqref{zeromeanerror} is the zero conditional mean assumption:

\vspace{-1 em}
\begin{align} \label{zerocondmean}
	\E( u | x_{1}, \dots, x_{K} ) = \E( u | \xbold) = 0.
\end{align}

Under equation \eqref{ols:mod} and assumption \eqref{zerocondmean}, we have the \textbf{population regression function}

\vspace{-1 em}
\begin{align} \label{pop:regfun}
	\E( y | x_{1}, \dots, x_{K} ) = \beta_{0} + \beta_{1} x_{1} + \dots \beta_{K} x_{K}. 
\end{align}

%%%%%%%%%%%%%%%%%%%%%%%%%%%%%%%%%%%%%%%%%%%%%%%%%%%%%%
\paragraph{Endogenous Variable}

An explanatory variable $x_{j}$ is said to be \textbf{endogenous} in equation \eqref{ols:mod} if it is correlated with $u$.
The usage of the word \textbf{endogenous}, in econometrics, is used broadly to describe any situation where an explanatory variable is correlated with the disturbance.
If $x_{j}$ is uncorrelated with $u$, then $x_{j}$ is said to be \textbf{exogenous} in equation \eqref{ols:mod}.
If assumption \eqref{zerocondmean} holds, then each explanatory variable is necessarily exogenous.

In applied econometrics, endogeneity usually arises in one of three ways:

\vspace{-1 em}
\begin{enumerate}[noitemsep]
	\item \textbf{Ommited Variables};
	\item \textbf{Measurement Error};
	\item \textbf{Simultaneity}. 
\end{enumerate}

%%%%%%%%%%%%%%%%%%%%%%%%%%%%%%%%%%%%%%%%%%%%%%%%%%%%%%
\subsection{Asymptotic Properties of OLS} 
\noindent
\citet[Sec. 4.2 -- Asymptotic Properties of OLS; p.51]{wool-2010}

Por conveniência, escrevemos a equação populacional em forma de vetor:

\vspace{-1 em}
\begin{align} \label{ols:mod:vec}
	y &= \mbs{x} \mbs{\beta} + u
\end{align}

\noindent
onde,

\vspace{-1 em}
\begin{description}[noitemsep]
\item [$\mbs{x} \equiv (x_{1}, \dots, x_{K})$] é um vetor $1 \times K$ de regressores;

\item [$\mbs{\beta} \equiv (\beta_{1}, \dots, \beta_{K})'$] é um vetor $K \times 1$.
\end{description}

\noindent
Uma vez que a maioria das equações contém um intercepto, assumiremos que $x_{1} \equiv 1$, visto que essa hipótese deixa a interpretação mais fácil.

%%%%%%%%%%%%%%%%%%%%%%%%%%%%%%%%%%%%%%%%%%%%%%%%%%%%%%
\paragraph{Amostra Aleatória} 

Assumimos que conseguimos obter uma amostra aleatória de tamanho $N$ da população para estimarmos $\mbs{\beta}$.
Dessa forma, $\{ (\mbs{x}_{i}, y_{i}); \, i = 1, 2, \dots, N \}$
são tratados como variáveis aleatória independentes, identicamente distribuídas, onde
$\mbs{x}_{i}$ é $1 \times K$ e $y_{i}$ é escalar.
Para cada observação $i$, temos:

%%%%%%%%%%%%%%%%%%%%%%%%%%%%%%%%%%%%%%%%%%%%%%%%%%%%%%
\vspace{-1 em}
\begin{align} \label{ols:mod:vec:i}
	y_{i} &= \mbs{x}_{i} \mbs{\beta} + u_{i}.
\end{align}

\noindent
onde
$\mbs{x}_{i}$
é um vetor $1 \times K$ de regressores.

%%%%%%%%%%%%%%%%%%%%%%%%%%%%%%%%%%%%%%%%%%%%%%%%%%%%%%
\paragraph{Notação Matricial [Meu]}
Empilhando as $N$ observações, obtemos a \textbf{Notação Matricial}:

\vspace{-1 em}
\begin{align} \label{ols:mod:mat}
	\mbs{y} &= X \mbs{\beta} + \mbs{u} 
\end{align}

\begin{description}[noitemsep]
\item [$\mbs{y}$]  é um vetor $N \times 1$;

\item [$X$]  é uma matriz $N \times K$ de regressores, com $N$ vetores, $\mbs{x}_{i}$, de dimensão $1 \times K$ empilhados;

\item [$\mbs{\beta}$] é um vetor $K \times 1$;

\item [$\mbs{u}$] é um vetor $N \times 1$;
\end{description}

\vspace{-1 em}
\begin{align*}
\mbs{y} = 
\begin{bmatrix}
	y_{1} \\ \vdots \\ y_{N}		
\end{bmatrix};
\quad
X = 
\begin{bmatrix}
	\mbs{x}_{1} \\ \vdots \\ \mbs{x}_{N}
\end{bmatrix} = 
\begin{bmatrix}
	x_{11}     & x_{12}     & \dots  & x_{1K} \\          
%	x_{21}     & x_{22}     & \dots  & x_{2K} \\         
	\vdots     & \vdots     & \ddots & \vdots \\        
	x_{N1} & x_{N2} & \dots  & x_{NK}		
\end{bmatrix};
\quad
\mbs{u} = 
\begin{bmatrix}
	u_{1} \\ \vdots \\ u_{N}		
\end{bmatrix}.
\end{align*}

%%%%%%%%%%%%%%%%%%%%%%%%%%%%%%%%%%%%%%%%%%%%%%%%%%%%%%
\subsubsection{Consistency}
\noindent
\citet[Sec. 4.2.1 -- Consistency; p.52-4]{wool-2010}

\paragraph{Assumptions}

\begin{description}[itemsep = 1ex]
	\item[OLS.1]  \textbf{population orthogonality condition:} $\E(\mbs{x}'u) = 0$;

	\item[OLS.2]  $posto[ \E( \mbs{x}' \mbs{x} ) ] = K$.
\end{description}

Because $\xbold$ contains a constant, \textbf{OLS.1} is equivalent to saying that $u$ has zero mean and is uncorrelated with each regressor.
Suficient for \textbf{OLS.1} is the zero conditional mean assumption \eqref{zerocondmean}.

Since $\E( \xbold'\xbold )$ is symmetric $K \times K$ matrix, \textbf{OLS.2} is equivalent to aassuming that $\E( \xbold'\xbold )$ is positive definite.

Under assumptions \textbf{OLS.1} and \textbf{OLS.2} the parameter $\betabold$ is \textbf{identified}.
In the context of models that are linear in the parameters under random sampling, identification of $\betabold$ simply means that $\betabold$ can be written in terms of population moments in observable variabels.
To see this, we premultiply equation \eqref{ols:mod:vec} by $\xbold'$ and take expectations:

\vspace{-1 em}
\begin{align*}
	y &= \xbold \betabold + u
	\\
	\xbold' y &= \xbold' \xbold \betabold + \xbold' u
	\\
	\E( \xbold' y) &= \E( \xbold' \xbold) \betabold + \E(\xbold' u)
	\\
	\E( \xbold' y) &= \E( \xbold' \xbold) \betabold
	\\
	\Aboxed{\betabold &= \left[ \E( \xbold' \xbold) \right]^{-1} \E( \xbold' y) }.
\end{align*}

Because $(\xbold, y)$is observed, $\betabold$ is identified.

%%%%%%%%%%%%%%%%%%%%%%%%%%%%%%%%%%%%%%%%%%%%%%%%%%%%%%
\paragraph{Analogy Principle}
The \textbf{analogy principle} for choosing and estimator says to turn to the population problem into its \textbf{sample counterparts} (\blue{Goldberger, 1968}; \blue{Manski, 1988}).
In the current application this step leads to the \textbf{method of moment}:
replace the population moments $\E( \xbold' \xbold)$ and $\E(\xbold' y)$ with the corresponding \textbf{sample averages}.
Doing so leads to the OLS estimator:

\vspace{-1 em}
\begin{align*}
\betahatbold &=
\left( N^{-1} \sum_{i=1}^{N} \mbs{x}_{i}' \mbs{x}_{i} \right)^{-1} 
\left( N^{-1} \sum_{i=1}^{N} \mbs{x}_{i}' y_{i} \right)
\\
\betahatbold &=
\betabold +
\left( N^{-1} \sum_{i=1}^{N} \mbs{x}_{i}' \mbs{x}_{i} \right)^{-1} 
\left( N^{-1} \sum_{i=1}^{N} \mbs{x}_{i}' u_{i} \right).
\end{align*}

\noindent
Isso pode ser escrito na forma matricial:

\vspace{-1 em}
\begin{align*}
\betahatbold &=	(X'X)^{-1} X'\ybold
\\
\betahatbold &=	\betabold + (X'X)^{-1} X'\ubold ,
\end{align*}

\noindent onde,

\vspace{-1 ex}
\begin{description}[noitemsep]
\item[$X$] é a matriz de dados $N \times K$ dos regressores com linha $i$ igual a $\mbs{x}_{i}$;

\item[$\ybold$] é o vetor de dados $N \times 1$ com o $i$-ésimo elemento de $\mbs{y}$ sendo representado por $y_{i}$.
\end{description}

Under \textbf{OLS.2} $X'X$ is nonsingular with probability approaching one and

\begin{align*}
\plim \left[ \left( N^{-1} \sum_{i=1}^{N} \xbold_{i}' \xbold_{i} \right)^{-1} \right] 
&= \Abold^{-1},
\end{align*}

\noindent where,
$\boxed{\Abold \equiv E(\xbold' \xbold)}$ (\red{see Corollary 3.1}).

Further, under \textbf{OLS.1}

\begin{align*}
\plim \left[ \left( N^{-1} \sum_{i=1}^{N} \xbold_{i}' u_{i} \right)^{-1} \right]
= E(\xbold'u) = \mbs{0}.
\end{align*}

Therefore, by \blue{Slutsky's Theorem} (\red{Leamma 3.4}), 

\begin{align*}
	\plim \betahatbold = \betabold + \Abold^{-1} \cdot \mbs{0} = \betabold
\end{align*}

%%%%%%%%%%%%%%%%%%%%%%%%%%%%%%%%%%%%%%%%%%%%%%%%%%%%%%
\paragraph{Resumo}
Resumimos os resultados acima com um teorema:

\begin{teo1}[Consistência do OLS]\label{ols:const}
Sob as Hipóteses \textbf{OLS.1} e \textbf{OLS.2}, o estimador de OLS, $\betahatbold$ obtido de uma amostra aleatória seguindo o modelo populacional \eqref{ols:mod:vec} é consistente para $\betabold$.
\end{teo1}

Sob as hipóteses do Teorema \ref{ols:const}, $\mbs{x} \betabold$ é uma \textbf{projeção linear} de $y$ em $\mbs{x}$.

%%%%%%%%%%%%%%%%%%%%%%%%%%%%%%%%%%%%%%%%%%%%%%%%%%%%%%
\subsubsection{Asymptotic Inference Using OLS}
\noindent
\citet[Sec. 4.2.2 -- Asymptotic INference Using OLS; p.54-5]{wool-2010}

%%%%%%%%%%%%%%%%%%%%%%%%%%%%%%%%%%%%%%%%%%%%%%%%%%%%%%
\clearpage
\section{Estimating Systems of Equations by OLS and GLS}

\noindent
\citet[C.7,  p.143--179]{wool-2010}\\

%%%%%%%%%%%%%%%%%%%%%%%%%%%%%%%%%%%%%%%%%%%%%%%%%%%%%%
\begin{enumerate}[noitemsep, label*=7.\arabic*.]
\item  Introduction
\item  Some Examples

\item  System OLS Estimation of a Multivariate Linear System
\begin{enumerate}[noitemsep, leftmargin=1 em, label*=\arabic*.]
\item Preliminaries
\item Asymptotic Porperties of System OLS
\item Testing Multiple Hypotheses
\end{enumerate}

\item Consistency and Asymptotic Normality of Generalized Least Squares
\begin{enumerate}[noitemsep, leftmargin=1 em, label*=\arabic*.]
\item Consitency
\item Asymptotic Normality
\end{enumerate}

\item Feasible GLS
\begin{enumerate}[noitemsep, leftmargin=1 em, label*=\arabic*.]
\item Asymptotic Properties
\item Asymptotic Variance of FGLS under a Standard Assumption
\end{enumerate}

\item Testing Using FGLS

\item Seemingly Unrelated Regressions, Revisited
\begin{enumerate}[noitemsep, leftmargin=1 em, label*=\arabic*.]
\item Comparison between OLS and FGLS for SUR Systems
\item Systems with Cross Equation Restrictions
\item Singular Variance Matrices in SUR Systems
\end{enumerate}

\item The Linear Panel Data Model, Revisited
\begin{enumerate}[noitemsep, leftmargin=1 em, label*=\arabic*.]
\item Assumptions for Pooled OLS
\item Dynamic Completeness
\item A Note on Time Series Persistence
\item Robust Asymptotic Variance Matrix
\item Testing for Serial Correlation and Heteroskedasticity after Pooled OLS
\item Feasible GLS Estimation under Strict Exogeneity
\end{enumerate}
\end{enumerate}


%%%%%%%%%%%%%%%%%%%%%%%%%%%%%%%%%%%%%%%%%%%%%%%%%%%%%%
\citet[Sec.7.3 -- System OLS Estimation of a Multivariate Linear System, p.147]{wool-2010} 

%%%%%%%%%%%%%%%%%%%%%%%%%%%%%%%%%%%%%%%%%%%%%%%%%%%%%%
\subsection{Preliminares}
\noindent
\citet[Sec.7.3.1]{wool-2010}

Assumimos que temos as seguintes observações \textit{cross section} $iid$:
$\seq{ (X_{i}, \mbs{y}_{i}): i=1, \dots, N}$, onde:

\vspace{-1 ex}
\begin{itemize}[itemsep = -1ex]
\item [$X_{i}$]  é uma matriz $G \times K$ e contém as variáveis explicativas que aparecem em qualquer lugar do sistema.
\item [$\mbs{y}_{i}$]  é um vetor $G \times 1$, que contém as variáveis dependentes para todas as equações $G$ (ou períodos de tempo, no caso de dados de painel).
\end{itemize}

O modelo linear multivariado para uma \red{observação (draw)} aleatória da população pode ser expresso como:

\vspace{-1 em}
\begin{align}\label{mod:SOLS}
	\mbs{y}_{i} = X_{i} \betabold + \mbs{u}_{i} \, , \quad i=1, \dots, N,
\end{align}

\noindent
onde:

\vspace{-1 em}
\begin{itemize}[itemsep = -1ex]
\item [$\betabold$] é um vetor $K \times 1$ de parâmetros de interesse; e
\item [$\mbs{u}_{i}$] é um vetor $G \times 1$ de não observáveis.
\end{itemize}

A equação \eqref{mod:SOLS} explica as $G$ variáveis $y_{i1}, \dots, y_{iG}$ em termos de $X_{i}$ e das não observáveis $\mbs{u}_{i}$.
Por causa da hipótese de amostra aleatória podemos escrever tudo em temos de uma observação genérica.

%%%%%%%%%%%%%%%%%%%%%%%%%%%%%%%%%%%%%%%%%%%%%%%%%%%%%%
\subsection{Propriedades Assintóticas do SOLS}
\noindent
\citet[Sec.7.3.1]{wool-2010}

\begin{description}[itemsep = 0ex]
\item [SOLS.1] $\E(X_{i}' \mbs{u}_{i}) = 0$.
		
\item [SOLS.2] $A \equiv \E( X_{i}' X_{i} )$ é não singular (tem posto pleno, posto igual a $K$). 
\end{description}

A hipótese \textbf{SOLS.1} é a mais fraca que podemos impor num aracabouço de regressão para conseguirmos um estimador de $\betabold$ consistente.
Essa hipótese permite que alguns elementos de $X_{i}$ sejam correlacionados com elementos de $\mbs{u}_{i}$.
Uma hipótese mais forte seria:

\vspace{-1 em}
\begin{align} \label{cond:mean:0}
	\E(\mbs{u}_{i} | X_{i} ) = \mbs{0}
\end{align}

Sob \textbf{SOLS.1}, temos:

\vspace{-1 em}
\begin{align*} 
\E[ X_{i}' ( \mbs{y}_{i} - X_{i} \betabold ) ] &= \mbs{0}
\\
\E( X_{i}' X_{i} ) \betabold &= \E( X_{i}' \mbs{y}_{i} )  
\end{align*}

Para cada $i$, $X_{i} \mbs{y}_{i}$ é um vetor aleatório $K \times 1$ e $X_{i}'X_{i}$ é uma matriz $K \times K$ aleatória simétrica, positiva semidefinida.
Então, $\E(X_{i}' X_{i})$ é sempre uma matriz $K \times K$ não aleatória simétrica, positiva semidefinida.
Para conseguirmos estimar $\betabold$ precisamos assumir que ele é o único vetor $K \times 1$ que satisfaz $\E( X_{i}' X_{i} ) \betabold = \E( X_{i}' \mbs{y}_{i} )$.
Por isso assumimos \textbf{SOLS.2} e sob \textbf{SOLS.1} e \textbf{SOLS.2}, podemos escrever $\betabold$ como:

\vspace{-1 em}
\begin{align} \label{beta:SOLS}
	\Aboxed{
\betabold &=
\left[ \E( X_{i}' X_{i} )  \right]^{-1}
\E( X_{i}' \mbs{y}_{i} )  }
\end{align}

\noindent
o que mostra que \textbf{SOLS.1} e \textbf{SOLS.2} identifica o vetor $\betabold$.
O\textbf{princípio da analogia} sugere que estimemos $\betabold$ pelas analogias amostrais de \eqref{beta:SOLS}.
Assim, definimos o estimador SOLS de $\betabold$ como:

\vspace{-1 em}
\begin{align} \label{betahat:SOLS}
\Aboxed{
\betahatbold^{SOLS} &=
\left( N^{-1} \sum_{i=1}^{N} X_{i}' X_{i}   \right)^{-1}
\left( N^{-1} \sum_{i=1}^{N} X_{i}' \mbs{y}_{i}   \right)
}.
\end{align}

Para computar $\betahatbold$ usando linguagem de computação é mais fácil utilizar a notação matricial

\vspace{-1 em}
\begin{align} \label{betahat:SOLS:mat}
\Aboxed{
\betahatbold^{SOLS} &=
\left(  X' X   \right)^{-1} \left(  X' \mbs{y}   \right)
}
\end{align}

\noindent
onde

\vspace{-1 em}
\begin{description}[itemsep = -1ex]
\item [$X \equiv (X_{1}', \dots, X_{N}')$]  é uma matriz $NG \times K$ dos $X_{i}$ empilhados.

\item [$\mbs{y} \equiv (\mbs{y}_{1}', \dots, \mbs{y}_{N}')$] é um vetor $NG \times 1$ das observações $\mbs{y}_{i}$ empilhadas.
\end{description}

%%%%%%%%%%%%%%%%%%%%%%%%%%%%%%%%%%%%%%%%%%%%%%%%%%%%%%
\paragraph{SOLS para SUR} Estimação SOLS para um modelo SUR é equivalente a OLS equação a equação.

%%%%%%%%%%%%%%%%%%%%%%%%%%%%%%%%%%%%%%%%%%%%%%%%%%%%%%
\subsubsection{Consistência}

Para provarmos a \textbf{consistência} do estimador, usamos a equação \eqref{betahat:SOLS}:

\vspace{-1 em}
\begin{align*}
\betahatbold^{SOLS} &=
\left( N^{-1}\sum_{i=1}^{N} X_{i}' X_{i}   \right)^{-1}
\left( N^{-1}\sum_{i=1}^{N} X_{i}' \mbs{y}_{i}   \right)
\\ &=
\left( N^{-1}\sum_{i=1}^{N} X_{i}' X_{i}   \right)^{-1}
\left[ N^{-1}\sum_{i=1}^{N} X_{i}' (X_{i} \betabold + \mbs{u}_{i})   \right]
\\ &=
\left( N^{-1}\sum_{i=1}^{N} X_{i}' X_{i}   \right)^{-1}
\left( N^{-1}\sum_{i=1}^{N} X_{i}' X_{i} \betabold    \right)
+
\left( N^{-1}\sum_{i=1}^{N} X_{i}' X_{i}   \right)^{-1}
\left( N^{-1}\sum_{i=1}^{N} X_{i}' \mbs{u}_{i}   \right)
\\ &=
\betabold
+
\left( N^{-1}\sum_{i=1}^{N} X_{i}' X_{i}   \right)^{-1}
\left( N^{-1}\sum_{i=1}^{N} X_{i}' \mbs{u}_{i}   \right)
\end{align*}

Por \textbf{SOLS.1},
$N^{-1} \sum_{i=1}^{N} X_{i}' \mbs{u}_{i} \arrowp \mbs{0}$;
e por \textbf{SOLS.2}
$\left( N^{-1} \sum_{i=1}^{N} X_{i}' X_{i} \right)^{-1} \arrowp A^{-1}$.

Resumimos esse resultado pelo seguinte Teorema:

%%%%%%%%%%%%%%%%%%%%%%%%%%%%%%%%%%%%%%%%%%%%%%%%%%%%%%
\begin{teo1}[Consistência do SOLS]\label{SOLS:const}
Sob Hipóteses \textbf{SOLS.1} e \textbf{SOLS.2}, temos 
\begin{align*}
\Aboxed{
	\betahatbold^{SOLS} \arrowp \betabold
}.
\end{align*}
\end{teo1}
%%%%%%%%%%%%%%%%%%%%%%%%%%%%%%%%%%%%%%%%%%%%%%%%%%%%%%

%%%%%%%%%%%%%%%%%%%%%%%%%%%%%%%%%%%%%%%%%%%%%%%%%%%%%%
\subsubsection{Normalidade Assintótica}

Para fazermos \textbf{Inferência}, precisamos achar a variância assintótica do estimador de OLS sob, essencialmente, as mesmas duas hipóteses. 
Tecnicamente, a seguinte derivação exige os elementos de
$X_{i}' \mbs{u}_{i} \mbs{u}_{i}' X_{i}$
tenham \textit{finite expected absolute value}.
De \eqref{betahat:SOLS} e \eqref{mod:SOLS}, escrevemos:

\vspace{-1 em}
\begin{align*} 
\betahatbold  &=
\betabold +
\left( N^{-1} \sum_{i=1}^{N} X_{i}' X_{i}   \right)^{-1}
\left( N^{-1} \sum_{i=1}^{N} X_{i}' \mbs{u}_{i}   \right)
\\ 
(\betahatbold - \betabold) &= 
\left( N^{-1} \sum_{i=1}^{N} X_{i}' X_{i}   \right)^{-1}
\left( N^{-1} \sum_{i=1}^{N} X_{i}' \mbs{u}_{i}   \right)
\\ 
\Aboxed{
\sqrt{N}(\betahatbold - \betabold) &= 
\left( N^{-1} \sum_{i=1}^{N} X_{i}' X_{i}   \right)^{-1}
\left( N^{-1/2} \sum_{i=1}^{N} X_{i}' \mbs{u}_{i}   \right)
}.
\end{align*}

Uma vez que $\E(X_{i}' \mbs{u}_{i})=0$, sob a hipótese \textbf{SOLS.1}, o CLT implica que:

\vspace{-1 em}
\begin{align*} 
N^{-1/2} \sum_{i=1}^{N} X_{i} \mbs{u}_{i} \arrowd N(\mbs{0}, B),
\end{align*}

\noindent
onde

\vspace{-1 em}
\begin{align*} 
B \equiv \E(X_{i}' \mbs{u}_{i} \mbs{u}_{i}' X_{i}) \equiv \Var(X_{i} \mbs{u}_{i}).
\end{align*}

\noindent
Em particular,

\vspace{-1 em}
\begin{align*} 
N^{-1/2} \sum_{i=1}^{N} X_{i} \mbs{u}_{i} = \bigOp(1).
\end{align*}

Porém,

\vspace{-1 em}
\begin{align*} 
	\left( N^{-1} \sum_{i=1}^{N} X_{i}' X_{i} \right)^{-1} = (X'X/N)^{-1} = A^{-1} + \litop(1).
\end{align*}

\noindent
Sendo Assim,

\vspace{-1 em}
\begin{align*} 
\sqrt{N}(\betahatbold - \betabold) &= 
\left[ 
A^{-1} +
\left( N^{-1} \sum_{i=1}^{N} X_{i}' X_{i}   \right)^{-1}
- A^{-1}
\right]
\left( N^{-1/2} \sum_{i=1}^{N} X_{i}' \mbs{u}_{i}   \right)
\\ &=
A^{-1}\left( N^{-1/2} \sum_{i=1}^{N} X_{i}' \mbs{u}_{i}   \right)
+
[(X'X/N)^{-1} - A^{-1}]
\left( N^{-1/2} \sum_{i=1}^{N} X_{i}' \mbs{u}_{i}   \right)
\\&=
A^{-1}\left( N^{-1/2} \sum_{i=1}^{N} X_{i}' \mbs{u}_{i}   \right)
+ \litop(1) \bigOp(1)
\\&=
A^{-1}\left( N^{-1/2} \sum_{i=1}^{N} X_{i}' \mbs{u}_{i}   \right)
+ \litop(1)
\end{align*}

Portanto, com apenas \textit{single-equation OLS and 2SLS}, obtemos a representação assintótica para $\sqrt{N}(\betahatbold - \betabold)$ que é uma combinação linear não aleatória de somas parciais que satisfazem o CLT.
Usando o \red{lema de equivalência assintótica}, temos:

\begin{align*} 
\sqrt{N}(\betahatbold - \betabold)
\arrowd
N(\mbs{0}, A^{-1} B A^{-1})
\end{align*}

Resumimos esse resultado com o seguinte Teorema:

%%%%%%%%%%%%%%%%%%%%%%%%%%%%%%%%%%%%%%%%%%%%%%%%%%%%%%
\begin{teo1}[Normalidade Assintótica do SOLS]\label{teo:sols:norm}
Sob Hipóteses \textbf{SOLS.1} e \textbf{SOLS.2}, temos que a seguinte equação vale:

\begin{align} \label{eq:sols:norm}
	\Aboxed{
\sqrt{N}(\betahatbold - \betabold)
\arrowd
N(\mbs{0}, A^{-1} B A^{-1})
}.
\end{align}
\end{teo1}
%%%%%%%%%%%%%%%%%%%%%%%%%%%%%%%%%%%%%%%%%%%%%%%%%%%%%%

%%%%%%%%%%%%%%%%%%%%%%%%%%%%%%%%%%%%%%%%%%%%%%%%%%%%%%
\subsubsection{Variância Assintótica}

A variância assintótica de $\betahatbold^{SOLS}$ é:

\vspace{-1 em}
\begin{align}\label{eq:avar:sols}
	\Avar(\betahatbold^{SOLS}) = A^{-1} B A^{-1}/N.
\end{align}

Assim, $\Avar(\betahatbold^{SOLS})$ tende a zero a uma taxa $1/N$, como esperado.
Estimação consistente de $A$ é:

\vspace{-1 em}
\begin{align*}
	\widehat{A} \equiv X'X/N = N^{-1} \sum_{i=1}^{N} X_{i}'X_{i}
\end{align*}

Um estimador consistente para $B$ pode ser achado usando o princípio da analogia.

\vspace{-1 em}
\begin{align*}
B = \E(X_{i}' \mbs{u}_{i} \mbs{u}_{i}' X_{i}), 
\quad
N^{-1}\sum_{i=1}^{N} X_{i}' \mbs{u}_{i} \mbs{u}_{i}' X_{i} \arrowp B.
\end{align*}

Uma vez que não podemos observar $\mbs{u}_{i}$, usamos os resíduos da estimação de SOLS:

\vspace{-1 em}
\begin{align*}
\widehat{\mbs{u}}_{i} \equiv \mbs{y}_{i} - X_{i} \betahatbold 
=
\ubold_{i} - X_{i} (\betahatbold - \betabold).
\end{align*}

Assim, definimos $\Bhat$ e usando LGN, podemos mostrar que:

\vspace{-1 em}
\begin{align*}
\Bhat \equiv N^{-1}\sum_{i=1}^{N} X_{i}' \uhatbold_{i} \uhatbold_{i}' X_{i} 
\arrowp B.
\end{align*}

\noindent
onde supomos que certos momentos envolvendo $X_{i}$ e $\ubold_{i}$ são finitos.

Portanto, $\Avar[\sqrt{N}(\betahatbold - \betabold)]$ é \textbf{consistentemente} estimado por $\Ahat^{-1} \Bhat \Ahat^{-1}$, e $\Avar(\betahatbold)$ é estimado como:

\vspace{-1 em}
\begin{align*}
\Vhat \equiv 
\left( \sum_{i=1}^{N} X_{i}' X_{i}  \right)^{-1}
\left( \sum_{i=1}^{N} X_{i}' \uhatbold_{i} \uhatbold_{i}'  X_{i}  \right)
\left( \sum_{i=1}^{N} X_{i}' X_{i}  \right)^{-1}.
\end{align*}

Sob as hipóteses \textbf{SOLS.1} e \textbf{SOLS.2}, nós fazemos inferência em $\betabold$ como $\betahatbold$ fosse normalmente distribuído com média $\betabold$ e variância $\Vhat$.


%%%%%%%%%%%%%%%%%%%%%%%%%%%%%%%%%%%%%%%%%%%%%%%%%%%%%%
\clearpage
\section{SOLS para Dados de Painel}
\noindent
\citet[Sec.7.8 -- The Linear Panel Data Model, Revisited. p.169]{wool-2010} 

No caso de dados de painel:

\vspace{-1 em}
\begin{align*}
\sum_{i=1}^{N} X_{i}' X_{i}
=
\sum_{i=1}^{N} \sum_{t=1}^{T} \mbs{x}_{it}' \mbs{x}_{it};
\quad
\sum_{i=1}^{N} X_{i}' \mbs{y}_{i}
=
\sum_{i=1}^{N} \sum_{t=1}^{T} \mbs{x}_{it}' y_{it}.
\end{align*}

Portanto, podemos escrever $\betahatbold$ como:

\vspace{-1 em}
\begin{align} \label{betahat:POLS}
\Aboxed{
\betahatbold^{POLS} =
\left( \sum_{i=1}^{N} \sum_{t=1}^{T} \mbs{x}_{it}' \mbs{x}_{it} \right)^{-1}
\left( \sum_{i=1}^{N} \sum_{t=1}^{T} \mbs{x}_{it}' y_{it} \right)
}.
\end{align}

Este estimador é chamado \textbf{estimador de Mínimos Quadrados Agrupados (POLS)} porque ele corresponde a rodar uma regressão OLS nas observações agrupadas através de $i$ e $t$. 
% This estimator is called the \textbf{pooled ordinary least squares (POLS) estimator} because it corresponds to running OLS oin the observation pooled across $i$ and $t$.
O estimador da equação \eqref{betahat:POLS} é o mesmo para unidades de \textit{cross section} amostradas em diferentes pontos do tempo.
O Teorema \ref{SOLS:const}, abaixo, mostraa que o estimador POLS é consistente sob as condições de ortogonalidade na hipótese \red{XX} e uma hipótese de posto completo.

$\E(\mbs{x}_{it})$

$X_{i}' \mbs{u}_{i} = \sum_{t=1}^{T} \mbs{x}_{it}' u_{it}$




%%%%%%%%%%%%%%%%%%%%%%%%%%%%%%%%%%%%%%%%%%%%%%%%%%%%%%
\clearpage
\section{System GLS (SGLS)}

\noindent
\citet[Sec.7.4 -- Consistency and Asymptotic Normality of Generalized Least Squares, p.153]{wool-2010} 

%%%%%%%%%%%%%%%%%%%%%%%%%%%%%%%%%%%%%%%%%%%%%%%%%%%%%%
\subsection*{Hipóteses}

Para implementarmos o estimador de \textbf{GLS} precisamos das seguintes hipótese:

\begin{enumerate}
\item %1

$\E(X_{i} \otimes \mbs{u}_{i}) = 0$.

Para SGLS ser consistente, precisamos que $\mbs{u}_{i}$ não seja correlacionada com nenhum elemento de $X_{i}$.

\item %2

$\Omega$ é positiva definida (para ter inversa).
$\E(X_{i}^{\prime} \Omega^{-1} X_{i})$ é \textbf{não} singular (para ter invesa).

Onde, $\Omega$ é a seguinte matriz \textbf{simétrica}, positiva-definida:

\vspace{-1.5 em}
\begin{align*}
\Omega = \E(\mbs{u}_{i} \mbs{u}_{i}^{\prime}).
\end{align*}
\end{enumerate}

%%%%%%%%%%%%%%%%%%%%%%%%%%%%%%%%%%%%%%%%%%%%%%%%%%%%%%
\subsection*{Estimação}

Agora, transformamos o sistema de equações ao realizarmos a pré-multiplicação do sistema por $\Omega^{-1/2}$:

\vspace{-1.5 em}
\begin{align*}
\Omega^{-1/2} \mbs{y}_{i} 
&=
\Omega^{-1/2} X_{i} \mbs{\beta}
+
\Omega^{-1/2} \mbs{u}_{i}
\\
\mbs{y}_{i}^{*}
&=
X_{i}^{*} \mbs{\beta}
+
\mbs{u}^{*}_{i}
\end{align*}

Estimando a equação acima por \textbf{SOLS}:

\vspace{-1.5 em}
\begin{align*}
\beta^{SOLS}
&=
\left( \sum_{i=1} X_{i}^{*'} X_{i}^{*} \right)^{-1}
\left( \sum_{i=1} X_{i}^{*'} \mbs{y}_{i}^{*} \right)
\\
&=
\left( \sum_{i=1} X_{i}^{'} \Omega^{-1/2} \Omega^{-1/2} X_{i} \right)^{-1}
\left( \sum_{i=1} X_{i}^{'} \Omega^{-1/2} \Omega^{-1/2} \mbs{y}_{i} \right)
\\
&=
\left( \sum_{i=1} X_{i}^{'} \Omega^{-1} X_{i} \right)^{-1}
\left( \sum_{i=1} X_{i}^{'} \Omega^{-1} \mbs{y}_{i} \right)
\end{align*}


%%%%%%%%%%%%%%%%%%%%%%%%%%%%%%%%%%%%%%%%%%%%%%%%%%%%%%
\clearpage
\section{GLS Factível}

\noindent
\citet[Sec.7.5 -- Feasible GLS, p.153]{wool-2010} 

%%%%%%%%%%%%%%%%%%%%%%%%%%%%%%%%%%%%%%%%%%%%%%%%%%%%%%
\subsection*{FSGLS: SGLS Factível}

Para obtermos $\beta^{SGLS}$ precisamos conhecer $\Omega$, o que não ocorre na prática.
Então, precisamos estimar $\Omega$ com um estimador consistente.
Para tanto usamos um procedimento de dois passos:

\begin{enumerate}
\item  % Passo 1
Estimar $\mbs{y}_{i} = X_{i} \mbs{\beta} + \mbs{u}_{i}$ via \textbf{SOLS} e guardar o resíduo estimado $\widehat{\mbs{u}_{i}}$.

\item  %Passo 2
Estimar $\Omega$ com o seguinte estimador $\widehat{\Omega}$:

\vspace{-1.5 em}
\begin{align*}
	\widehat{\Omega} 
	= 
	N^{-1} \sum_{i=1}^{N} \mbs{u}_{i} \mbs{u}_{i}'
\end{align*}
\end{enumerate}

Com a estimativa $\widehat{\Omega}$ feita, podemos obter $\beta^{FSGLS}$ pela fórmula do $\beta^{SGLS}$:

\vspace{-1.5 em}
\begin{align*}
	\beta^{FGLS}
	= 
	\left[ 
		\sum_{i} X_{i}' \widehat{\Omega}^{-1} X_{i}
	\right]^{-1}
	\left[ 
		\sum_{i} X_{i}' \widehat{\Omega}^{-1} \mbs{y}_{i}
	\right]
\end{align*}

Empilhando as $N$ observações:

\vspace{-1.5 em}
\begin{align*}
\beta^{FGLS}
= 
\left[ X \left( I_{N} \otimes \widehat{\Omega}^{-1} \right) X \right]^{-1}
\left[ X' \left( I_{N} \otimes \widehat{\Omega}^{-1} \right) \mbs{y} \right]
\end{align*}

Reescrevendo a equação acima:

\vspace{-1.5 em}
\begin{align*}
\beta^{FGLS}
&= 
\left[  X \left( I_{N} \otimes \widehat{\Omega}^{-1} \right) X \right]^{-1}
\left[  X' \left( I_{N} \otimes \widehat{\Omega}^{-1} \right) (X \beta + u) \right]
\\
&= 
\left[ X \left( I_{N} \otimes \widehat{\Omega}^{-1} \right) X \right]^{-1}
\left\{ 
\left[ X' \left( I_{N} \otimes \widehat{\Omega}^{-1} \right) X \beta \right]
% \\
% &
\; +
\left[ X' \left( I_{N} \otimes \widehat{\Omega}^{-1} \right) u \right]
\right\}
\\
&= 
\beta +
\left[ X \left( I_{N} \otimes \widehat{\Omega}^{-1} \right) X \right]^{-1}
\left[ X' \left( I_{N} \otimes \widehat{\Omega}^{-1} \right) u \right]
\end{align*}

%%%%%%%%%%%%%%%%%%%%%%%%%%%%%%%%%%%%%%%%%%%%%%%%%%%%%%
\subsection*{Valor Esperado}

\vspace{-1 em}
\begin{align*}
\E(\beta^{FGLS})
= 
\beta +
\left[ X \left( I_{N} \otimes \widehat{\Omega}^{-1} \right) X \right]^{-1}
\left[ X' \left( I_{N} \otimes \widehat{\Omega}^{-1} \right) u \right]
\end{align*}

Concluímos que, se 
$\widehat{\Omega} \xrightarrow{\enskip p \enskip} \Omega$,
então,
$\beta^{FSGLS} \xrightarrow{\enskip p \enskip} \beta$,

%%%%%%%%%%%%%%%%%%%%%%%%%%%%%%%%%%%%%%%%%%%%%%%%%%%%%%
\subsection*{Variância}

\vspace{-1 em}
\begin{align*}
\Var(\beta^{FGLS})
&= 
\left[ X \left( I_{N} \otimes \widehat{\Omega}^{-1} \right) X \right]^{-1}
\left[ X' \left( I_{N} \otimes \widehat{\Omega}^{-1} \right) u \right]
\left\{ 
\left[ X \left( I_{N} \otimes \widehat{\Omega}^{-1} \right) X \right]^{-1}
\left[ X' \left( I_{N} \otimes \widehat{\Omega}^{-1} \right) u \right]
\right\}^{\prime}
\\
&=
\left[ X \left( I_{N} \otimes \widehat{\Omega}^{-1} \right) X \right]^{-1}
\left[
X' \left( I_{N} \otimes \widehat{\Omega}^{-1} \right) 
u u'
\left( I_{N} \otimes \widehat{\Omega}^{-1} \right) X
\right]
\left[ X \left( I_{N} \otimes \widehat{\Omega}^{-1} \right) X \right]^{-1}
\end{align*}

Tirando o valor Esperado e supondo que:

\vspace{-1.5 em}
\begin{align*}
\E(X_{i} \Omega^{-1} u_{i} u_{i}' X_{i}) = \E(X_{i} \Omega^{-1})
\end{align*}
temos:

\vspace{-1.5 em}
\begin{align*}
\E\left[ X' \left( I_{N} \otimes \widehat{\Omega}^{-1} \right)
	u u'
\left( I_{N} \otimes \widehat{\Omega}^{-1} \right)' X \right]
=
\E(X' \Omega^{-1} X)
\end{align*}
e temos:

\vspace{-1.5 em}
\begin{align*}
	\Var(\beta^{FSGLS}) = \left[ \E(X' \Omega^{-1} X \right]^{-1}.
\end{align*}


%%%%%%%%%%%%%%%%%%%%%%%%%%%%%%%%%%%%%%%%%%%%%%%%%%%%%%
\clearpage
\section{Basic Linear Unobserved Effects Panel Data Models}
\noindent
\citet[C.10, pp.247--291 ]{wool-2010}

%%%%%%%%%%%%%%%%%%%%%%%%%%%%%%%%%%%%%%%%%%%%%%%%%%%%%%
\begin{enumerate}[noitemsep, label*=10.\arabic*.]
\item  Motivation: The Omitted Variables Problem

\item  Assumptions about the Unobserved Effects and Explanatory Variables
\begin{enumerate}[noitemsep, leftmargin=1 em, label*=\arabic*.]
\item Random of Fixed Effects?
\item Strict Exogeneity Assumptions on the Explanatory Variables
\item Some Examples of Unobserved Effects Panel Data Models
\end{enumerate}

\item Estimating Unobserved Effects Models by Pooled OLS

\item \textbf{Random Effects Methods}
\begin{enumerate}[noitemsep, leftmargin=1 em, label*=\arabic*.]
\item Estimation and Inference under the Basic Random Effects Assumptions
\item Robust Variance Matrix Estimator
\item A General FGLS Analysis
\item Testing for the Presence of an Unobserved Effect
\end{enumerate}

\item \textbf{Fixed Effects Methods}
\begin{enumerate}[noitemsep, leftmargin=1 em, label*=\arabic*.]
\item Consistency of the Fixed Effects Estimator
\item Asymptotic Inference with Fixed Effects
\item The Dummy Variable Regression
\item Serial Correlation and the Robust Variance Matrix Estimator
\item Fixed Effects GLS
\item Using Fixed Effects Estimation for Policy Analysis
\end{enumerate}

\item \textbf{First Differencing Methods}
\begin{enumerate}[noitemsep, leftmargin=1 em, label*=\arabic*.]
\item Inference
\item Robust Variance Matrix
\item Testing for Serial Correlation
\item Policy Analysis Using First Differencing
\end{enumerate}

\item Comparison of Estimators
\begin{enumerate}[noitemsep, leftmargin=1 em, label*=\arabic*.]
\item Fixed Effects versus First Differencing
\item The Relationship between the Random Effects and Fixed Effects Estimators
\item The Hausman Test Comparing the RE and the FE Estimators
\end{enumerate}
\end{enumerate}



%%%%%%%%%%%%%%%%%%%%%%%%%%%%%%%%%%%%%%%%%%%%%%%%%%%%%%
\clearpage
\section{Endogeneity and GMM}


%%%%%%%%%%%%%%%%%%%%%%%%%%%%%%%%%%%%%%%%%%%%%%%%%%%%%%
\subsection*{Modelo}

No seguinte modelo \textit{cross-section}:

\vspace{-1 em}
\begin{align} \label{mod1}
	y_{i} = \beta_{0} + \beta_{1} x_{1i} + \beta_{2} x_{2i} + \err_{i}
	\; ; \quad i = 1, \dots, N.
\end{align}

\noindent
A variável explicativa $x_{k}$ é dita \textbf{endógena} se ela for correlacionada com erro.
Se $x_{k}$ for não correlacionada com o erro, então $x_{k}$ é dita \textbf{exógena}.

Endogeneidade surge, normalmente, de três maneiras diferentes:

\begin{enumerate}\itemsep0pt
	\item Variável Omitida;
	\item Simultaneidade;
	\item Erro de Medida.
\end{enumerate}

No modelo \eqref{mod1} vamos supor:

\begin{itemize}\itemsep0pt
	\item $x_{1}$ é exógena.
	\item $x_{2}$ é endógena.
\end{itemize}

%%%%%%%%%%%%%%%%%%%%%%%%%%%%%%%%%%%%%%%%%%%%%%%%%%%%%%
\subsection*{Hipóteses}

Assim, precisamos encontrar um instrumento $z_{i}$ para $x_{2}$, uma vez que queremos estimar $\beta_{0}$, $\beta_{1}$ e $\beta_{2}$ de maneira consistente.
Para $z_{i}$ ser um bom instrumento precisamos que $z$ tenha:

\begin{enumerate}\itemsep0pt
\item $Cov(z, \err) = 0$ $\implies$  $z$ é exógena em \eqref{mod1}.
\item $Cov(z, x_{2}) \neq 0$ $\implies$  correlação com $x_{2}$ após controlar para outras vaariáveis.
\end{enumerate}

%%%%%%%%%%%%%%%%%%%%%%%%%%%%%%%%%%%%%%%%%%%%%%%%%%%%%%
\subsection*{Estimação}

Indo para o problema de dados de painel, temos:

\vspace{-1 em}
\begin{align} \label{mod2}
	\mbs{y}_{i} = X_{i} \mbs{\beta} + \mbs{u}_{i}
	\; ; \quad i = 1, \dots, N.
\end{align}

\noindent
onde 
$\mbs{y}_{i}$ é um vetor $T \times 1$,
$X_{i}$ é uma matriz $T \times K$,
$\mbs{\beta}$ é o vetor de coeficientes $K \times 1$,
$\mbs{u}_{i}$ é o vetor de erros $T \times 1$.

Se é verdade que há endogeneidade em \eqref{mod2}, então:

\vspace{-1 em}
\begin{align*}
	\E(X_{i}^{\prime} \mbs{u}_{i}) \neq 0
\end{align*}

Definimos $Z_{i}$ como uma matriz $T \times L$ com $L \geq K$ de variáveis exógenas (incluindo o instrumento).
Queremos acabar com a endogeneidade, ou seja:

\vspace{-1 em}
\begin{align*}
	\E(Z_{i}^{\prime} \mbs{u}_{i}) = 0
\end{align*}

Supondo $L = K$ (apenas substituímos a variável endógena por um instrumento).

\vspace{-1 em}
\begin{align*}
\E[ Z_{i}^{\prime} ( \mbs{y}_{i} - X_{i} \mbs{\beta} ) ] &= 0
\\
\E( Z_{i}^{\prime} \mbs{y}_{i} ) - \E( Z_{i}^{\prime} X_{i} ) \mbs{\beta} &= 0
\\
\E( Z_{i}^{\prime} \mbs{y}_{i} ) &= \E( Z_{i}^{\prime} X_{i} ) \mbs{\beta}
\\
\Aboxed{
\mbs{\beta} &=
\left[ \E( Z_{i}^{\prime} X_{i} ) \right]^{-1}
\left[ \E( Z_{i}^{\prime} \mbs{y}_{i} ) \right]
}
\end{align*}

Se Usarmos estimadores amostrais:

\vspace{-1 em}
\begin{align*}
\mbs{\hat{\beta}} &=
\left[ N^{-1} \sum_{i=1}^{N} Z_{i}^{\prime} X_{i} \right]^{-1}
\left[ N^{-1} \sum_{i=1}^{N} Z_{i}^{\prime} \mbs{y}_{i} \right]
\\
\Aboxed{
\mbs{\hat{\beta}} &=
( Z^{\prime} X )^{-1} ( Z^{\prime} \mbs{y} ) }
\end{align*}

\vspace{1 em}
Se $L > K$, vamos considerar:

\vspace{-1 em}
\begin{align*}
\underset{\mbs{\beta}}{\text{Min}} \;
\E( Z_{i}\mbs{u}_{i} )^2
\end{align*}
\noindent onde:

\vspace{-1 em}
\begin{align*}
\E( Z_{i}\mbs{u}_{i} )^2 
&=
\E[ ( Z_{i}\mbs{u}_{i} )' ( Z_{i}\mbs{u}_{i} ) ]
=
( Z' \mbs{y} - Z' X \mbs{\beta} )' ( Z' \mbs{y} - Z' X \mbs{\beta} )
\\
&=
\mbs{y}' ZZ' \mbs{y}
-
\mbs{y}' ZZ' X \mbs{\beta}
-
\mbs{\beta}' X' ZZ' \mbs{y}
+
\mbs{\beta}' X' ZZ' X \mbs{\beta}
\end{align*}

Derivando em relação em $\mbs{\beta}$ e igualando a zero:

\vspace{-1 em}
\begin{align*}
-2 \mbs{y}' ZZ' X + 2 \mbs{\beta}'X' ZZ' X &= 0
\\
\mbs{\beta}'X' ZZ' X &= \mbs{y}' ZZ' X 
\\
\mbs{\beta}' &= ( \mbs{y}' ZZ' X ) ( X' ZZ' X )^{-1}
\\
\Aboxed{
\mbs{\beta} &= ( X' ZZ' X )^{-1} ( X' ZZ' \mbs{y} ) }
\end{align*}

Um estimador mais eficiente pode ser encontrado fazendo:

\vspace{-1 em}
\begin{align*}
\underset{\mbs{\beta}}{\text{Min}} \;
\E[ ( Z_{i}' \mbs{y} - Z' X \mbs{\beta} )' W ( Z_{i}' \mbs{y} - Z' X \mbs{\beta} ) ].
\end{align*}

\noindent
Escolhendo $\widehat{W}$, a priori, temos:

\vspace{-1 em}
\begin{align*}
\underset{\mbs{\beta}}{\text{Min}} \;
\left\{ 
\mbs{y}' Z \widehat{W} Z' \mbs{y}
-
\mbs{y}' Z \widehat{W} Z' X \mbs{\beta}
-
\mbs{\beta}' X'  Z \widehat{W} Z' \mbs{y}
+
\mbs{\beta}' X'  Z \widehat{W} Z' X \mbs{\beta}
\right\}
\end{align*}

Derivando em relação em $\mbs{\beta}$ e igualando a zero:

\vspace{-1 em}
\begin{align*}
-2 \mbs{y}' Z \widehat{W} Z' X + 2 \mbs{\beta}'X' Z \widehat{W} Z' X &= 0
\\
\mbs{\beta}'X' Z \widehat{W} Z' X &= \mbs{y}' Z \widehat{W} Z' X 
\\
\mbs{\beta}' &= ( \mbs{y}' Z \widehat{W} Z' X ) ( X' Z \widehat{W} Z' X )^{-1}
\\
\Aboxed{
\mbs{\beta}^{GMM} &= ( X' Z \widehat{W}' Z' X )^{-1} ( X' Z \widehat{W}' Z' \mbs{y} ) }
\end{align*}

%%%%%%%%%%%%%%%%%%%%%%%%%%%%%%%%%%%%%%%%%%%%%%%%%%%%%%
\subsection*{Valor Esperado} 

\vspace{-1 em}
\begin{align*}
\Aboxed{
\E( \mbs{\beta}^{GMM} ) &=
\mbs{\beta} +
\E[ ( X' Z \widehat{W}' Z' X )^{-1} ( X' Z \widehat{W}' Z' \mbs{u} ) ] }.
\end{align*}

%%%%%%%%%%%%%%%%%%%%%%%%%%%%%%%%%%%%%%%%%%%%%%%%%%%%%%
\subsection*{Variância} 

\vspace{-1 em}
\begin{align*}
\Var( \mbs{\beta}^{GMM} ) &=
\E \left\{ 
\left[ ( X' Z \widehat{W}' Z' X )^{-1} ( X' Z \widehat{W}' Z' \mbs{u} ) \right]
\left[ ( X' Z \widehat{W}' Z' X )^{-1} ( X' Z \widehat{W}' Z' \mbs{u} ) \right]'
\right\}
\\ &=
\E \left\{ 
( X' Z \widehat{W}' Z' X )^{-1}
X' Z \widehat{W}' Z' \mbs{u} \mbs{u}' Z \widehat{W} Z' X 
( X' Z \widehat{W} Z' X )^{-1}
\right\}.
\end{align*}

\noindent
Definindo $\Delta = \E(Z' \mbs{u}\mbs{u}' Z)$ com $\Delta = W^{-1}$:

\vspace{-1 em}
\begin{align*}
\Var( \mbs{\beta}^{GMM} ) &=
\E \left\{ 
( X' Z \widehat{W}' Z' X )^{-1}
X' Z \widehat{W}' W^{-1} \widehat{W} Z' X 
( X' Z \widehat{W} Z' X )^{-1}
\right\}
\\ &=
\E \left\{ 
( X' Z \widehat{W}' Z' X )^{-1}
( X' Z \widehat{W}' Z' X )
( X' Z \widehat{W} Z' X )^{-1}
\right\}.
\\
\Aboxed{
\Var( \mbs{\beta}^{GMM} ) &=
\E \left[
( X' Z \widehat{W} Z' X )^{-1}
\right] }.
\end{align*}

\noindent
Se tivéssemos definido $W = (Z'Z)^{-1}$, teríamos $\beta^{2SLS}$.

%%%%%%%%%%%%%%%%%%%%%%%%%%%%%%%%%%%%%%%%%%%%%%%%%%%%%%
\clearpage
\section{Random Effects (RE, EA)}


%%%%%%%%%%%%%%%%%%%%%%%%%%%%%%%%%%%%%%%%%%%%%%%%%%%%%%
\subsection*{Modelo}

O modelo linear de \textbf{efeitos não observados}:

\vspace{-1 em}
\begin{align} \label{mod1:EA}
	y_{it} = \mbs{x}_{it} \mbs{\beta} + c_{i} + u_{it},
\end{align}

\noindent
onde
$t = 1, \dots, T$ e $i = 1, \dots, N$.

O modelo contém explicitamente um componente não observado que não varia no tempo $c_{i}$.
Abordamos esse componente como parte do erro, não como parâmetro a ser estimado.
Para a análise de \textbf{Efeitos Aleatórios, (EA) ou (RE)}, supomos que os regressões $\mbs{x}_{it}$ são \textbf{não correlacionados} com $c_{i}$, mas fazemos hipóteses mais restritas que o \textbf{POLS}; pois assim exploramos a presença de \textbf{correlação serial} do erro composto por GLS e garantimos a consitência do estimador de FGLS.

Podemos reescrever \eqref{mod1:EA} como:

\vspace{-1 em}
\begin{align} \label{mod2:EA}
	y_{it} = \mbs{x}_{it} \mbs{\beta} + v_{it},
\end{align}

\noindent
onde
$t = 1, \dots, T$, $i = 1, \dots, N$ e $\boxed{v_{it} = c_{i} + u_{it}}$ é o erro composto.

Agora, vamos empilhar os $t$'s e reescrever \eqref{mod2:EA} como:

\vspace{-1 em}
\begin{align} \label{mod3:EA}
	\mbs{y}_{i} = X_{i} \mbs{\beta} + \mbs{v}_{i},
\end{align}

\noindent
onde
$i = 1, \dots, N$ e $\boxed{\mbs{v}_{i} = c_{i} \mbs{1}_{T} + \mbs{u}_{i}}$.

%%%%%%%%%%%%%%%%%%%%%%%%%%%%%%%%%%%%%%%%%%%%%%%%%%%%%%
\subsection*{Hipóteses de $\mbs{\widehat{\beta}}^{RE}$}

As Hipóteses que usamos para $\mbs{\widehat{\beta}}^{RE}$ são:

\begin{enumerate} \itemsep0pt
\item  
Usamos o modelo correto e $c_{i}$ não é endógeno.

\begin{enumerate}[label =\alph*)]
\item 
	$\E( u_{it} \, | \,  x_{i1}, \dots, x_{iT}, c_{i} ) = 0$,
	$i = 1, \dots, N$.
\item        
	$\E( c_{it} \, | \, x_{i1}, \dots, x_{iT} ) = \E( c_{i} ) = 0$,
	$i = 1, \dots, N$.
\end{enumerate}

\item  Posto completo de $\E( X_{i}' \Omega^{-1} X_{i} )$.

Definindo a matriz $T \times T$, $\boxed{\Omega \equiv \E(\mbs{v}_{i} \mbs{v}_{i}')}$, queremos que $\E( X_{i} \Omega^{-1} X_{i} )$ tenha posto completo (posto = $K$).
\end{enumerate}

A matriz $\Omega$ é simétrica $\Omega' = \Omega$ e positiva definida $\det(\Omega) > 0$.
Assim podemos achar $\Omega^{1/2}$ e $\Omega^{-1/2}$ com $\Omega = \Omega^{1/2} \Omega^{1/2}$ e $\Omega^{-1} = \Omega^{-1/2} \Omega^{-1/2}$.

%%%%%%%%%%%%%%%%%%%%%%%%%%%%%%%%%%%%%%%%%%%%%%%%%%%%%%
\subsection*{Estimação}

Premultiplicando \eqref{mod3:EA} port $\Omega^{-1/2}$ do dois lados, temos:

\vspace{-1 em}
\begin{align} 
\notag
\Omega^{-1/2}\mbs{y}_{i} &= \Omega^{-1/2}X_{i} \mbs{\beta} + \Omega^{-1/2}\mbs{v}_{i}
\\
\label{mod4:EA}
\mbs{y}_{i}^{*} &= X_{i}^{*} \mbs{\beta} + \mbs{v}_{i}^{*},
\end{align}

Estimando o modelo acima por POLS:

\vspace{-1 em}
\begin{align} 
\notag
\mbs{\beta}^{POLS} &= 
\left( \sum_{i=1}^{N} X_{i}^{*}' X_{i}^{*} \right)^{-1}
\left( \sum_{i=1}^{N} X_{i}^{*}' \mbs{y}_{i}^{*} \right)
\\ \notag
&=
\left( \sum_{i=1}^{N} X_{i}' \Omega^{-1} X_{i} \right)^{-1}
\left( \sum_{i=1}^{N} X_{i}' \Omega^{-1} \mbs{y}_{i} \right)
\\ \label{beta:RE:1}
&=
\left( X' (I_{N} \otimes \Omega^{-1}) X \right)^{-1}
\left( X' (I_{N} \otimes \Omega^{-1}) \mbs{y} \right).
\end{align}

O problema, agora, é estimar $\Omega$.
Supondo:
\begin{itemize}\itemsep0pt
\item $\E(u_{it}u_{it}) = \sigma_{u}^{2}$;
\item $\E(u_{it}u_{is}) = 0$.
\end{itemize}
Como $\Omega = \E(\mbs{v}_{i} \mbs{v}_{i}') = \E[ ( c_{i} \mbs{1}_{T} + \mbs{u}_{i} ) ( c_{i} \mbs{1}_{T} + \mbs{u}_{i} )' ]$, temos que:

\vspace{-1 em}
\begin{align*} 
\E(v_{it}v_{it}) &=
	\E( c_{i}^{2} + 2c_{i} u_{it} + u_{it}^{2}) 
	=
	\sigma_{c}^{2} + \sigma_{u}^{2}
\\
\E(v_{it}v_{is})	&=
	\E[ ( c_{i} + u_{it} ) ( c_{i} + u_{is} ) ]
	=
	\E( c_{i}^{2} + c_{i} u_{is} + u_{it} c_{i} + u_{it} u_{is} )
	=
	\sigma_{c}^{2}.
\end{align*}

Assim, 

\vspace{-1 em}
\begin{align*}
\Omega 
= 
\E(\mbs{v}_{i} \mbs{v}_{i}') = \sigma^{2}_{u} I_{T} + \sigma_{c}^{2} \mbs{1}_{T} \mbs{1}_{T}'
\end{align*}

\noindent
onde
$\sigma^{2}_{u} I_{T}$ 
é uma matriz diagonal, e 
$\sigma_{c}^{2} \mbs{1}_{T} \mbs{1}_{T}'$ é uma matriz com todos os elementos iguais a $\sigma_{c}^{2}$.

Agora, rodando POLS em \eqref{mod3:EA} e guardando os resíduos, temos:

\vspace{-1 em}
\begin{align*}
\hat{v}_{it}^{POLS}
= 
\hat{y}_{it}^{POLS} - \mbs{x}_{it} \mbs{\hat{\beta}}^{POLS}
\end{align*}

\noindent
e conseguimos estimar $\sigma_{v}^{2}$ e $\sigma_{c}^{2}$ por estimadores amostrais:

\begin{itemize}\itemsep0pt
\item 
como $\sigma_{v}^{2} = \E(v_{it}^{2})$:

\vspace{-1.5 em}
\begin{align*}
\hat{\sigma}_{v}^{2} =
(NT - K)^{-1} 
\sum_{i=1}^{N}
\sum_{t=1}^{T}
\hat{v}_{it}^2
\end{align*}
\vspace{-1.5 em}

\item 
como $\sigma_{c}^{2} = \E(v_{it} v_{is})$:

\vspace{-1.5 em}
\begin{align*}
\hat{\sigma}_{c}^{2} =
\left[ N \frac{T ( T-1 )}{2} - K  \right]^{-1}
\sum_{i=1}^{N}
\sum_{t=1}^{T-1}
\sum_{s=t+1}^{T}
\hat{v}_{it} \hat{v}_{is}
\end{align*}
\vspace{-1.5 em}

\item $N$ indivíduos;

\item $T$ elementos da diagonal principal de $\Omega$

\item $\frac{T ( T - 1)}{2}$ elementos da matriz triangular superior dos elementos fora da diagonal.

\item $K$ regressores.
\end{itemize}

Agora que temos $\hat{\sigma}^2_{v}$ e $\hat{\sigma}^2_{c}$ podemos achar $\hat{\sigma}^{2}_{u}$ pela equação $\boxed{\hat{\sigma}_{u}^{2} = \hat{\sigma}_{v}^{2} - \hat{\sigma}_{c}^{2}}$.
Dessa forma, achamos os $T^2$ elementos de $\widehat{\Omega}$, e podemos escrever:

\vspace{-1 em}
\begin{align*}
\widehat{\Omega}
= 
\hat{\sigma}^{2}_{u} I_{T} + \hat{\sigma}_{c}^{2} \mbs{1}_{T} \mbs{1}_{T}'
\end{align*}

Com $\widehat{\Omega}$ estimado, reescrevemos \eqref{beta:RE:1} como:

\vspace{-1 em}
\begin{align} \label{beta:RE:2}
\mbs{\beta}^{RE} = 
\left[ X' (I_{N} \otimes \widehat{\Omega}^{-1}) X \right]^{-1}
\left[ X' (I_{N} \otimes \widehat{\Omega}^{-1}) \mbs{y} \right].
\end{align}


%%%%%%%%%%%%%%%%%%%%%%%%%%%%%%%%%%%%%%%%%%%%%%%%%%%%%%
\subsection*{Valor Esperado}

\vspace{-1 em}
\begin{align*}
	\Aboxed{
\E( \mbs{\beta}^{RE} ) = 
\mbs{\beta} +
\left[ X' (I_{N} \otimes \widehat{\Omega}^{-1}) X \right]^{-1}
\left[ X' (I_{N} \otimes \widehat{\Omega}^{-1}) \mbs{v} \right] }.
\end{align*}

%%%%%%%%%%%%%%%%%%%%%%%%%%%%%%%%%%%%%%%%%%%%%%%%%%%%%%
\subsection*{Variância}

\vspace{-1 em}
\begin{align*} 
\Var( \mbs{\beta}^{RE} ) = 
E
\left\{ 
\left[ X' ( I_{N} \otimes \widehat{\Omega}^{-1} ) X \right]^{-1}
\left[
X' ( I_{N} \otimes \widehat{\Omega}^{-1} )
\mbs{v} \mbs{v}'
( I_{N} \otimes \widehat{\Omega}^{-1} )' X
\right]
\left[ X' ( I_{N} \otimes \widehat{\Omega}^{-1} ) X \right]
\right\},
\end{align*}

\noindent
como $\E( \mbs{v}_{i} \mbs{v}_{i}' ) =\Omega$,

\vspace{-1 em}
\begin{align*} 
	\Aboxed{
\Var( \mbs{\beta}^{RE} ) = 
E
\left[ X' ( I_{N} \otimes \widehat{\Omega}^{-1} ) X \right] }.
\end{align*}

%%%%%%%%%%%%%%%%%%%%%%%%%%%%%%%%%%%%%%%%%%%%%%%%%%%%%%
\clearpage
\section{Fixed Effects (EF, FE)}


%%%%%%%%%%%%%%%%%%%%%%%%%%%%%%%%%%%%%%%%%%%%%%%%%%%%%%
\subsection*{Modelo}

O modelo linear de \textbf{efeitos não observados}:

\vspace{-1 em}
\begin{align} \label{mod1:FE}
	y_{it} = \mbs{x}_{it} \mbs{\beta} + c_{i} + u_{it},
\end{align}

\noindent
onde
$t = 1, \dots, T$ e $i = 1, \dots, N$.

O modelo contém explicitamente um componente não observado que não varia no tempo $c_{i}$.
Abordamos esse componente como parte do erro, não como parâmetro a não observado.
No caso da análise de \textbf{Efeitos Fixos (EF, FE)}, permitimos que esse componente $c_{i}$ seja correlacionado com $\mbs{x}_{it}$.
Assim, se decidíssemos estimar o modelo \eqref{mod1:FE} por POLS, ignorando $c_{i}$, teríamos problemas de inconsistência devido a \textbf{endogeneidade}.

As $T$ equações do modelo \eqref{mod1:FE} podem ser reescritas como:

\vspace{-1 em}
\begin{align} \label{mod2:FE}
	\mbs{y}_{i} = X_{i} \mbs{\beta} + c_{1} \mbs{1}_{T} + \mbs{u}_{i},
\end{align}

\noindent
com
$\mbs{v}_{i} = c_{i} \mbs{1}_{T} + \mbs{u}_{i}$ sendo os erros compostos.

%%%%%%%%%%%%%%%%%%%%%%%%%%%%%%%%%%%%%%%%%%%%%%%%%%%%%%
\subsection*{Matriz $M^{0}$}

Definimos a matriz $M^{0}$ como:

\vspace{-1 em}
\begin{align*}
	M^{0} &=
	I_{T} - T^{-1} \mbs{1}_{T} \mbs{1}_{T}'
	=
	I_{T} - \mbs{1}_{T} (\mbs{1}_{T}' \mbs{1}_{T})^{-1} \mbs{1}_{T}'.
\end{align*}

\noindent
A matriz $M^{0}$ é idempotente e simétrica.

\begin{align*}
	M^{0} \mbs{x} &= \mbs{x} - \overline{\mbs{x}} \mbs{1}_{T}
	= \ddot{\mbs{x}}.
\end{align*}

Podemos transformar o modelo \eqref{mod2:FE} ao premultiplicarmos todo o modelo por $M^{0}$.

\vspace{-1 em}
\begin{align*} 
M^{0} \mbs{y}_{i} &= M^{0} X_{i} \mbs{\beta} + M^{0} ( c_{1} \mbs{1}_{T} ) + M^{0} \mbs{u}_{i},
\quad i = 1, \dots, N.
\end{align*}

\vspace{-1 em}
\begin{align*} 
M^{0} ( c_{1} \mbs{1}_{T} ) = 
( I_{T} - T^{-1} \mbs{1}_{T} \mbs{1}_{T}' ) c_{i} \mbs{1}_{T} 
=
c_{i} \mbs{1}_{T} - T^{-1} c_{i} \mbs{1}_{T} \mbs{1}_{T}' \mbs{1}_{T} 
=
c_{i} \mbs{1}_{T} - c_{i} \mbs{1}_{T} 
\implies
\boxed{ M^{0} ( c_{1} \mbs{1}_{T} ) = 0 }
\end{align*}

\vspace{-1 em}
\begin{align} \label{mod2:FE}
\ddot{\mbs{y}}_{i} &= \ddot{X}_{i} \mbs{\beta} + \ddot{\mbs{u}_{i}},
\quad i = 1, \dots, N.
\end{align}

%%%%%%%%%%%%%%%%%%%%%%%%%%%%%%%%%%%%%%%%%%%%%%%%%%%%%%
\subsection*{Estimação POLS}

Aplicando POLS no modelo \eqref{mod2:FE}

\vspace{-1 em}
\begin{align} \label{beta:pols:FE}
\Aboxed{
\mbs{\beta}^{FE} =
\left[ \sum_{i=1}^{N} \ddot{X}_{i}' \ddot{X}_{i} \right]^{-1}
\left[ \sum_{i=1}^{N} \ddot{X}_{i}' \ddot{\mbs{y}}_{i} \right]
}
\end{align}

%%%%%%%%%%%%%%%%%%%%%%%%%%%%%%%%%%%%%%%%%%%%%%%%%%%%%%
\subsection*{Hipóteses}

As Hipóteses que usamos para $\mbs{\widehat{\beta}}^{FE}$ são:

\begin{description}
\setlength\itemsep{.5 em}

\item [FE.1:] Exogeneidade Estrita:
$\E( u _{it} \, | \, \mbs{x}_{i1}, \dots, \mbs{x}_{iT}, c_{i}) = 0$, para $t=1, \dots, T$ e $i = 1, \dots, N$.

\item  [FE.2:] Posto completo de $\E( X_{i}' \Omega^{-1} X_{i} )$ (para inverter a matriz).
$posto[ \E( X_{i}' \Omega^{-1} X_{i} ) ]  = K$.

\item [FE.3:] Homoscedasticidade:
	$\E(\mbs{u}_{i} \mbs{u}_{i}' \,|\, X_{i}, c_{i}) = \sigma_{u}^{2} I_{T}$.
\end{enumerate}

%%%%%%%%%%%%%%%%%%%%%%%%%%%%%%%%%%%%%%%%%%%%%%%%%%%%%%
\subsection*{Valor Esperado}
Usando \textbf{FE.1} e \textbf{FE.2}, apenas.

\vspace{-1 em}
\begin{align*}
\E( \mbs{\beta}^{FE} ) &=
\mbs{\beta} +
\E\left[
\left( \sum_{i=1}^{N} \ddot{X}_{i}' \ddot{X}_{i} \right)^{-1}
\left( \sum_{i=1}^{N} \ddot{X}_{i}' \ddot{\mbs{u}}_{i} \right)
\right]
\\
\Aboxed{
\E( \mbs{\beta}^{FE} ) &=
\mbs{\beta} +
\E \left[ ( \ddot{X}' \ddot{X} )^{-1} (\ddot{X}' \ddot{\mbs{u}}) \right]
}
\end{align*}

\noindent
Sabendo que 
$\ddot{X} = ( I_{N} \otimes M^{0} ) X$
e
$\ddot{\mbs{u}} = ( I_{N} \otimes M^{0} ) \mbs{u}$,
definimos:

\vspace{-1 em}
\begin{align*}
\E( \mbs{\beta}^{FE} ) &=
\mbs{\beta} +
\E \left\{
\left[  
X' ( I_{N} \otimes M^{0} )( I_{N} \otimes M^{0} ) X 
\right]^{-1}
\left[ 
X' ( I_{N} \otimes M^{0} )( I_{N} \otimes M^{0} ) \mbs{u}
\right]
\right\}
\\
\Aboxed{
\E( \mbs{\beta}^{FE} ) &=
\mbs{\beta} +
\E \left\{
\left[  
X' ( I_{N} \otimes M^{0} ) X 
\right]^{-1}
\left[ 
X' ( I_{N} \otimes M^{0} ) \mbs{u}
\right]
\right\} }
\end{align*}

%%%%%%%%%%%%%%%%%%%%%%%%%%%%%%%%%%%%%%%%%%%%%%%%%%%%%%
\subsection*{Variância}

Usamos a variância do estimador para inferência.
Usando \textbf{FE.1} e \textbf{FE.2}, apenas:

\vspace{-1 em}
\begin{align*} 
	\Aboxed{
\Var( \mbs{\beta}^{FE} ) = 
\E \left[
( \ddot{X}' \ddot{X} )^{-1}
(\ddot{X}' \ddot{\mbs{u}}) (\ddot{\mbs{u}}' \ddot{X} )
( \ddot{X}' \ddot{X} )^{-1} 
\right]}
\end{align*}


\begin{description}
\item [Pão:]
\begin{align*}
\E\left[ ( \ddot{X}' \ddot{X} )^{-1} \right] &=
\E \left\{ \left[
X' ( I_{N} \otimes M^{0} ) ( I_{N} \otimes M^{0} ) X
\right]^{-1} \right\}
\\ &=
\E \left\{ \left[
X' ( I_{N} \otimes M^{0} ) X
\right]^{-1} \right\}
\end{align*}

\item [Recheio:]
\begin{align*}
\E\left[
(\ddot{X}' \ddot{\mbs{u}}) (\ddot{\mbs{u}}' \ddot{X} ) 
\right] 
&=
\E \left[
X' ( I_{N} \otimes M^{0} ) ( I_{N} \otimes M^{0} ) 
\mbs{u} \mbs{u}'
( I_{N} \otimes M^{0} ) ( I_{N} \otimes M^{0} ) X
\right]
\\ &=
\E \left[
X' ( I_{N} \otimes M^{0} ) \mbs{u} \mbs{u}' ( I_{N} \otimes M^{0} ) X
\right]
\end{align*}
\end{description}

\vspace{-1 em}
\begin{align*} 
\Var( \mbs{\beta}^{FE} ) &= \text{Pão Recheio Pão}
\\ 
\Aboxed{
\Var( \mbs{\beta}^{FE} ) &= 
\E \left\{ \left[
X' ( I_{N} \otimes M^{0} ) X
\right]^{-1} \right\}
\E \left[
X' ( I_{N} \otimes M^{0} ) \mbs{u} \mbs{u}' ( I_{N} \otimes M^{0} ) X
\right]
\E \left\{ \left[
X' ( I_{N} \otimes M^{0} ) X
\right]^{-1} \right\}
}
\end{align*}

%%%%%%%%%%%%%%%%%%%%%%%%%%%%%%%%%%%%%%%%%%%%%%%%%%%%%%
\subsection*{Variância sob Homocedasticidade}

Usando \textbf{FE.3}, temos

\begin{description}
\item [Recheio':]
\begin{align*}
\E \left[ X' ( I_{N} \otimes M^{0} ) \right]
\sigma^2_{u} I_{NT}
\E \left[ ( I_{N} \otimes M^{0} ) X \right]
=
\sigma^2_{u}
\E \left[ X' ( I_{N} \otimes M^{0} ) X \right]
\end{align*}
\end{description}

\noindent
\red{ $( I_{N} \otimes M^{0} )$ é uma matrix de dimensão $NT \times NT$, visto que $I_{N}$ é $N\times N$ e $M^{0}$ é $T \times T$.}

\vspace{-2 em}
\begin{align*}
\Var( \mbs{\beta}^{FE} ) &= \text{Pão Recheio' Pão} 
\\ &=
\E \left\{ \left[
X' ( I_{N} \otimes M^{0} ) X
\right]^{-1} \right\}
\sigma_{u}^{2} \E\left[ X' ( I_{N} \otimes M^{0} ) X \right]
\E \left\{ \left[
X' ( I_{N} \otimes M^{0} ) X
\right]^{-1} \right\}
\\  &=
\E \left\{ \left[
X' ( I_{N} \otimes M^{0} ) X
\right]^{-1} \right\}
\sigma_{u}^{2} I_{NT}
\\
\Aboxed{ \Var( \mbs{\beta}^{FE} ) &= \sigma_{u}^{2} \cdot  \E\left[ X' ( I_{N} \otimes M^{0} ) X \right] }
\end{align*}


%%%%%%%%%%%%%%%%%%%%%%%%%%%%%%%%%%%%%%%%%%%%%%%%%%%%%%
\clearpage
\section{First Difference (FD, PD)}


%%%%%%%%%%%%%%%%%%%%%%%%%%%%%%%%%%%%%%%%%%%%%%%%%%%%%%
\subsection*{Modelo}

O modelo linear de \textbf{efeitos não observados}:

\vspace{-1 em}
\begin{align} \label{mod1:FD}
	y_{it} = \mbs{x}_{it} \mbs{\beta} + c_{i} + u_{it},
\end{align}

\noindent
para
$t = 1, \dots, T$ e $i = 1, \dots, N$.

O modelo contém explicitamente um componente não observado, $c_{i}$, que não varia no tempo.
Tratamos o componente não observado como parte do erro, não como parâmetro a ser estimado.
Aqui permitimos que $c_{i}$ seja correlacionado com $\mbs{x}_{it}$.
Deste modo, \textbf{não} podemos ignorar a sua presença e estimar \eqref{mod1:FD} por POLS, visto que isso resultaria num estimador inconsistente devido a \textbf{endogeneidade}.

Assim, transformamos o modelo para eliminar $c_{i}$ e conseguirmos fazer uma estimação consistente de $\mbs{\beta}$.
A trasnformação a ser feita é a primeira diferença.
Para tanto, seguimos os seguintes passos:

\begin{itemize}\itemsep0pt
\item Reescrevemos \eqref{mod1:FD} defasado:

\vspace{-1 em}
\begin{align}  \label{mod2:FD}
	y_{it-1} = \mbs{x}_{it-1} \mbs{\beta} + c_{i} + u_{it-1}
\end{align}

\item Tiramos a diferença entre \eqref{mod2:FD} e \eqref{mod1:FD}:

\vspace{-1 em}
\begin{align}
\nonumber
y_{it} - y_{it-1} &=
(\mbs{x}_{it} - \mbs{x}_{it-1}) \mbs{\beta} +
c_{i} - c_{i} +
u_{it} - u_{it-1}
\\
\label{mod3:FD}
\Delta y_{it} &=
\Delta \mbs{x}_{it} \mbs{\beta} +
\Delta u_{it}. 
\end{align}

\noindent
para
$t = 2, \dots, T$ e $i = 1, \dots, N$.
\end{itemize}

Reescrevendo \eqref{mod3:FD} no formato matricial empilhando $T$:

\vspace{-1 em}
\begin{align} \label{mod4:FD}
	\Delta \mbs{y}_{i} = \Delta X_{i} \mbs{\beta} + \mbs{e}_{i}
\end{align}

\noindent
com 
$\boxed{e_{it} = \Delta u_{it}}$.

\begin{itemize}\itemsep0pt
\item
$\Delta \mbs{y}_{i}$ vetor $( T - 1 ) \times 1$ 
\item
$\Delta X_{i}$  matriz  $( T - 1 ) \times K$
\item
$\mbs{\beta}$ vetor $K \times 1$
\item
$\mbs{e}_{i}$ vetor $(T - 1 ) \times 1$
\end{itemize}


%%%%%%%%%%%%%%%%%%%%%%%%%%%%%%%%%%%%%%%%%%%%%%%%%%%%%%
\subsection*{Estimação POLS}

O estimador $\widehat{\mbs{\beta}}^{FD}$ é o POLS da regressão no modelo \eqref{mod4:FD}, assim:

\vspace{-1 em}
\begin{align} \label{beta:pols:FD}
\Aboxed{
\mbs{\beta}^{FD} =
\left[ \sum_{i=1}^{N} \Delta X_{i}' \Delta X_{i} \right]^{-1}
\left[ \sum_{i=1}^{N} \Delta X_{i}' \Delta \mbs{y}_{i} \right]
}
\end{align}

%%%%%%%%%%%%%%%%%%%%%%%%%%%%%%%%%%%%%%%%%%%%%%%%%%%%%%
\subsection*{Hipóteses}

As Hipóteses que usamos para $\mbs{\widehat{\beta}}^{FD}$ são:

\begin{description}
\setlength\itemsep{.5 em}

\item [FD.1:] Exogeneidade Estrita:
$\E( u _{it} \, | \, \mbs{x}_{i1}, \dots, \mbs{x}_{iT}, c_{i}) = 0$, para $t=1, \dots, T$ e $i = 1, \dots, N$.

\item  [FD.2:] Posto completo de $\E( \Delta X_{i}' \Delta X_{i} )$ (para inverter a matriz).
$posto[ \E( \Delta X_{i} ' \Delta X_{i} ) ]  = K$.

\item [FD.3:] Homoscedasticidade:
	$\E(\mbs{e}_{i} \mbs{e}_{i}' \,|\, X_{i}, c_{i}) = \sigma_{e}^{2} I_{T-1}$.
\end{enumerate}

%%%%%%%%%%%%%%%%%%%%%%%%%%%%%%%%%%%%%%%%%%%%%%%%%%%%%%
\subsection*{Valor Esperado}
Usando apenas \textbf{FD.1} e \textbf{FD.2}:

\vspace{-1 em}
\begin{align*}
\E( \mbs{\beta}^{FD} ) &=
\mbs{\beta} +
\E\left[
\left( \sum_{i=1}^{N} \Delta X_{i}' \Delta X_{i} \right)^{-1}
\left( \sum_{i=1}^{N} \Delta X_{i}' \mbs{e}_{i} \right)
\right]
\\
\Aboxed{
\E( \mbs{\beta}^{FD} ) &=
\mbs{\beta} +
\E \left[ ( \Delta X' \Delta X )^{-1} ( \Delta X' \mbs{e} \right]
}
\end{align*}

%%%%%%%%%%%%%%%%%%%%%%%%%%%%%%%%%%%%%%%%%%%%%%%%%%%%%%
\subsection*{Variância}

Usando apenas \textbf{FD.1} e \textbf{FD.2}:

\vspace{-1 em}
\begin{align*} 
\Aboxed{
\Var( \mbs{\beta}^{FD} ) = 
\E \left[
( \Delta X' \Delta X )^{-1}
( \Delta X' \mbs{e}  \mbs{e}' \Delta X )
( \Delta X' \Delta X )^{-1} 
\right]}
\end{align*}

%%%%%%%%%%%%%%%%%%%%%%%%%%%%%%%%%%%%%%%%%%%%%%%%%%%%%%
\subsection*{Variância sob Homocedasticidade}

Usando \textbf{FD.3}, temos

\vspace{-1 em}
\begin{align*} 
\Var( \mbs{\beta}^{FD} ) &= 
\sigma_{e}^{2}
\E \left[
( \Delta X' \Delta X )^{-1}
( \Delta X' \Delta X )
( \Delta X' \Delta X )^{-1} 
\right]
\\
\Aboxed{
\Var( \mbs{\beta}^{FD} ) &= 
\sigma^2_{e}
\E \left[
( \Delta X' \Delta X )^{-1} 
\right]}
\end{align*}

\noindent 
com

\vspace{-1 em}
\begin{align*} 
\sigma^2_{e} = 
\left[ N ( T - 1 ) - K \right]^{-1}
\left[  
\sum_{i=1}^{N} 
\sum_{t=1}^{T}
\hat{e}_{it}^{2}
\right],
\end{align*}

\noindent
que é a média de todos $\hat{e}^{2}_{it}$ contando $K$ regressores.

%%%%%%%%%%%%%%%%%%%%%%%%%%%%%%%%%%%%%%%%%%%%%%%%%%%%%%
\clearpage
\section{Exogeneidade Estrita e FDIV}


%%%%%%%%%%%%%%%%%%%%%%%%%%%%%%%%%%%%%%%%%%%%%%%%%%%%%%
\subsection*{Modelo}

No seguinte modelo

\vspace{-1 em}
\begin{align*} 
	y_{it} = \mbs{x}_{it} \mbs{\beta} + u_{it},
\end{align*}

\noindent
para
$t = 1, \dots, T$ e $i = 1, \dots, N$.

\begin{itemize}\itemsep0pt
\item
$y_{it}$ escalar;

\item
$\mbs{x}_{it}$  vetor $1 \times K$;

\item
$\mbs{\beta}$ vetor $K \times 1$;

\item
$u_{it}$ escalar.
\end{itemize}

\noindent
$\{x_{it}\}$ é estritamente \textbf{exógeno} se valer:

\vspace{-1 em}
\begin{align*}
	\E( u_{it} \, | \, \mbs{x}_{i1}, \dots, \mbs{x}_{iT} ) = 0 \; , \qquad t = 1, \dots, T
\end{align*}

\noindent
ou seja:

\vspace{-1 em}
\begin{align*}
\E( y_{it} \, | \, \mbs{x}_{i1}, \dots, \mbs{x}_{iT} ) = \mbs{x}_{it} \mbs{\beta} 
\; , \qquad t = 1, \dots, T
\end{align*}

\noindent
o que é equivalente a hipótese de que utilizamos o modelo linear correto.

Para o seguinte modelo:

\vspace{-1.5 em}
\begin{align*}
y_{it} = \mbs{z}_{it} \mbs{\gamma} + \rho y_{i t - 1} + c_{i} + u_{it}.
\; , \qquad t = 2, \dots, T
\end{align*}

\noindent
é \textbf{impossível} termos exogeneidade estrita.
Isso porque, nesse modelo, de efeitos não observados temos:

\vspace{-1.5 em}
\begin{align*}
	\E( y_{it} \, | \, \mbs{z}_{i1}, \dots, \mbs{z}_{iT}, y_{it-1}, c_{i}) \neq 0.
\end{align*}

\noindent
Isso ocorre porque, $y_{it}$ é afetado por $y_{it-1}$ que contribui para $y_{it}$ com, pelo menos, $\rho c_{i}$.

\begin{equation*}
\left.
\begin{aligned}
y_{it} &= \mbs{z}_{it} \mbs{\gamma} + \rho y_{i t - 1} + c_{i} + u_{it}
\\
y_{it-1} &= \mbs{z}_{it-1} \mbs{\gamma} + \rho y_{i t - 2} + c_{i} + u_{it-1}
\end{aligned}
\right\} 
\implies
y_{it} = \mbs{z}_{it} \mbs{\gamma} +
\rho (\mbs{z}_{it-1} \mbs{\gamma} + \rho y_{i t - 2} + c_{i} + u_{it-1})
+ c_{i} + u_{it}.
\end{equation*}

Para eliminarmos este efeito, podemos tirar a primeira diferença do modelo:

\vspace{-1 em}
\begin{align}
\nonumber
y_{it} - y_{it-1} &= 
(\mbs{z}_{it} - \mbs{z}_{it-1}) \mbs{\gamma} +
\rho (y_{i t - 1} -  y_{i t - 2} ) +
(c_{i} - c_{i}) + (u_{it} - u_{it-1})
\\
\label{mod1:FDIV}
\Aboxed{
\Delta y_{it} &= 
\Delta \mbs{z}_{it} \mbs{\gamma} + \rho \Delta y_{i t - 1} + \Delta u_{it}
\, , \qquad t=3, \dots, T}
\end{align}

%%%%%%%%%%%%%%%%%%%%%%%%%%%%%%%%%%%%%%%%%%%%%%%%%%%%%%
\subsection*{Estimação}

Não podemos estimar o modelo \eqref{mod1:FDIV} por POLS, uma vez que $Cov(\Delta y_{it-1}, \Delta u_{it} ) \neq 0$.
Como saída, podemos estimar por P2SLS, usando instrumentos para $\Delta y_{it-1}$ (alguns intrumentos para $\Delta y_{it-1}$ são $y_{it-2}, y_{it-3}, \dots, y_{i1}$).

%%%%%%%%%%%%%%%%%%%%%%%%%%%%%%%%%%%%%%%%%%%%%%%%%%%%%%
\subsubsection*{P2SLS}

\vspace{-1 em}
\begin{align*}
	y_{it} = \mbs{x}_{it}' \mbs{\beta} + u_{it}
\end{align*}

\begin{itemize}\itemsep0pt
\item $i = 1, \dots, N$
\item $t = 1, \dots, T$
\item $y_{it}$ escalar;
\item $\mbs{x}_{it}$  vetor $K \times 1$;
\item $\mbs{\beta}$ vetor $K \times 1$;
\item $u_{it}$ escalar.
\end{itemize}

\vspace{-1 em}
\begin{align*}
\boxed{
\mbs{\beta}^{P2SLS} =  ( X' P_{Z} X )^{-1} ( X' P_{Z} \mbs{y} ) }
\end{align*}

\noindent
com

\vspace{-1 em}
\begin{align*}
\boxed{P_{Z} = Z'(Z'Z)^{-1}Z }
\end{align*}

\noindent
onde
$P_{Z}$ é a matriz de projeção em $Z$.

%%%%%%%%%%%%%%%%%%%%%%%%%%%%%%%%%%%%%%%%%%%%%%%%%%%%%%
\subsubsection*{FDIV}

\vspace{-1 em}
\begin{align*}
y_{it} &= \mbs{x}_{it}' \mbs{\beta} + c_{i} + u_{it}
\; , \quad i = 1, \dots, N
\; , \quad t = 1, \dots, T
\\
\Delta y_{it} &= \Delta \mbs{x}_{it}' \mbs{\beta} + \Delta u_{it}
\; , \quad i = 1, \dots, N
\; , \quad t = 2, \dots, T
\end{align*}

Vamos supor $\Delta x_{it}'$ tem variável endógena ($y_{it}$, no caso).
$\mbs{w}_{it}$ é um vetor $1 \times L_{t}$ de instrumentos, onde $L_{t} \geq K$.
Se os instrumentos forem diferentes:

\vspace{-1 em}
\begin{align*}
	W_{i} = diag( \mbs{w}_{i2}', \mbs{w}_{i3}', \dots, \mbs{w}_{iT}')
\end{align*}

\noindent
onde $W_{i}$ é uma matriz $( T - 1 ) \times L$

\vspace{-1 em}
\begin{align*}
	L = L_{2} + L_{3} + \dots + L_{T}
\end{align*}

%%%%%%%%%%%%%%%%%%%%%%%%%%%%%%%%%%%%%%%%%%%%%%%%%%%%%%
\subsection*{Hipóteses}

\begin{description}
\item[FDIV.1:] $\E( \mbs{w}_{it} \Delta u_{it}')$ para $i = 1, \dots, N$, $t = 2, \dots, T$.
\item[FDIV.2:] $Posto\left[ \E( W_{i}' W_{i} ) \right] = L$
\item[FDIV.3:] $Posto\left[ \E( W_{i}' \Delta X_{i} ) \right] = K$
\end{description}

%%%%%%%%%%%%%%%%%%%%%%%%%%%%%%%%%%%%%%%%%%%%%%%%%%%%%%
\subsection*{Estimação FDIV}

\vspace{-1 em}
\begin{align*}
\boxed{
\mbs{\beta}^{FDIV} =  
\left(
\Delta X' P_{W} \Delta X 
\right)^{-1}
\left(
\Delta X' P_{W} \Delta \mbs{y}
\right)
}
\qquad
\boxed{
P_{W} = W(W'W)^{-1}W'}
\end{align*}

%%%%%%%%%%%%%%%%%%%%%%%%%%%%%%%%%%%%%%%%%%%%%%%%%%%%%%
\subsection*{Valor Esperado}

\vspace{-1 em}
\begin{align*}
\E( \mbs{\beta}^{FDIV} ) =  
\beta + 
\left( \Delta X' P_{W} \Delta X \right)^{-1}
\left( \Delta X' P_{W} \mbs{e} \right)
\end{align*}

%%%%%%%%%%%%%%%%%%%%%%%%%%%%%%%%%%%%%%%%%%%%%%%%%%%%%%
\subsection*{Variância}

\vspace{-1 em}
\begin{align*}
\Var( \mbs{\beta}^{FDIV} ) &=
\E\left\{  
\left[ \E( \mbs{\beta}^{FDIV} ) - \beta \right] 
\left[ \E( \mbs{\beta}^{FDIV} ) - \beta \right]'
\right\}
\\
&=
\E\left\{  
\left[ \Delta X' P_{W} \Delta X \right]^{-1}
\left[ \Delta X' P_{W} \mbs{e} \right]
\left[ \Delta X' P_{W} \mbs{e} \right]'
\left[ \Delta X' P_{W} \Delta X \right]^{-1}
\right\}
\\
&=
\E\left[
\left( \Delta X' P_{W} \Delta X \right)^{-1}
\left( \Delta X' P_{W} \mbs{e} \mbs{e}' P_{W} \Delta X \right)
\left( \Delta X' P_{W} \Delta X \right)^{-1}
\right]
\end{align*}

\noindent
$e_{i} = \Delta u_{it}$.

%%%%%%%%%%%%%%%%%%%%%%%%%%%%%%%%%%%%%%%%%%%%%%%%%%%%%%
\clearpage
\section{Latent Variables, Probit and Logit}

%%%%%%%%%%%%%%%%%%%%%%%%%%%%%%%%%%%%%%%%%%%%%%%%%%%%%%
\subsection*{Modelo}

Suponha $y^{*}$ não observável (\textbf{latente}) seguindo o seguinte modelo:

\vspace{-1 em}
\begin{align} \label{mod1:probit}
	y_{i}^{*} = \mbs{x}_{i}' \mbs{\beta} + \err_{i}.
\end{align}

\noindent
Defina $y$ como:

\vspace{-1 em}
\begin{align*}
y_{i} =
\begin{cases}
	1 \, , \quad y^{*}_{i} \geq 0
\\
	0 \, , \quad y^{*}_{i} < 0
\end{cases}
\end{align*}

\noindent
temos que:

\vspace{-1 em}
\begin{align*}
	P( y_{i} = 1 | \mbs{x} ) &= p( \mbs{x} )
	\\
	P( y_{i} = 0 | \mbs{x} ) &= 1 - p( \mbs{x} ).
\end{align*}

Além disso, pela definição de $y_{i}$, equação \eqref{mod1:probit}, temos:

\vspace{-1 em}
\begin{align*}
	P( y_{i} = 1 | \mbs{x} ) &= P(y_{i}^{*} \geq 0 \, | \mbs{x} )
\\
&= P( \mbs{x}_{i}' \mbs{\beta} + \err_{i} \geq 0 \, | \mbs{x} )
\\
&= P( \err_{i} \geq - \mbs{x}_{i}' \mbs{\beta}  \, | \mbs{x} ).
\end{align*}

\noindent
Agora, supondo que $\err_{i}$ tem FDA, $G$, tal que $G'=g$ é simétrica ao redor de zero:

\vspace{-1 em}
\begin{align*}
P( y_{i} = 1 | \mbs{x} ) 
&= 1 - P( \err_{i} < - \mbs{x}_{i}' \mbs{\beta}  \, | \mbs{x} )
\\
&= 1 - G( - \mbs{x}_{i}' \mbs{\beta}  \, | \mbs{x} )
\\
&= G( \mbs{x}_{i}' \mbs{\beta} ).
\end{align*}

Se $G(\cdot)$ for uma distribuição:

\begin{description}
	\item [Normal Padrão:] $\hat{\mbs{\beta}}$ é o estimador \textbf{probit}.
	\item [Logística:] $\hat{\mbs{\beta}}$ é o estimador \textbf{logit}.
\end{description}

Supondo $\mbs{y}_{i} \, | \, \mbs{x} \sim Bernoulli(p(\mbs{x}))$, sua fmp é dada por:

\vspace{-1 em}
\begin{align*}
f( y_{i} \, | \, \mbs{x}_{i} ; \mbs{\beta} ) 
&= 
\left[ G(\mbs{x}_{i}' \mbs{\beta} )  \right]^{y_{i}}
\left[ 1 - G(\mbs{x}_{i}' \mbs{\beta} )  \right]^{1 - y_{i}}
\; , \quad y=0,1.
\end{align*}

Para estimarmos $\hat{\mbs{\beta}}$ por máxima verossimilhança, temos de encontrar $\mbs{\beta} \in B$, onde $B$ é o espaço paramétrico, tal que $\mbs{\beta}$ maximize o valor da distribuição conjunta de $\mbs{y}$, ou seja:

\vspace{-1 em}
\begin{align*}
	\underset{\mbs{\beta} \in B}{\text{Max }} 
	\prod_{i=1}^{N}
	f( y_{i} \, | \, \mbs{x}_{i} ; \mbs{\beta} ).
\end{align*}

\noindent

Tirando o logaritmo e dividindo tudo por $N$ (podemos fazer isso pois são transformações monotônicas e não alteram o lugar onde $\mbs{\beta}$ ótimo irá parar):

\vspace{-1 em}
\begin{align*}
\underset{\mbs{\beta} \in B}{\text{Max }} 
\left\{ 
N^{-1} \sum_{i=1}^{N}
\ln \left[ f( y_{i} \, | \, \mbs{x}_{i} ; \mbs{\beta} ) \right]
\right\}.
\end{align*}

\noindent
Podemos definir
$\ell_{i}( \mbs{\beta} ) = \ln[ f( y_{i} \, | \, \mbs{x}_{i} ; \mbs{\beta} ) ]$
como sendo a verossimilhança condicional da observação $i$:

\vspace{-1 em}
\begin{align*}
\underset{\mbs{\beta} \in B}{\text{Max }} 
\left\{ 
N^{-1} \sum_{i=1}^{N} \ell_{i} (\mbs{\beta})
\right\}.
\end{align*}

Dessa forma, podemos ver que o problema acima é a analogia amostral de:

\vspace{-1 em}
\begin{align*}
\underset{\mbs{\beta} \in B}{\text{Max }} 
\E \left[ 
\ell_{i} ( \mbs{\beta} )
\right].
\end{align*}

Definindo o \textit{vector score} da observação $i$:

\vspace{-1 em}
\begin{align*}
s_{i} (\mbs{\beta}) = 
\left[ \nabla_{\mbs{\beta}} \ell_{i} (\mbs{\beta}) \right]'
=
\begin{bmatrix}
	\dfrac{\partial{\ell_{i} (\mbs{\beta})}}{\partial{\beta_{1}}},
	\dots,
	\dfrac{\partial{\ell_{i} (\mbs{\beta})}}{\partial{\beta_{K}}}
\end{bmatrix}
\end{align*}

Definindo a \textbf{Matriz Hessiana} da observação $i$:

\vspace{-1 em}
\begin{align*}
H_{i} (\mbs{\beta}) = 
\nabla_{\mbs{\beta}} s_{i} (\mbs{\beta}) = 
\nabla_{\mbs{\beta}}^2 \ell_{i} (\mbs{\beta})
\end{align*}

Tendo essas definições, o \textbf{Teorema do Valor Médio} (TVM) nos diz que no intervalo $[a, b]$, existe um número, $c$, tal que:

\vspace{-1 em}
\begin{align*}
	f'(c) = \frac{f(b) - f(a)}{b - a}.
\end{align*}

\begin{center}
\red{FAZER DESENHO}
\end{center}

Trocando 
$f(\cdot)$ por $s_{i}(\cdot)$, 
$a$ por $\mbs{\beta}_{0}$, 
$b$ por $\widehat{\mbs{\beta}}$ e
$c$ por $\bar{\mbs{\beta}}$,
temos:

\vspace{-1 em}
\begin{align*}
H_{i} ( \bar{\mbs{\beta}} ) =
\frac{s_{i}( \widehat{\mbs{\beta}} ) - s_{i}( \mbs{\beta}_{0} )}{\widehat{\mbs{\beta}} - \mbs{\beta}_{0}},
\end{align*}

\noindent
tirando médias dos dois lados:

\vspace{-1 em}
\begin{align*}
N^{-1} \sum_{i=1}^{N} 
H_{i} ( \bar{\mbs{\beta}} ) 
=
\frac{1}{\widehat{\mbs{\beta}} - \mbs{\beta}_{0}}
N^{-1} \sum_{i=1}^{N} 
\left[ 
s_{i}( \widehat{\mbs{\beta}} ) - s_{i}( \mbs{\beta}_{0} )
\right]
\end{align*}

Supondo que
$\widehat{\mbs{\beta}}$
maximiza
$\ell (\mbs{\beta} \, | \, \mbs{y}, \mbs{x})$,
temos que:
$N^{-1} \sum_{i=1}^{N} s_{i}(\widehat{\mbs{\beta}}) = 0$.
E podemos reescrever a equação anterior como:

\vspace{-1 em}
\begin{align*}
\widehat{\mbs{\beta}} - \mbs{\beta}_{0}
&=
(-1)
\left[ N^{-1} \sum_{i=1}^{N} H_{i} ( \bar{\mbs{\beta}} ) \right]^{-1}
N^{-1} \sum_{i=1}^{N} s_{i}( \mbs{\beta}_{0} ) 
\\
\sqrt{N} ( \widehat{\mbs{\beta}} - \mbs{\beta}_{0} )
&=
\left[
- N^{-1} \sum_{i=1}^{N} H_{i} ( \bar{\mbs{\beta}} )
\right]^{-1}
\sqrt{N} \cdot N^{-1} \sum_{i=1}^{N} s_{i}( \mbs{\beta}_{0} ) 
\\
\Aboxed{
\sqrt{N} ( \widehat{\mbs{\beta}} - \mbs{\beta}_{0} )
&=
\left[
- N^{-1} \sum_{i=1}^{N} H_{i} ( \bar{\mbs{\beta}} )
\right]^{-1}
N^{-1/2} \sum_{i=1}^{N} s_{i}( \mbs{\beta}_{0} ) }.
\end{align*}

\noindent
Onde

\vspace{-1 em}
\begin{align*}
\left[ 
- N^{-1} \sum_{i=1}^{N}
H_{i} ( \bar{\mbs{\beta}} ) \right]^{-1}
\xrightarrow{p}
A_{0}^{-1} \, ,
&&
N^{-1/2} \sum_{i=1}^{N} s_{i}( \mbs{\beta}_{0} ) 
\xrightarrow{d}
N( 0, B_{0} ).
\end{align*}

\noindent
Assim, temos que:

\vspace{-1 em}
\begin{align*}
\Aboxed{
\sqrt{N} ( \widehat{\mbs{\beta}} - \mbs{\beta}_{0} )
\to
N ( 0, A_{0}^{-1} B_{0} A_{0}^{-1} )}.
\end{align*}

A forma mais simples de achar $\Var ( {\widehat{\mbs{\beta}}} )$ é:

\vspace{-1 em}
\begin{align*}
	\Aboxed{
\Var( \widehat{\mbs{\beta}} )
&=
- \E[ H_{i} ( \widehat{\mbs{\beta}} ) ]^{-1} }.
\end{align*}

%%%%%%%%%%%%%%%%%%%%%%%%%%%%%%%%%%%%%%%%%%%%%%%%%%%%%%
\clearpage
\section{ATT, ATE, Propensity Score}


%%%%%%%%%%%%%%%%%%%%%%%%%%%%%%%%%%%%%%%%%%%%%%%%%%%%%%
\subsection*{Modelo}

\begin{itemize}\itemsep0pt
\item
$y_{1}$ $\rightarrow$ variável de interesse com tratamento

\item
$y_{0}$ $\rightarrow$ variável de interesse sem tratamento
\end{itemize}

\vspace{-1 em}
\begin{align*}
w = 
\begin{cases}
1 & \text{se tratam}
\\
0 & \text{se não tratam}
\end{cases}
\end{align*}
 
Idealmente, para isolarmos completamente o efeito de $w=1$, gostaríamos de pode calcular:

\vspace{-1 em}
\begin{align*}
	N^{-1} \sum_{i=1}^{N}
	\left( y_{i1} - y_{i0} \right).
\end{align*}

Ou seja, o efeito que o tratamento causa sobre um indivíduo com todo o resto permanecendo constante.
Em outras palavras, queríamos que houvesse dois mundos paralelos observáveis onde seria possível observar o que acontece com $y_{i}$ com e sem tratamento.
Infelizmente, para ccada indivíduo $i$, observamos apenas $y_{i1}$ ou $y_{i0}$, nunca ambos.

Antes de continuarmos, faremos as seguintes definições:

\begin{description}
	\item[ATE:]  $\E( y_{1} - y_{0} )$
	\item[ATT:]  $\E( y_{1} - y_{0} \, | \, w = 1 )$ (ATE no tratado).
\end{description}

%%%%%%%%%%%%%%%%%%%%%%%%%%%%%%%%%%%%%%%%%%%%%%%%%%%%%%
\subsection*{$ATE$ e $ATT$ condicional a variáveis $\mbs{x}$ }

\vspace{-1 em}
\begin{align*}
ATE( \mbs{x} ) &= \E( y_1 - y_0 \, | \mbs{x})
\\
ATT( \mbs{x} ) &= \E( y_1 - y_0 \, | \mbs{x}, w = 1)
\end{align*}

\noindent
\underline{OBS:}

\vspace{-1 em}
\begin{align*}
\E( y_1 - y_0 ) &= \E \left[ \E( y_1 - y_0 \, | w) \right]
\\
\E( y_1 - y_0 \, | w ) &=
\E( y_1 - y_0 \, | w = 0 ) \cdot P(w=0)
+
\E( y_1 - y_0 \, | w = 1 ) \cdot P(w=1).
\end{align*}

%%%%%%%%%%%%%%%%%%%%%%%%%%%%%%%%%%%%%%%%%%%%%%%%%%%%%%
\subsection*{Métodos Assumindo Ignorabilidade do Tratamento}

\begin{description}
\item[ATE.1:] Ignorabilidade. 
\\
$w$ e $(y_{1}, y_{0})$ são independentes condicionais a $\mbs{x}$.

\item[ATE.1':] Ignorabilidade da Média. 

\vspace{-.75 em}
\begin{enumerate}[label =\alph*)] \itemsep0pt
\item $\E( y_{0} \, | \, w, \mbs{x} ) = \E( y_{0} \, | \, \mbs{x} )$
\item $\E( y_{1} \, | \, w, \mbs{x} ) = \E( y_{1} \, | \, \mbs{x} )$
\end{enumerate}

\end{description}

Vamos definir

\vspace{-1 em}
\begin{align*}
\E( y_{0} \, | \, \mbs{x} ) &= \mu_{0}( \mbs{x} )
\\
\E( y_{1} \, | \, \mbs{x} ) &= \mu_{1}( \mbs{x} ).
\end{align*}

Sob \textbf{ATE.1} e \textbf{ATE.1'}:

\vspace{-1 em}
\begin{align*}
	ATE( \mbs{x} ) &= \E( y_{1} - y_{0} \, | \mbs{x} ) = \mu_{1}(\mbs{x}) - \mu_{0}(\mbs{x}) 
	\\
	ATT( \mbs{x} ) &= \E( y_{1} - y_{0} \, | \mbs{x}, w=1 ) = \mu_{1}(\mbs{x}) - \mu_{0}(\mbs{x}) 
\end{align*}

\begin{description}
\item[ATE.2:] \textit{Overlap} \\
Para todo $\mbs{x}$, $P(w=1 \, | \, \mbs{x} ) \in ( 0, 1 )$, 
$p(\mbs{x}) = p(w=1 | \mbs{x})$.
\end{description}

$p(\mbs{x})$ é o \textit{Propensity Score}, ele representa a probabilidade de $y_{i}$ ser tratado dado o valor das covariáveis $\mbs{x}$.
Essa hipótese é importante visto que podemos expressar o $ATE$ em função de $p(\mbs{x})$.

\vspace{1 em}
Para o $ATT$ vamos supor:

\begin{description}
\item[ATT.1':] 
	$\E( y_{0} \, | \mbs{x}, w ) = \E( y_{0} \, | \, \mbs{x} )$

\item[ATT.2:] \textit{Overlap:} Para todo $\mbs{x}$, $P(w=1 | \mbs{x} ) < 1$.
\end{description}

%%%%%%%%%%%%%%%%%%%%%%%%%%%%%%%%%%%%%%%%%%%%%%%%%%%%%%
\subsection*{Propensity Score}

Como foi dito anteriormente, apenas observamos ou $y_{1}$ ou $y_{0}$ para a mesma pessoa, mas não ambos.
Mais precisamente, junto com $w$, o resultado observado é:

\vspace{-1 em}
\begin{align*}
	y = wy_{1} + (1 - w) y_{0}
\end{align*}

\noindent
como  $w$ é binário, $w^2 = w$, assim, temos:

\vspace{-1 em}
\begin{align*}
w y &= w^{2} y_{1} + (w - w^{2}) y_{0}
\implies
\boxed{w y = w y_{1} }
\\
( 1 - w ) y &= (w - w^{2}) y_{1} + ( w^{2} - 2w + 1 ) y_{0}
\implies
\boxed{( 1 - w ) y = (1 - w) y_{0}}.
\end{align*}

Fazemos isso para tentar isolar $\mu_{0}(\mbs{x})$ e $\mu_{1}(\mbs{x})$:

%%%%%%%%%%%%%%%%%%%%%%%%%%%%%%%%%%%%%%%%%%%%%%%%%%%%%%
\paragraph{$\mu_{1}( \mbs{x} )$}

\begin{align*}
\E( w y | \mbs{x} ) &= \E\left[  \E \left( w y_{1} | \mbs{x}, w  \right) | \mbs{x} \right]
\\ &=
\E \left[ w \mu_{1}(\mbs{x}) | \mbs{x} \right]
\\ &=
\mu_{1}(\mbs{x}) \E(w | \mbs{x} ).
\end{align*}

\noindent
Como $w$ é binaria: $\E(w| \mbs{x}) = P(w=1 | \mbs{x}) = p(\mbs{x})$.
Assim:

\vspace{-1 em}
\begin{align*}
\E( w y | \mbs{x} ) &= \mu_{1}(\mbs{x}) p(\mbs{x})
\\
\Aboxed{ \mu_{1}(\mbs{x}) &= \frac{\E(w y | \mbs{x})}{p(\mbs{x})} }
\end{align*}

%%%%%%%%%%%%%%%%%%%%%%%%%%%%%%%%%%%%%%%%%%%%%%%%%%%%%%
\paragraph{$\mu_{0}( \mbs{x} )$}

\vspace{-1 em}
\begin{align*}
	\E[ (1-w) y | \mbs{x} ] &= \E\left[  \E \left( (1 - w) y_{0} | \mbs{x}, w  \right) | \mbs{x} \right]
\\ &=
\E \left[ (1 - w) \mu_{0}(\mbs{x}) | \mbs{x} \right]
\\ &=
\mu_{0}(\mbs{x}) \E(w | \mbs{x} )
\\ 
\E[ (1-w) y | \mbs{x} ] 
&=
\mu_{0}(\mbs{x}) [1 - p(\mbs{x})] \implies
\\ 
\Aboxed{
\mu_{0}(\mbs{x})
&=
\frac{\E[ (1-w) y | \mbs{x} ] }{1 - p( \mbs{x} ) } }
\end{align*}

%%%%%%%%%%%%%%%%%%%%%%%%%%%%%%%%%%%%%%%%%%%%%%%%%%%%%%
\paragraph{ATE:}

\begin{align*}
\mu_{1}(\mbs{x}) - \mu_{0}(\mbs{x}) =
\E\left[ 
\frac{[w - p(\mbs{x})] y}{p(\mbs{x}) [1 - p(\mbs{x})]}
| \mbs{x}
\right]
\end{align*}

\begin{align*}
\Aboxed{
\widehat{ATE} =
N^{-1} \sum_{i=1}^{N}
\frac{[ w_{i} - p(\mbs{x}_{i} ) ] y_{i} }{ p( \mbs{x}_{i} ) [1 - p( \mbs{x}_{i} ) ] }
}
\end{align*}

%%%%%%%%%%%%%%%%%%%%%%%%%%%%%%%%%%%%%%%%%%%%%%%%%%%%%%
\paragraph{ATT:}

\begin{align*}
\E( y_{1} | \mbs{x}, w=1) - \E(y_{0} | \mbs{x}) =
\frac{1}{\hat{P}(w=1)}
\E\left[ 
\frac{[w - \hat{p}(\mbs{x})] y}{[ 1 - \hat{p}(\mbs{x}) ]}
| \mbs{x}
\right]
\end{align*}

\vspace{-1 em}
\begin{align*}
	\hat{P} (w = 1) = N^{-1} \sum_{i=1}^{N} w_{i}
\end{align*}

\vspace{-1.5 em}
\begin{align*}
\widehat{ATT} &=
\frac{N}{\sum_{i=1}^{N} w_{i} }
N^{-1} \sum_{i=1}^{N}
\frac{[ w_{i} - \hat{p}(\mbs{x}_{i} ) ] y_{i} }{[ 1 - \hat{p}( \mbs{x}_{i} ) ]}
\\
\Aboxed{
\widehat{ATT} &=
\frac{1}{\sum_{i=1}^{N} w_{i} }
\sum_{i=1}^{N}
\frac{[ w_{i} - \hat{p}(\mbs{x}_{i} ) ] y_{i} }{[1 - \hat{p}( \mbs{x}_{i} ) ]}
}
\end{align*}

%%%%%%%%%%%%%%%%%%%%%%%%%%%%%%%%%%%%%%%%%%%%%%%%%%%%%%
\clearpage
\section*{TODO}

%%%%%%%%%%%%%%%%%%%%%%%%%%%%%%%%%%%%%%%%%%%%%%%%%%%%%%
%   Bibliografia
\clearpage
\bibliographystyle{authordate3} % bibliography style
\renewcommand\bibname{REFERENCES} 
\bibliography{../refs.bib}

%%%%%%%%%%%%%%%%%%%%%%%%%%%%%%%%%%%%%%%%%%%%%%%%%%%%%%%
\end{document}


