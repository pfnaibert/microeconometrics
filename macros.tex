%%%%%%%%%%%%%%%%%%%%%%%%%%%%%%%%%%%%%%%%%%%%%%%%%%%%%%%%%%%%%%
% MACROS
% 2017/10/13 Sex 11:34:37 
%%%%%%%%%%%%%%%%%%%%%%%%%%%%%%%%%%%%%%%%%%%%%%%%%%%%%%%%%%%%%%%%%

%%%%%%%%%%%%%%%%%%%%%%%%%%%%%%%%%%%%%%%%%%%%%%%%%%%%%%
% pagestyle
\rhead{\thepage}
\lhead{\theauthor}
\chead{\thetitle}
\renewcommand{\headrulewidth}{.5 pt} % no rule 
\rfoot{}
\lfoot{}
\cfoot{}

%%%%%%%%%%%%%%%%%%%%%%%%%%%%%%%%%%%%%%%%%%%%%%%%%%%%%%%%%%%%%%%%%
% Maketitle customization
\renewcommand{\maketitle}{%
\begin{center}
{\Large\uppercase{\thetitle}}\\
\vspace{1 em}
{\large\textsc{By \theauthor}} \\
\vspace{1 em}
\thedate
\end{center}
}

%%%%%%%%%%%%%%%%%%%%%%%%%%%%%%%%%%%%%%%%%%%%%%%%%%%%%%%%%%%%%%%%%
% Defs, theorems, etc
% \usepackage{amsthm}

\theoremstyle{definition}
\newtheorem{defn}{Definition}[section]

\newtheorem{teo}{Theorem}[section]
% \newtheorem{cor}{Corollary}[theorem]
\newtheorem{cor}{Corollary}[section]
% \newtheorem{lem}{Lemma}[theorem]
\newtheorem{lem}{Lemma}[section]

\theoremstyle{remark}
\newtheorem*{remark}{Remark}

\renewcommand{\qedsymbol}{\blacksquare}

%%%%%%%%%%%%%%%%%%%%%%%%%%%%%%%%%%%%%%%%%%%%%%%%%%%%%%%%%%%%%%%%%
% Stats

%%%%%%%%%%%%%%%%%%%%%%%%%%%%%%%%%%%%%%%%%%%%%%%%%%%%%%%%%%%%%%%%%
% operators
\newcommand{\Avar}{\ensuremath{\text{Avar}}}
\newcommand{\Avarhat}{\ensuremath{\widehat{\text{Avar}}}}

\newcommand{\Var}{\ensuremath{\text{Var}}}
\newcommand{\Varhat}{\ensuremath{\widehat{\text{Var}}}}

\newcommand{\se}{\ensuremath{\text{se}}}

\newcommand{\Cov}{\ensuremath{\text{Cov}}}
\newcommand{\E}{\ensuremath{\text{E}}}
\newcommand{\p}{\ensuremath{\text{P}}}

\newcommand{\B}{\ensuremath{\text{B}}}
% \newcommand{\H}{\ensuremath{\text{H}}}
\renewcommand{\L}{\ensuremath{\text{L}}}

\newcommand{\rank}{\ensuremath{\text{rank}}}
\newcommand{\MSE}{\ensuremath{\text{MSE}}}

\newcommand{\chisq}{\ensuremath{\chi^{2}}}

%%%%%%%%%%%%%%%%%%%%%%%%%%%%%%%%%%%%%%%%%%%%%%%%%%%%%%%%%%%%%%%%%
% Lims
\newcommand{\plim}{\ensuremath{\text{plim} \,}}

\newcommand{\arrowp}{\ensuremath{ \overset{p}{\longrightarrow} } }
\newcommand{\arrowd}{\ensuremath{ \overset{d}{\longrightarrow} } }
\newcommand{\asim}{\ensuremath{ \overset{a}{\sim} } }

\newcommand{\bigO}{\ensuremath{\text{O}}}
\newcommand{\bigOp}{\ensuremath{\text{O}_{p}}}

\newcommand{\lito}{\ensuremath{\text{o}}}
\newcommand{\litop}{\ensuremath{\text{o}_{p}}}

%%%%%%%%%%%%%%%%%%%%%%%%%%%%%%%%%%%%%%%%%%%%%%%%%%%%%%%%%%%%%%%%%
% greeks

\newcommand{\mubold}{\ensuremath{\mbs{\mu}}}

\newcommand{\betahat}{\ensuremath{\hat{\beta}}}
\newcommand{\betahatbold}{\ensuremath{\widehat{\mbs{\beta}}}}
\newcommand{\betavec}{\ensuremath{\mbs{\beta}}}

\newcommand{\gammavec}{\ensuremath{\mbs{\gamma}}}
\newcommand{\deltavec}{\ensuremath{\mbs{\delta}}}
\newcommand{\pivec}{\ensuremath{\mbs{\pi}}}
\newcommand{\pimat}{\ensuremath{\mbs{\Pi}}}

\newcommand{\thetabold}{\ensuremath{\mbs{\theta}}}
\newcommand{\Thetabold}{\ensuremath{\mbs{\Theta}}}

\newcommand{\thetahat}{\ensuremath{\hat{\theta}}}
\newcommand{\thetahatbold}{\ensuremath{\hat{\mbs{\theta}}}}

\newcommand{\thetatil}{\ensuremath{\tilde{\theta}}}
\newcommand{\thetatilbold}{\ensuremath{\tilde{\mbs{\theta}}}}

%%%%%%%%%%%%%%%%%%%%%%%%%%%%%%%%%%%%%%%%%%%%%%%%%%%%%%%%%%%%%%%%%
% Vectors
\newcommand{\ubold}{\ensuremath{\mbs{u}}}
\newcommand{\uhat}{\ensuremath{\widehat{u}}}
\newcommand{\uhatbold}{\ensuremath{\widehat{\mbs{u}}}}

\newcommand{\bvec}{\ensuremath{\mbf{b}}}
\newcommand{\cvec}{\ensuremath{\mbf{c}}}
\newcommand{\fvec}{\ensuremath{\mbf{f}}}
\newcommand{\gvec}{\ensuremath{\mbf{g}}}
\newcommand{\rvec}{\ensuremath{\mbf{r}}}
\newcommand{\vvec}{\ensuremath{\mbf{v}}}
\newcommand{\vhat}{\ensuremath{\hat{v}}}
\newcommand{\xvec}{\ensuremath{\mbs{x}}}
\newcommand{\yvec}{\ensuremath{\mbs{y}}}
\newcommand{\zvec}{\ensuremath{\mbf{z}}}
\newcommand{\wvec}{\ensuremath{\mbf{w}}}


%%%%%%%%%%%%%%%%%%%%%%%%%%%%%%%%%%%%%%%%%%%%%%%%%%%%%%%%%%%%%%%%%
% Matrices

\newcommand{\Abold}{\ensuremath{\mbf{A}}}
\newcommand{\Bbold}{\ensuremath{\mbf{B}}}
\newcommand{\Cbold}{\ensuremath{\mbf{C}}}
\newcommand{\Dbold}{\ensuremath{\mbf{D}}}

\newcommand{\Rbold}{\ensuremath{\mbf{R}}}
\newcommand{\Vbold}{\ensuremath{\mbf{V}}}
\newcommand{\Xbold}{\ensuremath{\mbf{X}}}
\newcommand{\Ybold}{\ensuremath{\mbf{y}}}
\newcommand{\Zbold}{\ensuremath{\mbf{Z}}}
\newcommand{\Wbold}{\ensuremath{\mbf{Z}}}

\newcommand{\Ahat}{\ensuremath{\widehat{A}}}
\newcommand{\Bhat}{\ensuremath{\widehat{B}}}
\newcommand{\Dhat}{\ensuremath{\widehat{D}}}
\newcommand{\Vhat}{\ensuremath{\widehat{V}}}

\newcommand{\Ahatbold}{\ensuremath{\widehat{\mbf{A}}}}
\newcommand{\Bhatbold}{\ensuremath{\widehat{\mbf{B}}}}
\newcommand{\Chatbold}{\ensuremath{\widehat{\mbf{C}}}}
\newcommand{\Dhatbold}{\ensuremath{\widehat{\mbf{D}}}}
\newcommand{\Vhatbold}{\ensuremath{\widehat{\mbf{V}}}}

\renewcommand{\LG}{\mathcal{L}}
\renewcommand{\l}{\ell}

%%%%%%%%%%%%%%%%%%%%%%%%%%%%%%%%%%%%%%%%%%%%%%%%%%%%%%%%%%%%%%%%%

%%%%%%%%%%%%%%%%%%%%%%%%%%%%%%%%%%%%%%%%%%%%%%%%%%%%%%%%%%%%%%%%%
% inner product
% produto interno na forma <x,y>
%   \DeclarePairedDelimiterX\innerp[2]{\langle}{\rangle}{#1,#2}
%   \DeclarePairedDelimiterX\floor[1]{\lfloor}{\rfloor}{#1}
\DeclareMathOperator{\Tr}{Tr}
\DeclareMathOperator*{\argmin}{\arg\!\min}
\DeclareMathOperator*{\argmax}{\arg\!\max}

%%%%%%%%%%%%%%%%%%%%%%%%%%%%%%%%%%%%%%%%%%%%%%%%%%%%%%%%%%%%%%%%%
% sequencias
%  fazer $\{x_n\}$ e $\{x\}_n^N$ mais rápido.
\renewcommand{\seq}[1]{\{#1\}}

%%%%%%%%%%%%%%%%%%%%%%%%%%%%%%%%%%%%%%%%%%%%%%%%%%%%%%%%%%%%%%%%%
% Atalhos para notação de conjuntos dos números reais, naturais, etc
% \renewcommand{\R}{\mathbb{R}}

\newcommand{\model}{\ensuremath{\mathcal{M}}}
\newcommand{\RS}{\ensuremath{R^2}}
\newcommand{\N}{\ensuremath{\mathbb{N}}}
\newcommand{\Q}{\ensuremath{\mathbb{Q}}}
\newcommand{\R}{\ensuremath{\mathbb{R}}} 
\newcommand{\Rn}{\ensuremath{\mathbb{R}^n}} 
\newcommand{\Rm}{\ensuremath{\mathbb{R}^m}} 
\newcommand{\Rp}{\ensuremath{\mathbb{R}^p}} 
\newcommand{\I}{\ensuremath{\mathbb{R}\setminus\mathbb{Q}}}
\newcommand{\md}{\ensuremath{\mathcal{D}}}

%%%%%%%%%%%%%%%%%%%%%%%%%%%%%%%%%%%%%%%%%%%%%%%%%%%%%%%%%%%%%%%
% atalhos para outros operadores matemáticos
% \DeclareMathOperator{\L}{\mathcal{L}}
\DeclareMathOperator{\eu}{\mathrm{e}}
\DeclareMathOperator{\de}{\mathrm{d}}
\DeclareMathOperator{\up}{\uparrow}
\DeclareMathOperator{\dn}{\downarrow}

%%%%%%%%%%%%%%%%%%%%%%%%%%%%%%%%%%%%%%%%%%%%%%%%%%%%%%%%%%%%%%%
% Derivadas
\newcommand{\dpf}[2]{\frac{d#1}{d#2}}
\newcommand{\dsf}[2]{\frac{d^2#1}{d#2^2}}
\newcommand{\opt}[1]{{#1}^{*}}

%%%%%%%%%%%%%%%%%%%%%%%%%%%%%%%%%%%%%%%%%%%%%%%%%%%%%%%%%%%%%%%
% Derivadas Parciais
\newcommand{\pdf}[2]{\frac{\partial#1}{\partial#2}}
\newcommand{\psdf}[2]{\frac{\partial^2#1}{\partial#2^2}}
\newcommand{\pcdf}[3]{\frac{\partial^2#1}{\partial#2 \partial#3}}

%%%%%%%%%%%%%%%%%%%%%%%%%%%%%%%%%%%%%%%%%%%%%%%%%%%%%%%%%%%%%%%
% função valor
\newcommand{\vth}{V_{\theta}}

%%%%%%%%%%%%%%%%%%%%%%%%%%%%%%%%%%%%%%%%%%%%%%%%%%%%%%%%%%%%%%%
% Atalhos para cores:

\newcommand{\red}[1]{\textcolor{red}{#1}}
\newcommand{\blue}[1]{\textcolor{blue}{#1}}
\newcommand{\green}[1]{\textcolor{green}{#1}}
\newcommand{\magenta}[1]{\textcolor{magenta}{#1}}
% \newcommand{\hl}[1]{\colorbox{yellow}{#1}}

%%%%%%%%%%%%%%%%%%%%%%%%%%%%%%%%%%%%%%%%%%%%%%%%%%%%%%%%%%%%%%%
% GREEK LETTERS
\newcommand{\ga}{\ensuremath{\alpha}} 
\newcommand{\gab}{\ensuremath{\boldsymbol{\alpha}}} 
\newcommand{\gb}{\ensuremath{\beta}} 
\newcommand{\gbb}{\ensuremath{\boldsymbol{\beta}}} 
\newcommand{\G}{\ensuremath{\Gamma}} 
\newcommand{\gg}{\ensuremath{\gamma}} 
\newcommand{\D}{\ensuremath{\Delta}} 
\newcommand{\gl}{\ensuremath{\lambda}} 
\newcommand{\gL}{\ensuremath{\Lambda}} 
\newcommand{\gSi}{\ensuremath{\Sigma^{-1}}}  % inv sigma
\newcommand{\gS}{\ensuremath{\Sigma}} 
\newcommand{\gs}{\ensuremath{\sigma}} 
\newcommand{\gsh}{\ensuremath{\hat{\sigma}}} 
\newcommand{\gss}{\ensuremath{\sigma^2}} 
\newcommand{\gssh}{\ensuremath{\hat{\sigma}^2}}  % sigma hat
\newcommand{\gSib}{\ensuremath{\boldsymbol{\Sigma}^{-1}}} 
\newcommand{\gSb}{\ensuremath{\boldsymbol{\Sigma}}} 
\newcommand{\gsb}{\ensuremath{\boldsymbol{\sigma}}} 
\newcommand{\gib}{\ensuremath{\boldsymbol{\iota}}} 
\newcommand{\gi}{\ensuremath{\iota}}
\newcommand{\err}{\ensuremath{\varepsilon}} 
\newcommand{\gO}{\ensuremath{\Omega}} 

%%%%%%%%%%%%%%%%%%%%%%%%%%%%%%%%%%%%%%%%%%%%%%%%%%%%%%%%%%%%%%%
% 
\newcommand{\bm}[1]{\boldsymbol{#1}}
\newcommand{\mbs}[1]{\boldsymbol{#1}}
\newcommand{\mbf}[1]{\mathbf{#1}}

%%%%%%%%%%%%%%%%%%%%%%%%%%%%%%%%%%%%%%%%%%%%%%%%%%%%%%%%%%%%%%%
% Não faço a mínima ideia o que esse comandos fazem
% \renewcommand{\cancel}{\color{red}}
% \newcommand{\encircle}[1]{\tikz[baseline= (X.base)]\node(X)[draw, shape=circle, inner sep=0]{\strut#1};}
%%%%%%%%%%%%%%%%%%%%%%%%%%%%%%%%%%%%%%%%%%%%%%%%%%%%%%%%%%%%%%%



