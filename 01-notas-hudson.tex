\documentclass[11pt, oneside, a4paper, article]{article}
% \documentclass[11pt, oneside, a4paper, article, ms]{memoir}
% \documentclass[11pt,oneside,a4paper]{letter}
\usepackage[a4paper, margin=1in]{geometry}   % MARGENS

%%%%%%%%%%%%%%%%%%%%%%%%%%%%%%%%%%%%%%%%%%%%%%%%%%%%%%
% Pacotes básicos 
\usepackage[T1]{fontenc}		% Selecao de codigos de fonte.
\usepackage[utf8]{inputenc}		% Codificacao do documento (conversão automática dos acentos)

%%%%%%%%%%%%%%%%%%%%%%%%%%%%%%%%%%%%%%%%%%%%%%%%%%%%%%
% languages
\usepackage[english]{babel}
% \selectlanguage{english}

%%%%%%%%%%%%%%%%%%%%%%%%%%%%%%%%%%%%%%%%%%%%%%%%%%%%%%
% AMS math
\usepackage{amsfonts, amssymb, amsthm}
\usepackage[fleqn]{amsmath}
\setlength{\mathindent}{0pt}

\usepackage{mathtools, latexsym}
% \usepackage{mathabx}

%%%%%%%%%%%%%%%%%%%%%%%%%%%%%%%%%%%%%%%%%%%%%%%%%%%%%%
% Pacotes de citações
\usepackage{natbib}	% Citações}

%%%%%%%%%%%%%%%%%%%%%%%%%%%%%%%%%%%%%%%%%%%%%%%%%%%%%%
\usepackage{indentfirst}		% Indenta o primeiro parágrafo de cada seção.
\usepackage{color}				% Controle das cores
\usepackage{graphicx}			% Inclusão de gráficos
\usepackage{microtype} 			% para melhorias de justificação
%\usepackage{listings}			% Formatação para inserir códigos
\usepackage[normalem]{ulem}
\usepackage[all]{xy}
\usepackage{xcolor}
\usepackage{ragged2e}           % formatação texto
\usepackage{bm}                 % bold symbols 
\usepackage[colorlinks, citecolor=blue, urlcolor=blue, linkcolor=red]{hyperref} % referencias dentro do texo (*QUEBRA MEMOIR*)
\usepackage{url}                % URL
\usepackage{graphicx}			% Inclusão de gráficos
\usepackage{subcaption}			% Faz subfiguras
%\usepackage{subfigure}         % subfigures
% \usepackage{setspace}         % Espaçamento
\usepackage{enumitem}
% \usepackage{enumerate}
\usepackage[sharp]{easylist}
% \usepackage{titlesec}		    % Customização de seçoes
\usepackage{titling}          % personalized other things
\usepackage{fancyhdr}         % personalized page style
\usepackage{lipsum}           % dummy text
% \usepackage{exercise}       % exercises
\usepackage[flushleft]{threeparttable} % notas nas tabelas

% \usepackage{lscape}				% Gira a página em 90 graus
\usepackage{pdflscape} % páginas em formato paisagem
\usepackage{multirow} % permite fazer tabelas com multirows
\usepackage{tabularx} % 
\usepackage{tikz} % desenhos
\tikzset{>=latex}

\usepackage[pagewise]{lineno}           % show line numbers
% \linenumbers

%%%%%%%%%%%%%%%%%%%%%%%%%%%%%%%%%%%%%%%%%%%%%%%%%%%%%%
%%%%%%%%%%%%%%%%%%%%%%%%%%%%%%%%%%%%%%%%%%%%%%%%%%%%%%%%%%%%%%
% MACROS
% 2017/10/13 Sex 11:34:37 
%%%%%%%%%%%%%%%%%%%%%%%%%%%%%%%%%%%%%%%%%%%%%%%%%%%%%%%%%%%%%%%%%

%%%%%%%%%%%%%%%%%%%%%%%%%%%%%%%%%%%%%%%%%%%%%%%%%%%%%%
% pagestyle
\rhead{\thepage}
\lhead{\theauthor}
\chead{\thetitle}
\renewcommand{\headrulewidth}{.5 pt} % no rule 
\rfoot{}
\lfoot{}
\cfoot{}

%%%%%%%%%%%%%%%%%%%%%%%%%%%%%%%%%%%%%%%%%%%%%%%%%%%%%%%%%%%%%%%%%
% Maketitle customization
\renewcommand{\maketitle}{%
\begin{center}
{\Large\uppercase{\thetitle}}\\
\vspace{1 em}
{\large\textsc{By \theauthor}} \\
\vspace{1 em}
\thedate
\end{center}
}

%%%%%%%%%%%%%%%%%%%%%%%%%%%%%%%%%%%%%%%%%%%%%%%%%%%%%%%%%%%%%%%%%
% Defs, theorems, etc
% \usepackage{amsthm}

\theoremstyle{definition}
\newtheorem{defn}{Definition}[section]

\newtheorem{teo}{Theorem}[section]
% \newtheorem{cor}{Corollary}[theorem]
\newtheorem{cor}{Corollary}[section]
% \newtheorem{lem}{Lemma}[theorem]
\newtheorem{lem}{Lemma}[section]

\theoremstyle{remark}
\newtheorem*{remark}{Remark}

\renewcommand{\qedsymbol}{\blacksquare}

%%%%%%%%%%%%%%%%%%%%%%%%%%%%%%%%%%%%%%%%%%%%%%%%%%%%%%%%%%%%%%%%%
% Stats

%%%%%%%%%%%%%%%%%%%%%%%%%%%%%%%%%%%%%%%%%%%%%%%%%%%%%%%%%%%%%%%%%
% operators
\newcommand{\ssr}{\ensuremath{\text{SSR}}}
\newcommand{\tss}{\ensuremath{\text{TSS}}}

\newcommand{\E}{\ensuremath{\text{E}}}
\newcommand{\Var}{\ensuremath{\text{Var}}}
\newcommand{\Cov}{\ensuremath{\text{Cov}}}
\newcommand{\Corr}{\ensuremath{\text{Corr}}}
\newcommand{\Avar}{\ensuremath{\text{Avar}}}
\newcommand{\se}{\ensuremath{\text{se}}}

\newcommand{\Avarhat}{\ensuremath{\widehat{\text{Avar}}}}
\newcommand{\Varhat}{\ensuremath{\widehat{\text{Var}}}}

\newcommand{\p}{\ensuremath{\text{P}}}

\newcommand{\B}{\ensuremath{\text{B}}}
% \newcommand{\H}{\ensuremath{\text{H}}}
% \renewcommand{\L}{\ensuremath{\text{L}}}

\newcommand{\rank}{\ensuremath{\text{rank}}}
\newcommand{\posto}{\ensuremath{\text{posto}}}
\newcommand{\MSE}{\ensuremath{\text{MSE}}}

\newcommand{\chisq}{\ensuremath{\chi^{2}}}

\newcommand{\nor}{\ensuremath{\mathcal{N}}}

\newcommand{\proj}{\ensuremath{\text{proj}}}


%%%%%%%%%%%%%%%%%%%%%%%%%%%%%%%%%%%%%%%%%%%%%%%%%%%%%%%%%%%%%%%%%
% Lims
\newcommand{\plim}{\ensuremath{\text{plim} \,}}

\newcommand{\arrowp}{\ensuremath{ \overset{p}{\longrightarrow} } }
\newcommand{\arrowd}{\ensuremath{ \overset{d}{\longrightarrow} } }
\newcommand{\asim}{\ensuremath{\overset{a}{\sim}}}

\newcommand{\bigO}{\ensuremath{\text{O}}}
\newcommand{\bigOp}{\ensuremath{\text{O}_{p}}}

\newcommand{\lito}{\ensuremath{\text{o}}}
\newcommand{\litop}{\ensuremath{\text{o}_{p}}}

%%%%%%%%%%%%%%%%%%%%%%%%%%%%%%%%%%%%%%%%%%%%%%%%%%%%%%%%%%%%%%%%%
% greeks
\newcommand{\alphavec}{\ensuremath{\mbs{\alpha}}}

\newcommand{\sigmahat}{\ensuremath{\hat{\sigma}}}
\newcommand{\Sigmahat}{\ensuremath{\widehat{\Sigma}}}

\newcommand{\Sigmainv}{\ensuremath{\Sigma^{-1}}}
\newcommand{\Sigmainvhat}{\ensuremath{\widehat{\Sigma^{-1}}}}

\newcommand{\Omegahat}{\ensuremath{\widehat{\Omega}}}

\newcommand{\mubold}{\ensuremath{\mbs{\mu}}}
\newcommand{\muvec}{\ensuremath{\mbs{\mu}}}

\newcommand{\betahat}{\ensuremath{\hat{\beta}}}
\newcommand{\betahatbold}{\ensuremath{\widehat{\mbs{\beta}}}}

\newcommand{\betavec}{\ensuremath{\mbs{\beta}}}
\newcommand{\betavechat}{\ensuremath{\widehat{\mbs{\beta}}}}

\newcommand{\gammavec}{\ensuremath{\mbs{\gamma}}}
\newcommand{\deltavec}{\ensuremath{\mbs{\delta}}}
\newcommand{\pivec}{\ensuremath{\mbs{\pi}}}
\newcommand{\pimat}{\ensuremath{\mbs{\Pi}}}

\newcommand{\thetabold}{\ensuremath{\mbs{\theta}}}
\newcommand{\Thetabold}{\ensuremath{\mbs{\Theta}}}
\newcommand{\thetahat}{\ensuremath{\hat{\theta}}}
\newcommand{\thetaboldhat}{\ensuremath{\hat{\mbs{\theta}}}}

\newcommand{\thetatil}{\ensuremath{\tilde{\theta}}}
\newcommand{\thetatilbold}{\ensuremath{\tilde{\mbs{\theta}}}}

%%%%%%%%%%%%%%%%%%%%%%%%%%%%%%%%%%%%%%%%%%%%%%%%%%%%%%%%%%%%%%%%%
% Vectors

\newcommand{\onevec}{\ensuremath{\mbs{1}}}
\newcommand{\zerovec}{\ensuremath{\mbs{0}}}


\newcommand{\bvec}{\ensuremath{\mbf{b}}}
\newcommand{\cvec}{\ensuremath{\mbf{c}}}
\newcommand{\fvec}{\ensuremath{\mbf{f}}}
\newcommand{\gvec}{\ensuremath{\mbf{g}}}
\newcommand{\hvec}{\ensuremath{\mbf{h}}}
\newcommand{\rvec}{\ensuremath{\mbf{r}}}
\newcommand{\uvec}{\ensuremath{\mbs{u}}}
\newcommand{\vvec}{\ensuremath{\mbf{v}}}
\newcommand{\vhat}{\ensuremath{\hat{v}}}
\newcommand{\xvec}{\ensuremath{\mbs{x}}}
\newcommand{\yvec}{\ensuremath{\mbs{y}}}
\newcommand{\zvec}{\ensuremath{\mbf{z}}}
\newcommand{\wvec}{\ensuremath{\mbf{w}}}

\newcommand{\uhat}{\ensuremath{\hat{u}}}
\newcommand{\yhat}{\ensuremath{\hat{y}}}

\newcommand{\uvechat}{\ensuremath{\widehat{\mbs{u}}}}
\newcommand{\yvechat}{\ensuremath{\widehat{\mbs{y}}}}

%%%%%%%%%%%%%%%%%%%%%%%%%%%%%%%%%%%%%%%%%%%%%%%%%%%%%%%%%%%%%%%%%
% Matrices

\newcommand{\Amat}{\ensuremath{\mbf{A}}}
\newcommand{\Bmat}{\ensuremath{\mbf{B}}}
\newcommand{\Cmat}{\ensuremath{\mbf{C}}}
\newcommand{\Dmat}{\ensuremath{\mbf{D}}}
\newcommand{\Imat}{\ensuremath{\mbf{I}}}
\newcommand{\Mmat}{\ensuremath{\mbf{M}}}
\newcommand{\Mzero}{\ensuremath{\mbf{M}^{0}}}

\newcommand{\Pmat}{\ensuremath{\mbf{P}}}
\newcommand{\Qmat}{\ensuremath{\mbf{Q}}}
\newcommand{\Rmat}{\ensuremath{\mbf{R}}}
\newcommand{\Vmat}{\ensuremath{\mbf{V}}}
\newcommand{\Xmat}{\ensuremath{\mbf{X}}}
\newcommand{\Ymat}{\ensuremath{\mbf{y}}}
\newcommand{\Zmat}{\ensuremath{\mbf{Z}}}
\newcommand{\Wmat}{\ensuremath{\mbf{Z}}}

\newcommand{\Ahat}{\ensuremath{\widehat{A}}}
\newcommand{\Bhat}{\ensuremath{\widehat{B}}}
\newcommand{\Dhat}{\ensuremath{\widehat{D}}}
\newcommand{\Vhat}{\ensuremath{\widehat{V}}}

\newcommand{\Amathat}{\ensuremath{\widehat{\mbf{A}}}}
\newcommand{\Bmathat}{\ensuremath{\widehat{\mbf{B}}}}
\newcommand{\Vmathat}{\ensuremath{\widehat{\mbf{V}}}}

\newcommand{\Ahatbold}{\ensuremath{\widehat{\mbf{A}}}}
\newcommand{\Bhatbold}{\ensuremath{\widehat{\mbf{B}}}}
\newcommand{\Chatbold}{\ensuremath{\widehat{\mbf{C}}}}
\newcommand{\Dhatbold}{\ensuremath{\widehat{\mbf{D}}}}
\newcommand{\Vhatbold}{\ensuremath{\widehat{\mbf{V}}}}

\renewcommand{\LG}{\mathcal{L}}
\renewcommand{\l}{\ell}

%%%%%%%%%%%%%%%%%%%%%%%%%%%%%%%%%%%%%%%%%%%%%%%%%%%%%%%%%%%%%%%%%

%%%%%%%%%%%%%%%%%%%%%%%%%%%%%%%%%%%%%%%%%%%%%%%%%%%%%%%%%%%%%%%%%
% inner product
% produto interno na forma <x,y>
\DeclarePairedDelimiterX\innerp[2]{\langle}{\rangle}{#1,#2}
%   \DeclarePairedDelimiterX\floor[1]{\lfloor}{\rfloor}{#1}
\DeclareMathOperator{\Tr}{Tr}
\DeclareMathOperator*{\argmin}{\arg\!\min}
\DeclareMathOperator*{\argmax}{\arg\!\max}

%%%%%%%%%%%%%%%%%%%%%%%%%%%%%%%%%%%%%%%%%%%%%%%%%%%%%%%%%%%%%%%%%
% sequencias
%  fazer $\{x_n\}$ e $\{x\}_n^N$ mais rápido.
\renewcommand{\seq}[1]{\{#1\}}

%%%%%%%%%%%%%%%%%%%%%%%%%%%%%%%%%%%%%%%%%%%%%%%%%%%%%%%%%%%%%%%%%
% Atalhos para notação de conjuntos dos números reais, naturais, etc
% \renewcommand{\R}{\mathbb{R}}

\newcommand{\model}{\ensuremath{\mathcal{M}}}
\newcommand{\RS}{\ensuremath{R^2}}
\newcommand{\N}{\ensuremath{\mathbb{N}}}
\newcommand{\Q}{\ensuremath{\mathbb{Q}}}
\newcommand{\R}{\ensuremath{\mathbb{R}}} 
\newcommand{\Rn}{\ensuremath{\mathbb{R}^n}} 
\newcommand{\Rm}{\ensuremath{\mathbb{R}^m}} 
\newcommand{\Rp}{\ensuremath{\mathbb{R}^p}} 
\newcommand{\I}{\ensuremath{\mathbb{R}\setminus\mathbb{Q}}}
\newcommand{\md}{\ensuremath{\mathcal{D}}}

%%%%%%%%%%%%%%%%%%%%%%%%%%%%%%%%%%%%%%%%%%%%%%%%%%%%%%%%%%%%%%%
% atalhos para outros operadores matemáticos
% \DeclareMathOperator{\L}{\mathcal{L}}
\DeclareMathOperator{\eu}{\mathrm{e}}
\DeclareMathOperator{\de}{\mathrm{d}}
\DeclareMathOperator{\up}{\uparrow}
\DeclareMathOperator{\dn}{\downarrow}

%%%%%%%%%%%%%%%%%%%%%%%%%%%%%%%%%%%%%%%%%%%%%%%%%%%%%%%%%%%%%%%
% Derivadas
\newcommand{\dpf}[2]{\frac{d#1}{d#2}}
\newcommand{\dsf}[2]{\frac{d^2#1}{d#2^2}}
\newcommand{\opt}[1]{{#1}^{*}}

%%%%%%%%%%%%%%%%%%%%%%%%%%%%%%%%%%%%%%%%%%%%%%%%%%%%%%%%%%%%%%%
% Derivadas Parciais
\newcommand{\pdf}[2]{\frac{\partial#1}{\partial#2}}
\newcommand{\psdf}[2]{\frac{\partial^2#1}{\partial#2^2}}
\newcommand{\pcdf}[3]{\frac{\partial^2#1}{\partial#2 \partial#3}}

%%%%%%%%%%%%%%%%%%%%%%%%%%%%%%%%%%%%%%%%%%%%%%%%%%%%%%%%%%%%%%%
% função valor
\newcommand{\vth}{V_{\theta}}

%%%%%%%%%%%%%%%%%%%%%%%%%%%%%%%%%%%%%%%%%%%%%%%%%%%%%%%%%%%%%%%
% Atalhos para cores:

\newcommand{\red}[1]{\textcolor{red}{#1}}
\newcommand{\blue}[1]{\textcolor{blue}{#1}}
\newcommand{\green}[1]{\textcolor{green}{#1}}
\newcommand{\magenta}[1]{\textcolor{magenta}{#1}}
% \newcommand{\hl}[1]{\colorbox{yellow}{#1}}

%%%%%%%%%%%%%%%%%%%%%%%%%%%%%%%%%%%%%%%%%%%%%%%%%%%%%%%%%%%%%%%
% GREEK LETTERS
\newcommand{\ga}{\ensuremath{\alpha}} 
\newcommand{\gab}{\ensuremath{\boldsymbol{\alpha}}} 
\newcommand{\gb}{\ensuremath{\beta}} 
\newcommand{\gbb}{\ensuremath{\boldsymbol{\beta}}} 
\newcommand{\G}{\ensuremath{\Gamma}} 
\newcommand{\gg}{\ensuremath{\gamma}} 
\newcommand{\D}{\ensuremath{\Delta}} 
\newcommand{\gl}{\ensuremath{\lambda}} 
\newcommand{\gL}{\ensuremath{\Lambda}} 
\newcommand{\gib}{\ensuremath{\boldsymbol{\iota}}} 
\newcommand{\gi}{\ensuremath{\iota}}
\newcommand{\err}{\ensuremath{\varepsilon}} 

%%%%%%%%%%%%%%%%%%%%%%%%%%%%%%%%%%%%%%%%%%%%%%%%%%%%%%%%%%%%%%%
% 
\newcommand{\bm}[1]{\boldsymbol{#1}}
\newcommand{\mbs}[1]{\boldsymbol{#1}}
\newcommand{\mbf}[1]{\mathbf{#1}}

%%%%%%%%%%%%%%%%%%%%%%%%%%%%%%%%%%%%%%%%%%%%%%%%%%%%%%%%%%%%%%%
% Não faço a mínima ideia o que esse comandos fazem
% \renewcommand{\cancel}{\color{red}}
% \newcommand{\encircle}[1]{\tikz[baseline= (X.base)]\node(X)[draw, shape=circle, inner sep=0]{\strut#1};}
%%%%%%%%%%%%%%%%%%%%%%%%%%%%%%%%%%%%%%%%%%%%%%%%%%%%%%%%%%%%%%%




% \pagestyle{headings}
\pagestyle{fancy}
\numberwithin{equation}{section}
% \setcounter{equation}{0}

%%%%%%%%%%%%%%%%%%%%%%%%%%%%%%%%%%%%%%%%%%%%%%%%%%%%%%
% Begin
\begin{document}

%%%%%%%%%%%%%%%%%%%%%%%%%%%%%%%%%%%%%%%%%%%%%%%%%%%%%%
% METADATA
\title{Microeconometrics: Lecture Notes }
\author{Paulo F. Naibert}
% \date{25/06/2020}
% \date{\today}

%%%%%%%%%%%%%%%%%%%%%%%%%%%%%%%%%%%%%%%%%%%%%%%%%%%%%%
\thispagestyle{empty}
\pagenumbering{gobble}

\begin{center}
	\textbf{UNIVERSIDADE FEDERAL DO RIO GRANDE DO SUL}
	\\
	\textbf{PROGRAMA DE PÓS-GRADUAÇÃO EM ECONOMIA}
	\\
	\textbf{Microeconometria -- 2015/3}

	\vfill
	\textbf{\thetitle}

	\vfill
	\textbf{Autor: Paulo Ferreira Naibert } 
	\\
	\textbf{Professor: Hudson Torrent} 


\end{center}

\vfill

\begin{center}
	\textbf{Porto Alegre \\ 30/06/2020 \\ Revisão: \today}
\end{center}

%%%%%%%%%%%%%%%%%%%%%%%%%%%%%%%%%%%%%%%%%%%%%%%%%%%%%%
\clearpage
\pagenumbering{arabic}


%%%%%%%%%%%%%%%%%%%%%%%%%%%%%%%%%%%%%%%%%%%%%%%%%%%%%%
\clearpage
\section{Regressão MQO Clássico}
\noindent
\citet[C.4 -- The Single-Equation Linear Model and OLS Estimation, p.49--76]{wool-2010}

%%%%%%%%%%%%%%%%%%%%%%%%%%%%%%%%%%%%%%%%%%%%%%%%%%%%%%
\subsection{Modelo de equações lineares} 

O modelo populacional que estudamos é linear em seus parâmetros,

\vspace{-1 em}
\begin{align} \label{ols:mod}
	y &= \beta_{0} + \beta_{1} x_{1} + \dots + \beta_{K} x_{K} + u
\end{align}
onde:

\begin{description}[\noitemsep]
	\item [$y, x_{1}, \dots, x_{K}$]  são escalares aleatórios e observáveis (i.e., conseguimos observá-los em uma amostra aleatória da população);

	\item [$u$] é o \textit{random disturbance} não observável, ou erro; 

	\item [$\beta_{0}, \beta_{1}, \dots, \beta_{K}$] são parâmetros (constantes) que gostaríamos de estimar.
\end{description}

%%%%%%%%%%%%%%%%%%%%%%%%%%%%%%%%%%%%%%%%%%%%%%%%%%%%%%
\subsection*{Notação Vetorial} 
\noindent
\citet[Sec. 4.2 -- Asymptotic Properties of OLS; p.51]{wool-2010}

Por conveniência, escrevemos a equação populacional em forma de vetor:

\vspace{-1 em}
\begin{align} \label{ols:mod:vec}
	y &= \xvec \betavec + u
\end{align}

\noindent
onde,

\vspace{-1 em}
\begin{description}[noitemsep]
	\item [$\xvec \equiv (x_{1}, \dots, x_{K})$] é um vetor $1 \times K$ de regressores;

	\item [$\betavec \equiv (\beta_{1}, \dots, \beta_{K})'$] é um vetor $K \times 1$.
\end{description}

\noindent
Uma vez que a maioria das equações contém um intercepto, assumiremos que $x_{1} \equiv 1$, visto que essa hipótese deixa a interpretação mais fácil.

%%%%%%%%%%%%%%%%%%%%%%%%%%%%%%%%%%%%%%%%%%%%%%%%%%%%%%
\subsection*{Amostra Aleatória} 

Assumimos que conseguimos obter uma amostra aleatória de tamanho $N$ da população para estimarmos $\betavec$.
Dessa forma, $\{ (\xvec_{i}, y_{i}); \, i = 1, 2, \dots, N \}$
são tratados como variáveis aleatória independentes, identicamente distribuídas, onde
$\xvec_{i}$ é $1 \times K$ e $y_{i}$ é escalar.
Para cada observação $i$, temos:

%%%%%%%%%%%%%%%%%%%%%%%%%%%%%%%%%%%%%%%%%%%%%%%%%%%%%%
\vspace{-1 em}
\begin{align} \label{ols:mod:vec:i}
	y_{i} &= \xvec_{i} \betavec + u_{i}.
\end{align}

\noindent
onde
$\xvec_{i}$
é um vetor $1 \times K$ de regressores.

%%%%%%%%%%%%%%%%%%%%%%%%%%%%%%%%%%%%%%%%%%%%%%%%%%%%%%
\subsection{Hipóteses} 

\paragraph{OLS.1} 
$y_{i} &= \xvec_{i} \betavec + u_{i} \, , \quad i = 1, \dots, N$;

%%%%%%%%%%%%%%%%%%%%%%%%%%%%%%%%%%%%%%%%%%%%%%%%%%%%%%
\paragraph{OLS.2}  $\Xmat$ é \textbf{não} estocástica;

%%%%%%%%%%%%%%%%%%%%%%%%%%%%%%%%%%%%%%%%%%%%%%%%%%%%%%
\paragraph{OLS.3} $\{ u_{i} \}_{i=1}^{N}$  é  $iid$ com e para cada $i = 1, \dots, N$:

\vspace{-1.5 em}
\begin{align*}
	\E(u_{i}) &= 0
	\\
	\Var(u_{i}) &= \E(u_{i}^2) = \sigma^2
\end{align*}

%%%%%%%%%%%%%%%%%%%%%%%%%%%%%%%%%%%%%%%%%%%%%%%%%%%%%%
\paragraph{OLS.2'} $\Xmat$ é estocástica;

%%%%%%%%%%%%%%%%%%%%%%%%%%%%%%%%%%%%%%%%%%%%%%%%%%%%%%
\paragraph{OLS.3'} 

\begin{align*}
	\E(u_{i} | \Xmat) &= 0, 
	\\
	\Var(u_{i} | \Xmat) &= 
	E
	\left\{ \left[ 
			u_{i} - \E( u_{i} | \Xmat)
	\right]^2 | \Xmat \right\}
	=
	\E(u_{i}^2 | \Xmat) = \sigma^2.
\end{align*}

\begin{remark}
	$\E(u_{i} | \Xmat) = 0$ implica que $u_{i}$ é \textbf{não correlacionado} com todos os regressores $x_{k}$ para $k=1,\dots, K$. \red{Exogeneidade estrita}.
\end{remark}

%%%%%%%%%%%%%%%%%%%%%%%%%%%%%%%%%%%%%%%%%%%%%%%%%%%%%%
\subsection{Estimação} 

Usando \textbf{OLS.1:}

\vspace{-1.5 em}
\begin{align*}
	y_{i} &= \xvec_{i} \betavec + u_{i} 
	\\
	\xvec_{i}' y_{i} &= \xvec_{i}' \xvec_{i} \betavec + \xvec_{i}' u_{i} 
	\\
	\E( \xvec_{i}' y_{i}) &= \E( \xvec_{i}' \xvec_{i} ) \betavec + \E( \xvec_{i}' u_{i} ) 
\end{align*}
Usando $\boxed{\E( \xvec_{i}' u_{i} ) = 0}$ 
\red{[Qual seria essa hipótese?]}

\vspace{-1 em}
\begin{align}
	\notag
	\E( \xvec_{i}' y_{i}) &= \E( \xvec_{i}' \xvec_{i} ) \betavec
	\\
	\label{ols:beta}
	\Aboxed{\betavec &= [\E( \xvec_{i}' \xvec_{i} )]^{-1} \E( \xvec_{i}' y_{i}) }.
\end{align}

Agora, usando o \textbf{princípio da analogia} e utilizando \textbf{estimadores amostrais}:

\vspace{-1 em}
\begin{align}\label{ols:betahat:sums}
	\Aboxed{
		\betahatbold &= 
		\left( N^{-1} \sum_{i=1}^{N} \xvec_{i}' \xvec_{i} \right)^{-1}
	\left( N^{-1} \sum_{i=1}^{N} \xvec_{i}' y_{i} \right) }.
\end{align}

Podemos desenvolver essa equação para:

\vspace{-1 em}
\begin{align}
	\notag
	\betahatbold &= 
	\left( N^{-1} \sum_{i=1}^{N} \xvec_{i}' \xvec_{i} \right)^{-1}
	\left( N^{-1} \sum_{i=1}^{N} \xvec_{i}' (\xvec_{i} \betavec + \uvec_{i}) \right)
	\\
	\notag
	&=
	\left( N^{-1} \sum_{i=1}^{N} \xvec_{i}' \xvec_{i} \right)^{-1}
	\left( N^{-1} \sum_{i=1}^{N} \xvec_{i}' \xvec_{i} \betavec \right) +
	\left( N^{-1} \sum_{i=1}^{N} \xvec_{i}' \uvec_{i} \right)
	\\
	\label{ols:betahat:u}
	\Aboxed{
		\betahatbold &= 
		\betavec +
		\left( N^{-1} \sum_{i=1}^{N} \xvec_{i}' \xvec_{i} \right)^{-1}
	\left( N^{-1} \sum_{i=1}^{N} \xvec_{i}' \uvec_{i} \right) }.
\end{align}

%%%%%%%%%%%%%%%%%%%%%%%%%%%%%%%%%%%%%%%%%%%%%%%%%%%%%%
\subsubsection{Notação Matricial}
Empilhando as $N$ observações, obtemos a \textbf{Notação Matricial}:

\vspace{-1 em}
\begin{align} \label{ols:mod:mat}
	\yvec &= \Xmat \betavec + \uvec 
\end{align}

\begin{description}[noitemsep]
	\item [$\yvec$]  é um vetor $N \times 1$;

	\item [$\Xmat$]  é uma matriz $N \times K$ de regressores, com $N$ vetores, $\xvec_{i}$, de dimensão $1 \times K$ empilhados;

	\item [$\betavec$] é um vetor $K \times 1$;

	\item [$\uvec$] é um vetor $N \times 1$;
\end{description}

\vspace{-1 em}
\begin{align*}
	\yvec = 
	\begin{bmatrix}
		y_{1} \\ \vdots \\ y_{N}		
	\end{bmatrix};
	\quad
	\Xmat = 
	\begin{bmatrix}
		\xvec_{1} \\ \vdots \\ \xvec_{N}
	\end{bmatrix} = 
	\begin{bmatrix}
		x_{11}     & x_{12}     & \dots  & x_{1K} \\          
%	x_{21}     & x_{22}     & \dots  & x_{2K} \\         
		\vdots     & \vdots     & \ddots & \vdots \\        
		x_{N1} & x_{N2} & \dots  & x_{NK}		
	\end{bmatrix};
	\quad
	\uvec = 
	\begin{bmatrix}
		u_{1} \\ \vdots \\ u_{N}		
	\end{bmatrix}.
\end{align*}

As somas de vetores viram simples multiplicações de matrizes e a equação \eqref{ols:betahat:sums}, vira:

\vspace{-1 em}
\begin{align} 
	\label{ols:betahat:mat}
	\widehat{\betavec} = (N^{-1} \Xmat' \Xmat)^{-1} (N^{-1} \Xmat'\yvec)
	\implies
	\Aboxed{ \widehat{\betavec} = (\Xmat'\Xmat)^{-1}(\Xmat'\yvec)}
\end{align}

%%%%%%%%%%%%%%%%%%%%%%%%%%%%%%%%%%%%%%%%%%%%%%%%%%%%%%
\subsection{Valor Esperado} 

\vspace{-2 em}
\begin{align*} 
	\E ( \widehat{\betavec} ) 
	&= \E \left[ (\Xmat'\Xmat)^{-1}\Xmat'\yvec \right]
	= \E \left[ (\Xmat'\Xmat)^{-1}\Xmat'(\Xmat \betavec + \uvec) \right]
	= \E \left[ (\Xmat'\Xmat)^{-1}\Xmat'\Xmat \betavec + (\Xmat'\Xmat)^{-1}\Xmat'\uvec \right]
	\\
	&= \E (\betavec) + \E[(\Xmat'\Xmat)^{-1}\Xmat'\uvec ]
	\implies
	\boxed{
		\E ( \widehat{\betavec} ) 
	= \betavec + \E[(\Xmat'\Xmat)^{-1}\Xmat'\uvec ] }
\end{align*}

%%%%%%%%%%%%%%%%%%%%%%%%%%%%%%%%%%%%%%%%%%%%%%%%%%%%%%
\subsubsection{Viés} 

\vspace{-2 em}
\begin{align*} 
	\B( \widehat{\betavec} ) = \E ( \widehat{\betavec} ) - \betavec
	\implies
	\Aboxed{ \B( \widehat{\betavec} ) = \E[ (\Xmat'\Xmat)^{-1}\Xmat'\uvec ] }
\end{align*}

%%%%%%%%%%%%%%%%%%%%%%%%%%%%%%%%%%%%%%%%%%%%%%%%%%%%%%
\begin{remark}
	Sob \textbf{OLS.2'} e \textbf{OLS.3'}:

	\vspace{-2 em}
	\begin{align*}
		\E[(\Xmat'\Xmat)^{-1}\Xmat'\uvec ]
		= \E \left\{ \E\left[ (\Xmat'\Xmat)^{-1}\Xmat'\uvec | \Xmat \right]  \right\}  
		= \E \left\{  (\Xmat'\Xmat)^{-1}\Xmat'
			\underbracket[.75pt]{\E( \uvec | \Xmat )}_{= \zerovec}
		\right\} = 0
	\end{align*}

	\noindent
	ou seja, 
	$\B( \widehat{\betavec} ) = 0$, logo $\widehat{\betavec}$ é \textbf{não-viciado}.
	O que também é equivalente a  $\E( \widehat{\betavec} ) = \betavec$.
\end{remark}

%%%%%%%%%%%%%%%%%%%%%%%%%%%%%%%%%%%%%%%%%%%%%%%%%%%%%%
\subsection{Variância} 
Supondo \textbf{OLS.2'} e \textbf{OLS.3'}:

\vspace{-1 em}
\begin{align*} 
	\Var ( \widehat{\betavec} | \Xmat) 
	&= \E \left\{\left[ 
			\widehat{\betavec} - \E ( \widehat{\betavec} | \Xmat )
	\right]^2 | \Xmat \right\}
	\\
	&= \E \left\{ 
		\left[ \widehat{\betavec} - \E ( \widehat{\betavec} | \Xmat ) \right]
		\left[ \widehat{\betavec} - \E ( \widehat{\betavec} | \Xmat ) \right]'
	| \Xmat \right\}
	\\
	&= \E \left\{ 
		\left[ (\Xmat'\Xmat)^{-1}\Xmat' \uvec \right]
		\left[ (\Xmat'\Xmat)^{-1}\Xmat' \uvec \right]'
	| \Xmat \right\}
	\\
	&= \E \left[ (\Xmat'\Xmat)^{-1}\Xmat' \uvec \uvec' \Xmat (\Xmat'\Xmat)^{-1} | \Xmat \right]
	\\
	\Aboxed{
		\Var ( \widehat{\betavec} | \Xmat) 
		&= 
		(\Xmat'\Xmat)^{-1}\Xmat' 
		\E \left[ \uvec \uvec'| \Xmat \right]
	\Xmat (\Xmat'\Xmat)^{-1} }
\end{align*}

%%%%%%%%%%%%%%%%%%%%%%%%%%%%%%%%%%%%%%%%%%%%%%%%%%%%%%
\subsubsection{Homocedasticidade}
Supondo \textbf{homocedasticidade} e ausência de correlação serial: 
$\boxed{ \E \left[ \uvec \uvec'| \Xmat \right] = \sigma^2 \Imat_{N} }$.
Assim, 

\vspace{-1 em}
\begin{align*} 
	\Var ( \widehat{\betavec} | \Xmat) 
	= \sigma^2 (\Xmat'\Xmat)^{-1}\Xmat' \Imat_{N} \Xmat (\Xmat'\Xmat)^{-1}
	= \sigma^2 (\Xmat'\Xmat)^{-1}\Xmat'\Xmat (\Xmat'\Xmat)^{-1}
	\implies
	\Aboxed{ \Var ( \widehat{\betavec} | \Xmat) &= \sigma^2 (\Xmat'\Xmat)^{-1} }.
\end{align*}

%%%%%%%%%%%%%%%%%%%%%%%%%%%%%%%%%%%%%%%%%%%%%%%%%%%%%%
\clearpage
\section{Ausência de Exogeneidade Estrita}
Nem sempre poderemos supor \textbf{exogeneidade estrita}.
Por exemplo, no modelo com variável defasada mostrado abaixo:

\vspace{-1 em}
\begin{align*}
%  \left.
%  \begin{aligned}
	y_{t} &= \beta_{0} + \beta_{1} y_{t-1} + \beta_{2} x_{1t} + u_{t}
	\\
	y_{t-1} &= \beta_{0} + \beta_{1} y_{t-2} + \beta_{2} x_{1t-1} + u_{t-1}
%  \end{aligned}
%  \right\}
%  \implies
	\\
	y_{t} &=
	\beta_{0}(1 + \beta_{1})
	+
	\beta_{1}^2 y_{t-2}
	+
	\beta_{1} 
	\beta_{2} x_{1t-1} 
	+
	\beta_{2} x_{1t} 
	+
	u_{t}
	+
	\beta_{1} u_{t-1},
\end{align*}

\noindent
o erro é correlacionado com o regressor $y_{t-1}$.
Nesse caso, tentaremos obter apenas \textbf{consistência} e \textbf{variância assintótica} do estimador.
Para tanto, utilizaremos a equação \eqref{ols:betahat:u}: 

\vspace{-1 em}
\begin{align*}
	\betahatbold &= 
	\betavec +
	\left( N^{-1} \sum_{i=1}^{N} \xvec_{i}' \xvec_{i} \right)^{-1}
	\left( N^{-1} \sum_{i=1}^{N} \xvec_{i}' \uvec_{i} \right).
\end{align*}

%%%%%%%%%%%%%%%%%%%%%%%%%%%%%%%%%%%%%%%%%%%%%%%%%%%%%%
\vspace{1 em}
\begin{center}
	\Large{\red{Aqui comeceçaria a seção \ref{app:est}}.}
\end{center}
\vspace{1 em}
%%%%%%%%%%%%%%%%%%%%%%%%%%%%%%%%%%%%%%%%%%%%%%%%%%%%%%

%%%%%%%%%%%%%%%%%%%%%%%%%%%%%%%%%%%%%%%%%%%%%%%%%%%%%%
\subsection{Consistência}
Vamos definir a matriz $K \times K$, $\Amat \equiv \E(\xvec_{i}' \xvec_{i})$.
Supondo $\Amat$, finita e positiva definida, $\posto(\Amat) = K$.
Usando \textbf{LGN matricial} (Definição \ref{teo:lgn:mat:1} na página \pageref{teo:lgn:mat:1}), temos: 

\noindent
\red{[lembrar que as dimensões dos vetores estão invertidas: $1 \times K$ e \textbf{não} $K \times 1$]}

\vspace{-1 em}
\begin{align} \label{eq:Amat}
	N^{-1} \sum_{i=1}^{N} \xvec_{i}' \xvec_{i}
	\arrowp \Amat
	\implies
	\left( N^{-1} \sum_{i=1}^{N} \xvec_{i}' \xvec_{i} \right)^{-1}
	\arrowp \Amat^{-1}.
\end{align}

Além disso, vamos supor $\E(\xvec_{i}' u_{i}) = 0$, o que corresponde a $\Cov(\xvec_{i}, u_{i}) = 0$, ou seja, o erro $u_{i}$ \textbf{não} é correlacionado com os regressores da própria equação.
Isso é bem menos que exogeneidade estrita.
Então, 

\vspace{-1 em}
\begin{align*}
	N^{-1} \sum_{i=1}^{N} \xvec_{i}' u_{i} \arrowp \E(\xvec_{i}' u_{i}) = \zerovec_{K}.
\end{align*}

Logo,

\vspace{-1 em}
\begin{align*}
	\boxed{
		\betahatbold = 
		\betavec +
		\underbracket[1pt]{
			\left( \sum_{i=1}^{N} \xvec_{i}' \xvec_{i} \right)^{-1}
		\left( \sum_{i=1}^{N} \xvec_{i}' u_{i} \right)}_{\arrowp 0}
	}
\end{align*}

Então, 
$(\betahatbold - \betavec) \arrowp 0$ 
que é equivalente a 
$\betahatbold \arrowp \betavec$ e 
$\plim\betahatbold = \betavec$,
ou seja, $\betahatbold$ é \textbf{consistente} para $\betavec$.

%%%%%%%%%%%%%%%%%%%%%%%%%%%%%%%%%%%%%%%%%%%%%%%%%%%%%%
\subsection{Normalidade Assintótica} % do $\betahatbold^{OLS}$}

\vspace{-2 em}
\begin{align*}
	\betahatbold 
	&= 
	\betavec +
	\left( N^{-1} \sum_{i=1}^{N} \xvec_{i}' \xvec_{i} \right)^{-1}
	\left( N^{-1} \sum_{i=1}^{N} \xvec_{i}' u_{i} \right)
	\\
	(\betahatbold - \betavec)
	&=
	\left( N^{-1} \sum_{i=1}^{N} \xvec_{i}' \xvec_{i} \right)^{-1}
	\left( N^{-1} \sum_{i=1}^{N} \xvec_{i}' u_{i} \right)
	\\
	\sqrt{N} (\betahatbold - \betavec) &=
	\left( N^{-1} \sum_{i=1}^{N} \xvec_{i}' \xvec_{i} \right)^{-1}
	\left( N^{-1/2} \sum_{i=1}^{N} \xvec_{i}' u_{i} \right)
\end{align*}

\noindent
Supondo
$\E( x_{ik}^{2} u_{i}^{2} ) < + \infty$,
$k=1, \dots, K$, e definindo
$\Bmat = \E[\xvec_{i}' u_{i}' u_{i} \xvec_{i}] = \E[ u_{i}^2 \xvec_{i}' \xvec_{i} ]$.
Temos, pela Definição \ref{def:tcl} (\textbf{TCL}), que

\vspace{-1 em}
\begin{align} \label{eq:limxu}
	N^{-1/2} \sum_{i=1}^{N} \xvec_{i}' u_{i} \arrowd N(\zerovec, \Bmat)
	\implies
	N^{-1/2} \sum_{i=1}^{N} \xvec_{i}' u_{i} = \bigOp(1)
\end{align}

%%%%%%%%%%%%%%%%%%%%%%%%%%%%%%%%%%%%%%%%%%%%%%%%%%%%%%
\vspace{1 em}
Além disso, vamos utilizar a matriz \textbf{simétrica} e \textbf{não singular} $\Amat$ da equação \eqref{eq:Amat}
Assim, temos 

\vspace{-1 em}
\begin{align*}
	\sqrt{N} (\betahatbold - \betavec) &=
	\left( N^{-1} \sum_{i=1}^{N} \xvec_{i}' \xvec_{i} \right)^{-1}
	\left( N^{-1/2} \sum_{i=1}^{N} \xvec_{i}' u_{i} \right)
	\\ &=
	\left[ 
		\left( N^{-1} \sum_{i=1}^{N} \xvec_{i}' \xvec_{i} \right)^{-1} 
		+ \Amat^{-1} - \Amat^{-1}
	\right]
	\left( N^{-1/2} \sum_{i=1}^{N} \xvec_{i}' u_{i} \right)
	\\ &=
	\left[ 
		\left( N^{-1} \sum_{i=1}^{N} \xvec_{i}' \xvec_{i} \right)^{-1} 
		- \Amat^{-1}
	\right]
	\left( N^{-1/2} \sum_{i=1}^{N} \xvec_{i}' u_{i} \right)
	+ \Amat^{-1} 
	\left( N^{-1/2} \sum_{i=1}^{N} \xvec_{i}' u_{i} \right),
\end{align*}

\noident
Podemos inverter $\Amat$ porque ela tem posto completo (não singular).
Pelas propriedades de $\Amat$, temos:

\vspace{-1 em}
\begin{align*}
	N^{-1} \sum_{i=1}^{N} \xvec_{i}' \xvec_{i} \arrowp \Amat
	\implies
	\left( N^{-1} \sum_{i=1}^{N} \xvec_{i}' \xvec_{i} \right)^{-1}  - \Amat^{-1} = \litop(1).
\end{align*}

Então,

\vspace{-1 em}
\begin{align*}
	\sqrt{N} (\betahatbold - \betavec) &=
	\litop(1) \bigOp(1)
	+ \Amat^{-1} \left( N^{-1/2} \sum_{i=1}^{N} \xvec_{i}' u_{i} \right),
\end{align*}

Usando \eqref{eq:limxu} e o Lema \ref{lem:equiv:assin}.

\begin{align*}
	\Amat^{-1} \left( N^{-1/2} \sum_{i=1}^{N} \xvec_{i}' u_{i} \right)
	\arrowd 
	N(\zerovec, \Amat^{-1} \Bmat \Amat^{-1}).
\end{align*}

Lembrando que $\litop(1) \bigOp(1) = \litop(1)$, temos:

\vspace{-1 em}
\begin{align*}
	\sqrt{N} (\betahatbold - \betavec) \arrowd \mathcal{N}(\zerovec, \Amat^{-1} \Bmat \Amat^{-1})
	\implies
	\sqrt{N} (\betahatbold - \betavec) \asim \mathcal{N}(\zerovec, \Amat^{-1} \Bmat \Amat^{-1})
\end{align*}

%%%%%%%%%%%%%%%%%%%%%%%%%%%%%%%%%%%%%%%%%%%%%%%%%%%%%%
\subsection{Variância}

\vspace{-2 em}
\begin{align*}
	\Vmat &= \Amat^{-1} \Bmat \Amat^{-1} 
	\\
	\Vmat &=
	\E[ (\xvec_{i}' \xvec_{i} ) ]^{-1}
	\E[ (\xvec_{i}' u_{i}' u_{i} \xvec_{i} ) ]
	\E[ (\xvec_{i}' \xvec_{i} ) ]^{-1}
	\\
	\Vmat &=
	\E[ (\xvec_{i}' \xvec_{i} ) ]^{-1}
	\E[ (u_{i}^{2} \xvec_{i}' \xvec_{i} ) ]
	\E[ (\xvec_{i}' \xvec_{i} ) ]^{-1}.
\end{align*}

%%%%%%%%%%%%%%%%%%%%%%%%%%%%%%%%%%%%%%%%%%%%%%%%%%%%%%
\subsubsection{Homocedasticidade}

Sob \textbf{Homocedasticidade}, temos
$\Bmat = \E(u_{i}^2 \xvec_{i}' \xvec_{i}) &= \sigma^2 \E(\xvec_{i}' \xvec_{i})$, 
logo

\vspace{-1 em}
\begin{align*}
	\Aboxed{
	\Vmat &= \sigma^{2} \E[ (\xvec_{i}' \xvec_{i} ) ]^{-1} }.
\end{align*}

%%%%%%%%%%%%%%%%%%%%%%%%%%%%%%%%%%%%%%%%%%%%%%%%%%%%%%
\subsubsection{Estimador Amostral}

\vspace{-2 em}
\begin{align*}
	\widehat{\Vmat} &=
	\left( N^{-1} \sum_{i=1}^{N} \xvec_{i}' \xvec_{i} \right)^{-1}
	\left( N^{-1} \sum_{i=1}^{N} u_{i}^2 \xvec_{i}' \xvec_{i} \right)
	\left( N^{-1} \sum_{i=1}^{N} \xvec_{i}' \xvec_{i} \right)^{-1}
	\\ &=
	N
	\left( \sum_{i=1}^{N} \xvec_{i}' \xvec_{i} \right)^{-1}
	\left( \sum_{i=1}^{N} u_{i}^2 \xvec_{i}' \xvec_{i} \right)
	\left( \sum_{i=1}^{N} \xvec_{i}' \xvec_{i} \right)^{-1}
	\\
	\Aboxed{
		\widehat{\Vmat} &=
		N
		\left( \Xmat'\Xmat \right)^{-1}
		\left( \sum_{i=1}^{N} u_{i}^2 \xvec_{i}' \xvec_{i} \right)
	\left( \Xmat'\Xmat \right)^{-1} }.
\end{align*}

%%%%%%%%%%%%%%%%%%%%%%%%%%%%%%%%%%%%%%%%%%%%%%%%%%%%%%
\subsubsection{Variância do estimador de OLS }

\vspace{-1 em}
\begin{align*}
	\Var(\sqrt{N} \betahatbold) &= \Vmat
	\\
	\Var(\betahatbold) &= N^{-1} \Vmat
	\\
	\Aboxed{
		\Var(\betahatbold) &= 
		\left( \Xmat'\Xmat \right)^{-1}
		\left( \sum_{i=1}^{N} u_{i}^2 \xvec_{i}' \xvec_{i} \right)
	\left( \Xmat'\Xmat \right)^{-1} }.
\end{align*}

\noindent
A variância \textbf{Robusta} é:

\vspace{-1 em}
\begin{align*}
	\Aboxed{
		\widehat{\Var}(\betahatbold) &= 
		(\Xmat'\Xmat)^{-1} 
		\left( \sum_{i=1}^{N} \widehat{u}_{i}^{2} \xvec_{i}' \xvec_{i} \right)
	(\Xmat'\Xmat)^{-1} }.
\end{align*}

\noindent
A variância sob \textbf{Homocedasticidade} é:

\vspace{-1 em}
\begin{align*}
	\Aboxed{\widehat{\Var}(\betahatbold) &= \widehat{\sigma}^{2} (\Xmat' \Xmat)^{-1}} .
\end{align*}

%%%%%%%%%%%%%%%%%%%%%%%%%%%%%%%%%%%%%%%%%%%%%%%%%%%%%%
\clearpage
\section{System OLS (SOLS)}

\noindent
\citet[C.7 -- Estimating Systems of Equations by OLS and GLS, p.143--179]{wool-2010}\\
\citet[Sec.7.3 -- System OLS Estimation of a Multivariate Linear System, p.147]{wool-2010} 

%%%%%%%%%%%%%%%%%%%%%%%%%%%%%%%%%%%%%%%%%%%%%%%%%%%%%%
\subsection{Modelo Linear}

Assumimos que temos as seguintes observações \textit{cross section} $iid$:
$\seq{ (\Xmat_{i}, \yvec_{i}): i=1, \dots, N}$, onde:

\vspace{-1 em}
\begin{itemize}[itemsep = -1ex]
	\item [$\Xmat_{i}$]  é uma matriz $G \times K$ e contém as variáveis explicativas que aparecem em qualquer lugar do sistema.
	\item [$\yvec_{i}$]  é um vetor $G \times 1$, que contém as variáveis dependentes para todas as equações $G$ (ou períodos de tempo, no caso de dados de painel).
\end{itemize}

O modelo linear multivariado para uma \red{observação (draw)} aleatória da população pode ser expresso como:

\vspace{-1 em}
\begin{align}\label{sols:mod}
	\Aboxed{
		\yvec_{i} = \Xmat_{i} \betavec + \uvec_{i} \, , \quad i=1, \dots, N,
	}
\end{align}

\noindent
onde:

\vspace{-1 em}
\begin{itemize}[itemsep = -1ex]
	\item [$\betavec$] é um vetor $K \times 1$ de parâmetros de interesse; e
	\item [$\uvec_{i}$] é um vetor $G \times 1$ de não observáveis.
\end{itemize}

A equação \eqref{sols:mod} explica as $G$ variáveis $y_{i1}, \dots, y_{iG}$ em termos de $\Xmat_{i}$ e das não observáveis $\uvec_{i}$.
Por causa da hipótese de amostra aleatória podemos escrever tudo em temos de uma observação genérica.

%%%%%%%%%%%%%%%%%%%%%%%%%%%%%%%%%%%%%%%%%%%%%%%%%%%%%%
\subsection{Hipóteses}
\noindent
\citet[Sec.7.3.1]{wool-2010}

\paragraph{SOLS.1} $\E(\Xmat_{i}' \uvec_{i}) = \zerovec_{K \times 1}$.

\paragraph{SOLS.2} $\Amat \equiv \E( \Xmat_{i}' \Xmat_{i} )$ é não singular (tem posto pleno, posto igual a $K$). 

%%%%%%%%%%%%%%%%%%%%%%%%%%%%%%%%%%%%%%%%%%%%%%%%%%%%%%
\subsection{Estimação}
Note que, sob \textbf{SOLS.1}, temos:

\vspace{-1 em}
\begin{align} 
	\notag
	\E[ \Xmat_{i}' ( \yvec_{i} - \Xmat_{i} \betavec ) ] &= \zerovec
	\\
	\notag
	\E( \Xmat_{i}' \Xmat_{i} ) \betavec &= \E( \Xmat_{i}' \yvec_{i} )  
	\\
	\label{sols:beta}
	\Aboxed{
		\betavec &=
		\left[ \E( \Xmat_{i}' \Xmat_{i} )  \right]^{-1}
	\E( \Xmat_{i}' \yvec_{i} )  }
\end{align}

%%%%%%%%%%%%%%%%%%%%%%%%%%%%%%%%%%%%%%%%%%%%%%%%%%%%%%
Usando estimadores amostrais:

\vspace{-1 em}
\begin{align} \label{sols:betahat}
	\Aboxed{
		\betahatbold^{SOLS} &=
		\left( N^{-1} \sum_{i=1}^{N} \Xmat_{i}' \Xmat_{i}   \right)^{-1}
		\left( N^{-1} \sum_{i=1}^{N} \Xmat_{i}' \yvec_{i}   \right)
	}.
\end{align}

Para computar $\betahatbold$ usando linguagem de computação é mais fácil utilizar a notação matricial.
Para tanto, cortamos os $N^{-1}$ e substituímos os somatórios por multiplicações de matrizes.

\vspace{-1 em}
\begin{align} 
	\label{sols:betahat:mat}
	\Aboxed{
		\betahatbold^{SOLS} &=
		\left(  \Xmat' \Xmat   \right)^{-1} \left(  \Xmat' \yvec   \right)
	}
\end{align}

\noindent
onde
\begin{description}[noitemsep]
	\item [$\Xmat \equiv (\Xmat_{1}', \dots, \Xmat_{N}')$]  é uma matriz $NG \times K$ dos $\Xmat_{i}$ empilhados.

	\item [$\yvec \equiv (\yvec_{1}', \dots, \yvec_{N}')$] é um vetor $NG \times 1$ das observações $\yvec_{i}$ empilhadas.
\end{description}

%%%%%%%%%%%%%%%%%%%%%%%%%%%%%%%%%%%%%%%%%%%%%%%%%%%%%%
\subsection{Consistência}

Para provarmos a \textbf{consistência} do estimador, usamos as equações \eqref{sols:betahat} e \eqref{sols:mod}:

\vspace{-1 em}
\begin{align*}
	\betahatbold^{SOLS} &=
	\left( N^{-1}\sum_{i=1}^{N} \Xmat_{i}' \Xmat_{i}   \right)^{-1}
	\left( N^{-1}\sum_{i=1}^{N} \Xmat_{i}' \yvec_{i}   \right)
	\\ &=
	\left( N^{-1}\sum_{i=1}^{N} \Xmat_{i}' \Xmat_{i}   \right)^{-1}
	\left[ N^{-1}\sum_{i=1}^{N} \Xmat_{i}' (\Xmat_{i} \betavec + \uvec_{i})   \right]
	\\ &=
	\left( N^{-1}\sum_{i=1}^{N} \Xmat_{i}' \Xmat_{i}   \right)^{-1}
	\left( N^{-1}\sum_{i=1}^{N} \Xmat_{i}' \Xmat_{i} \betavec    \right)
	+
	\left( N^{-1}\sum_{i=1}^{N} \Xmat_{i}' \Xmat_{i}   \right)^{-1}
	\left( N^{-1}\sum_{i=1}^{N} \Xmat_{i}' \uvec_{i}   \right).
\end{align*}

\noindent
E chegamos em:

\vspace{-1 em}
\begin{align}\label{sols:betahat:u}
	\Aboxed{
		\betahatbold^{SOLS} &=
		\betavec
		+
		\left( N^{-1}\sum_{i=1}^{N} \Xmat_{i}' \Xmat_{i}   \right)^{-1}
		\left( N^{-1}\sum_{i=1}^{N} \Xmat_{i}' \uvec_{i}   \right)
	}.
\end{align}

Por \textbf{SOLS.1}:

\vspace{-1 em}
\begin{align*}
	N^{-1} \sum_{i=1}^{N} \Xmat_{i}' \uvec_{i} \arrowp \zerovec;
\end{align*}

\noindent e por \textbf{SOLS.2}

\vspace{-1 em}
\begin{align*}
	\left( N^{-1} \sum_{i=1}^{N} \Xmat_{i}' \Xmat_{i} \right)^{-1} \arrowp \Amat^{-1}.
\end{align*}

Resumimos esse resultado pelo seguinte Teorema:

%%%%%%%%%%%%%%%%%%%%%%%%%%%%%%%%%%%%%%%%%%%%%%%%%%%%%%
\begin{teo}[Consistência do SOLS]\label{SOLS:const}
	Sob Hipóteses \textbf{SOLS.1} e \textbf{SOLS.2}, temos 
	\begin{align*}
		\Aboxed{
			\betahatbold^{SOLS} \arrowp \betavec
		}.
	\end{align*}
\end{teo}
%%%%%%%%%%%%%%%%%%%%%%%%%%%%%%%%%%%%%%%%%%%%%%%%%%%%%%

%%%%%%%%%%%%%%%%%%%%%%%%%%%%%%%%%%%%%%%%%%%%%%%%%%%%%%
\subsection{Normalidade Assintótica}

De \eqref{sols:betahat:u}:

\vspace{-2 em}
\begin{align*} 
	\betahatbold  &=
	\betavec +
	\left( N^{-1} \sum_{i=1}^{N} \Xmat_{i}' \Xmat_{i}   \right)^{-1}
	\left( N^{-1} \sum_{i=1}^{N} \Xmat_{i}' \uvec_{i}   \right)
	\\ 
	(\betahatbold - \betavec) &= 
	\left( N^{-1} \sum_{i=1}^{N} \Xmat_{i}' \Xmat_{i}   \right)^{-1}
	\left( N^{-1} \sum_{i=1}^{N} \Xmat_{i}' \uvec_{i}   \right).
\end{align*}

\noindent
E chegamos em:

\vspace{-1 em}
\begin{align} \label{sols:betahat:rootn}
	\Aboxed{
		\sqrt{N}(\betahatbold - \betavec) &= 
		\left( N^{-1} \sum_{i=1}^{N} \Xmat_{i}' \Xmat_{i}   \right)^{-1}
		\left( N^{-1/2} \sum_{i=1}^{N} \Xmat_{i}' \uvec_{i}   \right)
	}.
\end{align}

Uma vez que $\E(\Xmat_{i}' \uvec_{i})=0$, sob a hipótese \textbf{SOLS.1}, a definição \ref{def:tcl} (\textbf{TCL}) implica que:

\vspace{-1 em}
\begin{align*} 
	N^{-1/2} \sum_{i=1}^{N} \Xmat_{i} \uvec_{i} \arrowd N(\zerovec, \Bmat),
\end{align*}

\noindent
onde

\vspace{-1 em}
\begin{align*} 
	\Bmat \equiv \E(\Xmat_{i}' \uvec_{i} \uvec_{i}' \Xmat_{i}) \equiv \Var(\Xmat_{i} \uvec_{i}).
\end{align*}

\noindent
Em particular,

\vspace{-1 em}
\begin{align*} 
	N^{-1/2} \sum_{i=1}^{N} \Xmat_{i} \uvec_{i} = \bigOp(1).
\end{align*}

Porém,

\vspace{-1 em}
\begin{align*} 
	\left( N^{-1} \sum_{i=1}^{N} \Xmat_{i}' \Xmat_{i} \right)^{-1} = (\Xmat'\Xmat/N)^{-1} 
	=
	\Amat^{-1} + \litop(1).
\end{align*}

\noindent
Sendo assim,

\vspace{-1 em}
\begin{align*} 
	\sqrt{N}(\betahatbold - \betavec) &= 
	\left[ 
		\Amat^{-1} +
		\left( N^{-1} \sum_{i=1}^{N} \Xmat_{i}' \Xmat_{i}   \right)^{-1}
		- \Amat^{-1}
	\right]
	\left( N^{-1/2} \sum_{i=1}^{N} \Xmat_{i}' \uvec_{i}   \right)
	\\ &=
	\Amat^{-1}\left( N^{-1/2} \sum_{i=1}^{N} \Xmat_{i}' \uvec_{i}   \right)
	+
	[(\Xmat'\Xmat/N)^{-1} - \Amat^{-1}]
	\left( N^{-1/2} \sum_{i=1}^{N} \Xmat_{i}' \uvec_{i}   \right)
	\\&=
	\Amat^{-1}\left( N^{-1/2} \sum_{i=1}^{N} \Xmat_{i}' \uvec_{i}   \right)
	+ \litop(1) \bigOp(1)
	\\&=
	\Amat^{-1}\left( N^{-1/2} \sum_{i=1}^{N} \Xmat_{i}' \uvec_{i}   \right)
	+ \litop(1)
\end{align*}

\vspace{-1 em}
\begin{align}\label{sols:betahat:asim}
	\Aboxed{
		\sqrt{N}(\betahatbold - \betavec)
		\arrowd
		N(\zerovec, \Amat^{-1} \Bmat \Amat^{-1})
	}
\end{align}
%%%%%%%%%%%%%%%%%%%%%%%%%%%%%%%%%%%%%%%%%%%%%%%%%%%%%%

%%%%%%%%%%%%%%%%%%%%%%%%%%%%%%%%%%%%%%%%%%%%%%%%%%%%%%
\subsection{Variância Assintótica}

\paragraph{SOLS.3: Homocedasticidade}
$\E(\Xmat_{i}' \uvec_{i} \uvec_{i}' \Xmat_{i}) = \sigma^{2} \E(\Xmat_{i}' \Xmat_{i})$.

%%%%%%%%%%%%%%%%%%%%%%%%%%%%%%%%%%%%%%%%%%%%%%%%%%%%%%
\vspace{1 em} 
De \eqref{sols:betahat:asim}, vamos definir $\Vmat = \Amat^{-1} \Bmat \Amat^{-1}$.
Sob \textbf{SOLS.3},
$\Vmat = \sigma^{2} \left[ \E(\Xmat_{i}' \Xmat_{i}) \right]^{-1}$.
Estimando:

\vspace{-1 em}
\begin{align*}
	\sigmahat^{2}
	=
	\frac{1}{NG - K}
	\sum_{i =1}^{N} \sum_{g=1}^{G} \uhat^2_{ig}
\end{align*}

\noindent onde $\uhat_{ig} = y_{ig} - \xvec_{ig} \betavechat^{SOLS}$

%%%%%%%%%%%%%%%%%%%%%%%%%%%%%%%%%%%%%%%%%%%%%%%%%%%%%%
\subsubsection{A Matriz Robusta}

\vspace{-2 em}
\begin{align*}
	\Vmathat =
	\left( \Xmat' \Xmat \right)^{-1}
	\left( \sum_{i=1}^{N} \Xmat_{i}' \uvechat_{i}' \uvechat_{i} \Xmat \right)
	\left( \Xmat' \Xmat \right)^{-1}
\end{align*}

\vspace{-1 em}
\begin{align*}
	\sum_{i=1}^{N} \Xmat_{i}' \Omegahat \Xmat_{i} \arrowp 
	\E(\Xmat_{i} \Omega \Xmat_{i})
\end{align*}
Mas \textbf{não} é verdade que $\Omegahat \arrowp \Omega$.

%%%%%%%%%%%%%%%%%%%%%%%%%%%%%%%%%%%%%%%%%%%%%%%%%%%%%%
\begin{itemize}[noitemsep]
	\item
		Havendo constante, \textbf{SOLS.1} $\implies \E(\uvec_{i})=0$
	\item
		Ausência de correlação entre os regressores de uma equação e o erro da própria equação $\implies$ \textbf{SOLS.1}.
\end{itemize}


%%%%%%%%%%%%%%%%%%%%%%%%%%%%%%%%%%%%%%%%%%%%%%%%%%%%%%
\subsubsection{Variância Asstintótica}

%%%%%%%%%%%%%%%%%%%%%%%%%%%%%%%%%%%%%%%%%%%%%%%%%%%%%%
\begin{center}
	\red{REVER}
\end{center}
%%%%%%%%%%%%%%%%%%%%%%%%%%%%%%%%%%%%%%%%%%%%%%%%%%%%%%

%%%%%%%%%%%%%%%%%%%%%%%%%%%%%%%%%%%%%%%%%%%%%%%%%%%%%%
\vspace{-1 em}
\begin{align}\label{eq:avar:sols}
	\Avar(\betahatbold^{SOLS}) = \Amat^{-1} \Bmat \Amat^{-1}/N.
\end{align}

Assim, $\Avar(\betahatbold^{SOLS})$ tende a zero a uma taxa $1/N$, como esperado.
Estimação consistente de $\Amat$ é:

\vspace{-1 em}
\begin{align*}
	\widehat{\Amat} \equiv \Xmat'\Xmat/N = N^{-1} \sum_{i=1}^{N} \Xmat_{i}'\Xmat_{i}
\end{align*}

Um estimador consistente para $\Bmat$ pode ser achado usando o princípio da analogia.

\vspace{-1 em}
\begin{align*}
	\Bmat = \E(\Xmat_{i}' \uvec_{i} \uvec_{i}' \Xmat_{i}), 
	\quad
	N^{-1}\sum_{i=1}^{N} \Xmat_{i}' \uvec_{i} \uvec_{i}' \Xmat_{i} \arrowp \Bmat.
\end{align*}

Uma vez que não podemos observar $\uvec_{i}$, usamos os resíduos da estimação de SOLS:

\vspace{-1 em}
\begin{align*}
	\widehat{\uvec}_{i} \equiv \yvec_{i} - \Xmat_{i} \betahatbold 
	=
	\uvec_{i} - \Xmat_{i} (\betahatbold - \betavec).
\end{align*}

Assim, definimos $\Bmathat$ e usando LGN, podemos mostrar que:

\vspace{-1 em}
\begin{align*}
	\Bmathat \equiv N^{-1}\sum_{i=1}^{N} \Xmat_{i}' \uvechat_{i} \uvechat_{i}' \Xmat_{i} 
	\arrowp \Bmat.
\end{align*}

\noindent
onde supomos que certos momentos envolvendo $\Xmat_{i}$ e $\uvec_{i}$ são finitos.

Portanto, $\Avar[\sqrt{N}(\betahatbold - \betavec)]$ é \textbf{consistentemente} estimado por $\Amathat^{-1} \Bmathat \Amathat^{-1}$, e $\Avar(\betahatbold)$ é estimado como:

\vspace{-1 em}
\begin{align*}
	\Vmathat \equiv 
	\left( \sum_{i=1}^{N} \Xmat_{i}' \Xmat_{i}  \right)^{-1}
	\left( \sum_{i=1}^{N} \Xmat_{i}' \uvechat_{i} \uvechat_{i}'  \Xmat_{i}  \right)
	\left( \sum_{i=1}^{N} \Xmat_{i}' \Xmat_{i}  \right)^{-1}.
\end{align*}

Sob as hipóteses \textbf{SOLS.1} e \textbf{SOLS.2}, nós fazemos inferência em $\betavec$ como $\betahatbold$ fosse normalmente distribuído com média $\betavec$ e variância $\Vmathat$.


%%%%%%%%%%%%%%%%%%%%%%%%%%%%%%%%%%%%%%%%%%%%%%%%%%%%%%
\clearpage
\section{Dados de Painel (POLS)}
\noindent
\citet[C.7 -- Estimating Systems of Equations by OLS and GLS. p.143-179]{wool-2010} \\
\citet[Sec.7.8 -- The Linear Panel Data Model, Revisited. p.169]{wool-2010} 

%%%%%%%%%%%%%%%%%%%%%%%%%%%%%%%%%%%%%%%%%%%%%%%%%%%%%%
\subsection{Modelo Linear para Dados de Painel}

No caso de dados de painel, temos a seguinte amostra aleatória:

\vspace{-1 em}
\begin{align}\label{pols:mod}
	y_{it} &= \xvec_{it} \betavec + u_{it} \, , \quad i=1, \dots, N, \quad t=1, \dots, T.
\end{align}

\noindent onde
\vspace{-1 ex}
\begin{description}[noitemsep]
	\item[$y_{it}$] é um escalar.
	\item[$\betavec$] é um vetor $K \times 1$.
	\item[$\xvec_{it}$] é um vetor $1 \times K$.
	\item[$u_{it}$] é um escalar.
\end{description}

\vspace{-2 em}
\begin{align*}
	\underset{1 \times K}{\xvec_{it}} = 
	\begin{bmatrix}
		x_{1,it} & \dots & x_{K, it}
	\end{bmatrix}
	\quad
	\underset{K \times 1}{\betavec} = 
	\begin{bmatrix}
		\beta_{1} \\ \vdots \\ \beta_{K}
	\end{bmatrix}
\end{align*}

%%%%%%%%%%%%%%%%%%%%%%%%%%%%%%%%%%%%%%%%%%%%%%%%%%%%%%
\paragraph{Notação Vetorial:}
$\yvec_{i} = \Xmat_{i} \betavec + \uvec_{i}$,  pra cada $i=1, \dots, N$.

\noindent onde
\vspace{-1 ex}
\begin{description}[noitemsep]
	\item[$\yvec_{i}$] é um vetor $T \times 1$.
	\item[$\betavec$] é um vetor $K \times 1$.
	\item[$\Xmat_{i}$] é uma matriz $T \times K$.
	\item[$\uvec_{it}$] é um vetor $T \times 1$.
\end{description}

% X vecs
\vspace{-1 em}
\begin{align*}
	\underset{T \times K}{\Xmat_{i}} = 
	\begin{bmatrix}
		\xvec_{1} \\ \vdots \\ \xvec_{T}
	\end{bmatrix}
	=
	\begin{bmatrix}
		x_{1,i1} & \dots  & x_{K, i1}	\\
		\vdots          & \ddots &  \vdots \\
		x_{1,iT} & \dots  & x_{K, iT}
	\end{bmatrix}
	\quad
	\underset{T \times 1}{\yvec_{i}} = 
	\begin{bmatrix}
		y_{i1} \\ \vdots \\ y_{iT}
	\end{bmatrix}
	\quad
	\underset{T \times 1}{\uvec_{i}} = 
	\begin{bmatrix}
		u_{i1} \\ \vdots \\ u_{iT}
	\end{bmatrix}
\end{align*}

%%%%%%%%%%%%%%%%%%%%%%%%%%%%%%%%%%%%%%%%%%%%%%%%%%%%%%
\paragraph{Notação Matricial:}
$\yvec = \Xmat \betavec + \uvec$

\noindent onde
\vspace{-1 ex}
\begin{description}[noitemsep]
	\item[$\yvec$] é um vetor $NT \times 1$.
	\item[$\betavec$] é um vetor $K \times 1$.
	\item[$\Xmat$] é uma matriz $NT \times K$.
	\item[$\uvec$] é um vetor $NT \times 1$.
\end{description}

% XMAT
\begin{align*}
	\underset{NT \times K}{\ddot{\Xmat}} = 
	\begin{bmatrix}
		\ddot{\Xmat}_{1} \\ \vdots \\ \ddot{\Xmat}_{N}
	\end{bmatrix}
	=
	\begin{bmatrix}
		\ddot{x}_{1,11} & \dots & \ddot{x}_{K, 11} \\
		\vdots          &       &  \vdots          \\
		\ddot{x}_{1,1T} & \dots & \ddot{x}_{K, 1T} \\
		\ddot{x}_{1,21} & \dots & \ddot{x}_{K, 21} \\
		\vdots          &       &  \vdots          \\
		\ddot{x}_{1,2T} & \dots & \ddot{x}_{K, 2T} \\
		\vdots          &       &  \vdots          \\
		\ddot{x}_{1,N1} & \dots & \ddot{x}_{K, N1} \\
		\vdots          &       &  \vdots          \\
		\ddot{x}_{1,NT} & \dots & \ddot{x}_{K, NT}
	\end{bmatrix}
	\quad
% Y vecs
	\underset{NT \times 1}{\ddot{\yvec}} = 
	\begin{bmatrix}
		\ddot{\yvec}_{1} \\ \vdots \\ \ddot{\yvec}_{N}
	\end{bmatrix}
	=
	\begin{bmatrix}
		\ddot{y}_{11} \\ \vdots \\ \ddot{y}_{1T} \\
		\ddot{y}_{21} \\ \vdots \\ \ddot{y}_{2T} \\
		\vdots \\
		\ddot{y}_{N1} \\ \vdots \\ \ddot{y}_{NT} \\
	\end{bmatrix}
	\quad
	\underset{NT \times 1}{\ddot{\uvec}} = 
	\begin{bmatrix}
		\ddot{\uvec}_{1} \\ \vdots \\ \ddot{\uvec}_{N}
	\end{bmatrix}
	=
	\begin{bmatrix}
		\ddot{u}_{11} \\ \vdots \\ \ddot{u}_{1T} \\
		\ddot{u}_{21} \\ \vdots \\ \ddot{u}_{2T} \\
		\vdots \\
		\ddot{u}_{N1} \\ \vdots \\ \ddot{u}_{NT} \\
	\end{bmatrix}
\end{align*}

%%%%%%%%%%%%%%%%%%%%%%%%%%%%%%%%%%%%%%%%%%%%%%%%%%%%%%
\subsection{Estimação}

\begin{align*}
	\Xmat' \Xmat 
	=
	\sum_{i=1}^{N} \Xmat_{i}' \Xmat_{i}
	=
	\sum_{i=1}^{N} \sum_{t=1}^{T} \xvec_{it}' \xvec_{it};
	\quad
	\Xmat' \yvec 
	=
	\sum_{i=1}^{N} \Xmat_{i}' \yvec_{i}
	=
	\sum_{i=1}^{N} \sum_{t=1}^{T} \xvec_{it}' y_{it}.
\end{align*}

Portanto, podemos escrever $\betahatbold$ como:

\vspace{-1 em}
\begin{align} \label{betahat:POLS}
	\Aboxed{
		\betahatbold^{POLS} =
		\left( \sum_{i=1}^{N} \sum_{t=1}^{T} \xvec_{it}' \xvec_{it} \right)^{-1}
		\left( \sum_{i=1}^{N} \sum_{t=1}^{T} \xvec_{it}' y_{it} \right)
	}.
\end{align}

Este estimador é chamado \textbf{estimador de Mínimos Quadrados Agrupados (POLS)} porque ele corresponde a rodar uma regressão OLS nas observações agrupadas através de $i$ e $t$. 
% This estimator is called the \textbf{pooled ordinary least squares (POLS) estimator} because it corresponds to running OLS oin the observation pooled across $i$ and $t$.
O estimador da equação \eqref{betahat:POLS} é o mesmo para unidades de \textit{cross section} amostradas em diferentes pontos do tempo.
%%%%%%%%%%%%%%%%%%%%%%%%%%%%%%%%%%%%%%%%%%%%%%%%%%%%%%

%%%%%%%%%%%%%%%%%%%%%%%%%%%%%%%%%%%%%%%%%%%%%%%%%%%%%%
\subsection{Hipóteses}

%%%%%%%%%%%%%%%%%%%%%%%%%%%%%%%%%%%%%%%%%%%%%%%%%%%%%%
\paragraph{POLS.1} $\E(\Xmat_{i}' \uvec_{i}) = \E(\xvec_{it}'u_{it}) = \zerovec_{K \times 1}$, para cada
$i=1, \dots, N$ e $t=1, \dots, T$.

\noindent
De fato, \textbf{POLS.1} $\implies$ \textbf{SOLS.1}.

%%%%%%%%%%%%%%%%%%%%%%%%%%%%%%%%%%%%%%%%%%%%%%%%%%%%%%
\paragraph{Obs:}
O modelo \eqref{pols:mod} permite $y_{i,t-1}$ como regressor, se satisfeita \textbf{POLS.1}.
%%%%%%%%%%%%%%%%%%%%%%%%%%%%%%%%%%%%%%%%%%%%%%%%%%%%%%

%%%%%%%%%%%%%%%%%%%%%%%%%%%%%%%%%%%%%%%%%%%%%%%%%%%%%%
\clearpage
\section{Alguns Testes}

Lembrando a equação do modelo \eqref{pols:mod}:

\vspace{-2 em}
\begin{align*}
	y_{it} = \xvec_{it} \betavec + u_{it}
\end{align*}

%%%%%%%%%%%%%%%%%%%%%%%%%%%%%%%%%%%%%%%%%%%%%%%%%%%%%%
\subsection{Autocorrelação dos Resíduos}

Nos dois testes apresentado, primeiro precisamos guardar os resíduos estimado.
Para tanto, rodamos a regressão do modelo \eqref{pols:mod} e guardamos os resíduos:

\vspace{-1 em}
\begin{align} \label{pols:uhat}
	\uhat_{it} = y_{it} - \xvec_{it} \betavechat^{POLS}.
\end{align}

%%%%%%%%%%%%%%%%%%%%%%%%%%%%%%%%%%%%%%%%%%%%%%%%%%%%%%
\paragraph{Com Exogeneidade Estrita}
Sob exogeneidade estrita, rodamos a seguinte regressão dos resíduos:

\vspace{-1.5 em}
\begin{align*}
	\uhat_{it} = \delta_{0} + \delta \uhat_{it-1}  + \err_{it}
	\, , \quad i=1,\dots, N; \quad t=1,\dots, T.
\end{align*}

\noindent
Então, testamos

\vspace{-1 em}
\begin{align*}
% \left\{
% \begin{aligned}
	H_{0} : \delta_{1} = 0
	\\
	H_{1} : \delta_{1} \neq 0
% \end{aligned}
% \right.
\end{align*}

\noindent
via teste $t$ (pode ser robusto).

%%%%%%%%%%%%%%%%%%%%%%%%%%%%%%%%%%%%%%%%%%%%%%%%%%%%%%
\paragraph{Sem Exogeneidade Estrita (Apenas exogeneidade contemporânea)}
Sem exogeneidade estrita, rodamos a seguinte regressão dos resíduos:

\vspace{-1.5 em}
\begin{align*}
	\uhat_{it} = \xvec_{it} \alphavec + \delta \uhat_{it-1}  + \err_{it}
	\, , \quad i=1,\dots, N; \quad t=1,\dots, T.
\end{align*}

\noindent
Então, testamos

\vspace{-1 em}
\begin{align*}
% \left\{
% \begin{aligned}
	H_{0} : \delta_{1} = 0
	\\
	H_{1} : \delta_{1} \neq 0
% \end{aligned}
% \right.
\end{align*}

\noindent
via teste $t$. 

%%%%%%%%%%%%%%%%%%%%%%%%%%%%%%%%%%%%%%%%%%%%%%%%%%%%%%
\subsection{Heterocedasticidade}
Com os resíduos da equação \eqref{pols:uhat}, rodamos a seguinte regressão:

\vspace{-1 em}
\begin{align*}
	\uhat^{2}_{it} = \gamma_{0} + \gamma_{1} \yhat_{it}' + \gamma_{2} \yhat^{2}_{it} + \err_{it}
\end{align*}

\noindent
onde $\yhat_{it} = \xvec_{it} \betavechat^{POLS}$.
Definindo $\hvec_{it} = ( \yhat_{it}', \yhat^{2}_{it} )$, podemos reescrever a equação acima como

\vspace{-1 em}
\begin{align*}
	\uhat^{2}_{it} = \gamma_{0} + \hvec_{it} \gammavec + \err_{it}
\end{align*}

\noindent
Então, testamos

\vspace{-1 em}
\begin{align*}
% \left\{
% \begin{aligned}
	H_{0} &: \gammavec = \zerovec \quad (\gamma_{1} = 0 \text{ e } \gamma_{2} = 0 )
	\\
	H_{1} &: \gammavec \neq \zerovec
% \end{aligned}
% \right.
\end{align*}

\noindent
via teste de Wald.

%%%%%%%%%%%%%%%%%%%%%%%%%%%%%%%%%%%%%%%%%%%%%%%%%%%%%%
\subsection{Teste de Wald}

Se é verdade que

\vspace{-1 em}
\begin{align*}
	\sqrt{N} (\betavechat - \betavec) 
	\arrowd
	\mathcal{N}(\zerovec, \Vmathat).
\end{align*}

\noindent
Seja $\Rmat$ uma matriz $Q \times K$ com $Q \leq K$ e $\posto(\Rmat) = Q$ (posto pleno), então

\vspace{-1 em}
\begin{align*}
	\sqrt{N} \Rmat (\betavechat - \betavec) 
	\arrowd
	\mathcal{N}(\zerovec, \Rmat \Vmathat \Rmat').
\end{align*}

e

\vspace{-1 em}
\begin{align*}
	\left[ \sqrt{N} \Rmat (\betavechat - \betavec)  \right]'
	\left( \Rmat \Vmat \Rmat' \right)^{-1}
	\left[ \sqrt{N} \Rmat (\betavechat - \betavec)  \right]
	\asim
	\chisq_{Q} 
\end{align*}

O resulado acima vale para $\Vmathat$ no lugar de $\Vmat$, desde que $\Vmathat \arrowp \Vmat$.
Ou seja, vale para estimadores \textbf{consistentes} de $\Vmat$.

%%%%%%%%%%%%%%%%%%%%%%%%%%%%%%%%%%%%%%%%%%%%%%%%%%%%%%
\paragraph{Teste de Wald}

\begin{align*}
% \left\{
% \begin{aligned}
	H_{0} &: \Rmat \betavec = \rvec
	\\
	H_{1} &: \Rmat \betavec \neq \rvec
% \end{aligned}
% \right.
\end{align*}

A estatística do teste acima é:

\vspace{-1 em}
\begin{align*}
	N
	\left[ \Rmat \betavechat - \rvec  \right]'
	\left( \Rmat \Vmathat \Rmat' \right)^{-1}
	\left[ \Rmat \betavechat - \rvec  \right]
	\asim
	\chisq_{Q} 
\end{align*}

%%%%%%%%%%%%%%%%%%%%%%%%%%%%%%%%%%%%%%%%%%%%%%%%%%%%%%
\paragraph{Remarks}
\begin{enumerate}
	\item $\Vmathat$ pode ser a matriz robusta.

	\item Uma aproximação, via distribuição $F$ é dado por:
		\begin{align*}
			\frac{\text{Est. Teste}}{Q} \asim F_{Q, N-K}
		\end{align*}

		\noindent
		com $\Avar(\betavechat) = \frac{N}{N-K} \Vmathat$.
\end{enumerate}
%%%%%%%%%%%%%%%%%%%%%%%%%%%%%%%%%%%%%%%%%%%%%%%%%%%%%%


%%%%%%%%%%%%%%%%%%%%%%%%%%%%%%%%%%%%%%%%%%%%%%%%%%%%%%
\clearpage
\section{Modelo de Efeitos Não Observados (UEM)}
\noindent
\citet[C.10 -- Basic Linear Unobserved Effects Panel Data Models, p.247--291]{wool-2010} \\
\citet[Sec.10.1 -- Motivation: The Omitted Variables Problem, p.247]{wool-2010}\\
\citet[Sec.10.2 -- Assumptions about the Unobserved Effects and Explanatory Variables, p.251]{wool-2010}\\
\citet[Sec.10.3 -- Estimating UEM by POLS, p.256]{wool-2010}

%%%%%%%%%%%%%%%%%%%%%%%%%%%%%%%%%%%%%%%%%%%%%%%%%%%%%%
\subsection{Modelo UEM}

O modelo básico de efeitos não observados (UEM) pode ser escrito para uma amostra \textit{cross-section} aleatória $i$ como:

\vspace{-1 em}
\begin{align}\label{uem:mod:1}
	y_{it} = \xvec_{it} \betavec + c_{i} + u_{it} \, , \quad t=1,\dots, T.
\end{align}
onde $c_{i}$ é o efeito não observado (componente não observado, variável latente, heterogeneidade não observada, efeito individual, heterogeneidade individual).
Estamos supondo $c_{i}$ \textbf{não} observável.

\vspace{1 em}
\noindent
Definindo os erros compostos $v_{it} = c_{i} + u_{it}$, temos:

\vspace{-1 em}
\begin{align}\label{uem:mod:2}
	y_{it} &= \xvec_{it} \betavec + v_{it}
\end{align}
Ou, em forma de vetor:

\vspace{-1 em}
\begin{align}\label{uem:mod:vec}
	\yvec_{i} &= \Xmat_{i} \betavec + \vvec_{i}
\end{align}

%%%%%%%%%%%%%%%%%%%%%%%%%%%%%%%%%%%%%%%%%%%%%%%%%%%%%%
\subsection{Estimação e Consistência}
\noindent
\citet[Sec.10.3 -- Estimating UEM by POLS, p.256]{wool-2010}.

\vspace{1 em}
Se usarmos o estimador POLS na equação \eqref{uem:mod:1}, o estimador será consistente se:

\vspace{-1 em}
\begin{align*}
	\E(\xvec_{it}' v_{it}) = \zerovec \, , \quad t=1, \dots T.
\end{align*}

Ou seja, precisamos que:

\vspace{-1 em}
\begin{align*}
	& 
	\E(\xvec_{it}' u_{it}) = \zerovec \, , \quad t=1, \dots T.
	\\
	&
	\E(\xvec_{it}' c_{i}) = \zerovec \, , \quad t=1, \dots T.
\end{align*}

%%%%%%%%%%%%%%%%%%%%%%%%%%%%%%%%%%%%%%%%%%%%%%%%%%%%%%
\begin{description}[]
	\item [Caso 1:] $\E(\xvec_{it}' c_{i}) = \zerovec$. \\
		\textbf{POLS} é consistente, mas não é eficiente.\\
		\textbf{Efeitos Aleatórios} é consistente e eficiente.\\
		EA é o \textbf{FGLS}  do modelo.

	\item [Caso 2:] $\E(\xvec_{it}' c_{i}) \neq \zerovec$. \\
		Se POLS é \textbf{inconsistente}. \\
		Nesse caso, usaremos o Modelo de \textbf{Efeitos Fixos} ou \textbf{Primeira Diferença}. \\
		EF e PD é o POLS nos modelos transformados.
\end{description}
%%%%%%%%%%%%%%%%%%%%%%%%%%%%%%%%%%%%%%%%%%%%%%%%%%%%%%

%%%%%%%%%%%%%%%%%%%%%%%%%%%%%%%%%%%%%%%%%%%%%%%%%%%%%%
\paragraph{Obs:}
Modelos com variáveis dependentes defasadas em $\xvec_{it}$ \textit{devem} violar a hipótese $\E(\xvec_{it}'u_{it} ) = \zerovec$ uma vez que $y_{i, t-1}$ e $c_{i}$ devem ser correlacionados.
Considerando $y_{i, t-1}$ como regressor:

\vspace{-1 em}
\begin{align*}
	\left.
		\begin{aligned}
			y_{it} &= \alpha y_{i, t-1} + \xvec_{it} \betavec + v_{it}
			\\
			y_{it-1} &= \alpha y_{i, t-2} + \xvec_{i,t-1} \betavec + v_{i,t-1}
		\end{aligned}
	\right\}
	\Cov(y_{i, t-1}, v_{it}) \neq 0.
\end{align*}
Mesmo se $\E(\xvec_{it}'u_{it} ) = \zerovec$  é verdadeiro, os erros compostos serão serialmente correlacionados devido a presença de $c_{i}$ em cada período de tempo.
Portanto, a inferência do POLS requer um estimador robusto de matriz de covariância e estatísticas robustas de teste.
%%%%%%%%%%%%%%%%%%%%%%%%%%%%%%%%%%%%%%%%%%%%%%%%%%%%%%

%%%%%%%%%%%%%%%%%%%%%%%%%%%%%%%%%%%%%%%%%%%%%%%%%%%%%%
\clearpage
\section{Modelo de Efeitos Fixos (EF), (Fixed Effects FE)}
\noindent
\citet[Sec.10.5 -- Fixed Effects Methods, p.265]{wool-2010}

%%%%%%%%%%%%%%%%%%%%%%%%%%%%%%%%%%%%%%%%%%%%%%%%%%%%%%
\subsection{Modelo}

O modelo linear de \textbf{efeitos individuais não observados (UEM)}:

\vspace{-1 em}
\begin{align} \label{fe:mod:1}
	y_{it} &= \xvec_{it} \betavec + c_{i} + u_{it} 
	\, , \qquad	i = 1, \dots, N; \quad t = 1, \dots, T.
\end{align}

\noindent
Estamos supondo $c_{i}$ \textbf{não} observável.
Definindo $v_{it} = c_{i} + u_{it}$.

\vspace{-1 em}
\begin{align}\label{fe:mod:2}
	y_{it} &= \xvec_{it} \betavec + v_{it}.
	\, , \qquad	i = 1, \dots, N; \quad t = 1, \dots, T.
\end{align}

\noindent
No modelo FE permitimos $\Cov(\xvec_{it}, c_{i} ) \neq 0$.

%%%%%%%%%%%%%%%%%%%%%%%%%%%%%%%%%%%%%%%%%%%%%%%%%%%%%%
\subsection{Transformação \textit{Within}}

Tirando a média do modelo ao longo de 
$t = 1, \dots, T$:

\vspace{-1 em}
\begin{align} \label{fe:mod:av} % eq:10.45 p.267
	\overline{y}_{i} = 
	\overline{\xvec}_{i} \betavec + \overline{c}_{i} + \overline{u}_{i} 
	\, , \qquad	i = 1, \dots, N.
\end{align}

\noindent
onde:

\vspace{-1 em}
\begin{align*}
	\overline{y}_{i} = T^{-1} \sum_{t=1}^{T} y_{it};
	\quad
	\overline{\xvec}_{i} = T^{-1} \sum_{t=1}^{T} \xvec_{it};
	\quad
	\overline{c}_{i} = T^{-1} \sum_{t=1}^{T} c_{i} = c_{i};
	\quad
	\overline{u}_{i} = T^{-1} \sum_{t=1}^{T} u_{it}.
\end{align*}

Então, subtraindo \eqref{fe:mod:av} de \eqref{fe:mod:1}:

\vspace{-1 em}
\begin{align*} % \label{fe:mod:ddot}
	y_{it} - \overline{y}_{i} &=
	(\xvec_{it} - \overline{\xvec}_{i}) \betavec +
	\underbracket[1pt]{c_{i} - \overline{c}_{i}}_{=0} +
	u_{it} - \overline{u}_{i} 
	\, , \qquad	i = 1, \dots, N; \quad t = 1, \dots, T.
\end{align*}

\noindent
Finalmente, obtemos:

\vspace{-1 em}
\begin{align} \label{fe:ddot}
	\ddot{y}_{it} &= \ddot{\xvec}_{it} \betavec + \ddot{u}_{it} 
	\, , \qquad	i = 1, \dots, N; \quad t = 1, \dots, T.
\end{align}

\noindent
Onde 
$\ddot{y}_{it} \equiv y_{it} - \overline{y}_{i}$, 
$\ddot{\xvec}_{it} \equiv \xvec_{it} - \overline{\xvec}_{i}$ e
$\ddot{u}_{it} \equiv u_{it} - \overline{u}_{i}$.
E eliminamos variáveis que não variam ao longo do tempo.

%%%%%%%%%%%%%%%%%%%%%%%%%%%%%%%%%%%%%%%%%%%%%%%%%%%%%%
\subsection{Notação Vetorial e Matriz Centralizadora (\textit{Centering Matrix}) $\Mzero$}

Utilizando notação vetorial para o modelo linear de \textbf{efeitos individuais não observados (UEM)}:

\vspace{-1 em}
\begin{align} \label{fe:mod:1:vec}
	\yvec_{i} &= \Xmat_{i} \betavec + c_{i} \onevec + \uvec_{i} 
	\, , \qquad	i = 1, \dots, N.
	\\
	\label{fe:mod:2:vec}
	\yvec_{i} &= \Xmat_{i} \betavec + \vvec_{it}.
	\, , \qquad	i = 1, \dots, N.
\end{align}

\noindent
Agora, definimos a matriz $\Mzero$ (\citet[p. 268]{wool-2010} usa a notação $\Qmat_{T}$ para essa matriz) como:

\vspace{-1 em}
\begin{align*}
	\Mzero &=
	\Imat_{T} - \onevec_{T} (\onevec_{T}' \onevec_{T})^{-1} \onevec_{T}'.
	=
	\Imat_{T} - T^{-1} \onevec_{T} \onevec_{T}'
\end{align*}

\noindent
A matriz $\Mzero$ tem dimensão $T \times T$.
Além disso, ela é idempotente ($\Mzero \Mzero = \Mzero$) e simétrica (\Mzero' = \Mzero).
%%%%%%%%%%%%%%%%%%%%%%%%%%%%%%%%%%%%%%%%%%%%%%%%%%%%%%

\vspace{1 em}
Podemos transformar o modelo \eqref{fe:mod:1:vec} ao premultiplicarmos todo o modelo por $\Mzero$:

\vspace{-1 em}
\begin{align*} 
	\Mzero \yvec_{i} &= ( \Imat_{T} - T^{-1} \onevec_{T} \onevec_{T}')\yvec_{i}	= \yvec_{i} - \onevec_{T} \overline{y}_{i} = \ddot{\yvec}_{i}
	\\
	\Mzero \Xmat_{i} &=
	\Xmat_{i} - T^{-1} \onevec_{T} \onevec_{T}' \overline{\Xmat}_{i} =
	\Xmat_{i} - \onevec_{T} \overline{\xvec}_{i} =
	\ddot{\Xmat}_{i}
	\\
	\Mzero \uvec_{i} &= ( \Imat_{T} - T^{-1} \onevec_{T} \onevec_{T}')\uvec_{i}	= \uvec_{i} - \onevec_{T} \overline{u}_{i} = \ddot{\uvec}_{i}
	\\
	\Mzero ( c_{1} \onevec_{T} ) &= 
	( \Imat_{T} - T^{-1} \onevec_{T} \onevec_{T}' ) c_{i} \onevec_{T} =
	c_{i}( \Imat_{T} - T^{-1} \onevec_{T} \onevec_{T}' ) \onevec_{T} =
	c_{i}( \onevec_{T} - \onevec_{T} ) = \zerovec_{T}
\end{align*}

\noindent
onde $\overline{\xvec}_{i}$ é o vetor $1 \times K$ com a média dos $K$ regressores.

\vspace{-1 em}
\begin{align}
	\notag
	\Mzero \yvec_{i} &= \Mzero \Xmat_{i} \betavec + \Mzero ( c_{1} \onevec_{T} ) + \Mzero \uvec_{i},
	\quad i = 1, \dots, N.
	\\
	\label{fe:mod:ddot:vec}
	\ddot{\yvec}_{i} &= \ddot{\Xmat}_{i} \betavec + \ddot{\uvec_{i}},
	\quad i = 1, \dots, N.
\end{align}

%%%%%%%%%%%%%%%%%%%%%%%%%%%%%%%%%%%%%%%%%%%%%%%%%%%%%%
\paragraph{Exemplo:} \citep[p.266]{wool-2010} Considere o modelo:

\vspace{-1.5 em}
\begin{align*}
	y_{it} =
	\beta_{0} + \beta_{2} d2_{t} + \dots + \beta_{T} dT_{t}
	+
	\zvec_{i} \deltavec + d2_{t} \zvec_{i} \deltavec_2  + \dots + dT_{t} \zvec_{i} \deltavec_{T} + 
	\xvec_{it} \alphavec + v_{it}
\end{align*}

Após a transformação:

\vspace{-1.5 em}
\begin{align*}
	\ddot{y}_{it} =
	\beta_{2} \ddot{d2}_{t} + \dots + \beta_{T} dT_{t}
	+
	\ddot{d2}_{t} \zvec_{i} \deltavec_2  + \dots + dT_{t} \zvec_{i} \deltavec_{T} + 
	\ddot{\xvec}_{it} \alphavec + \ddot{u}_{it}
\end{align*}

Então, não podemos estimar o coeficiente da variável sexo do indivíduo, por exemplo.
Mas podemos estimar se houve mudança desse efeito ao longo do tempo, em relação a categoria de referência.

%%%%%%%%%%%%%%%%%%%%%%%%%%%%%%%%%%%%%%%%%%%%%%%%%%%%%%
\subsection{Hipóteses}

\paragraph{FE.1:} Exogeneidade Estrita:
$\E( u _{it} \, | \, \xvec_{i1}, \dots, \xvec_{iT}, c_{i}) = 0$, para $t=1, \dots, T$ e $i = 1, \dots, N$.

\paragraph{FE.2:} Posto pleno de $\E( \ddot{\Xmat}_{i}' \ddot{\Xmat}_{i} )$ (para inverter a matriz).
$\posto[ \E( \ddot{\Xmat}_{i}'  \ddot{\Xmat}_{i} ) ]  = K$.

\paragraph{FE.3:} Homoscedasticidade:
$\E( \uvec_{i} \uvec_{i}' \,|\, \Xmat_{i}, c_{i}) = \sigma_{u}^{2} \Imat_{T}$.

%%%%%%%%%%%%%%%%%%%%%%%%%%%%%%%%%%%%%%%%%%%%%%%%%%%%%%
\subsection{Estimação POLS}
\noindent
\citet[p.269]{wool-2010} 

\noindent
Aplicando POLS no modelo transformado \eqref{fe:mod:ddot:vec}, temos:

\vspace{-1 em}
\begin{align} \label{fe:betahat}
	\Aboxed{
		\betavechat^{FE} =
		\left( \sum_{i=1}^{N} \ddot{\Xmat}_{i}' \ddot{\Xmat}_{i} \right)^{-1}
		\left( \sum_{i=1}^{N} \ddot{\Xmat}_{i}' \ddot{\yvec}_{i} \right)
		=
		\left( \sum_{i=1}^{N} \sum_{t=1}^{T} \ddot{\xvec}_{it}' \ddot{\xvec}_{it} \right)^{-1}
		\left( \sum_{i=1}^{N} \sum_{t=1}^{T} \ddot{\xvec}_{it}' \ddot{y}_{it} \right)
	}
\end{align}

\noindent
Este estimador também é chamado de \textbf{estimador within}.

%%%%%%%%%%%%%%%%%%%%%%%%%%%%%%%%%%%%%%%%%%%%%%%%%%%%%%
\subsection{Consistência}
\noindent
\citet[sec.10.5.1 -- Consistency of the Fixed Effects Estimator, p.265--269]{wool-2010} \\
\citet[p.269]{wool-2010} 

%%%%%%%%%%%%%%%%%%%%%%%%%%%%%%%%%%%%%%%%%%%%%%%%%%%%%%
\vspace{1 em}
\noindent
Usando a equação \eqref{fe:mod:ddot:vec} em \eqref{fe:betahat}, temos:

\vspace{-1 em}
\begin{align} \label{fe:betahat:2}
	\betavechat^{FE} =
	\betavec +
	\left[ N^{-1} \sum_{i=1}^{N} \ddot{\Xmat}_{i}' \ddot{\Xmat}_{i} \right]^{-1}
	\left[ N^{-1} \sum_{i=1}^{N} \ddot{\Xmat}_{i}' \ddot{\uvec}_{i} \right]
\end{align}

\noindent
Nota que para $\E(\ddot{\Xmat}_{i}' \ddot{\uvec}_{i}) = \zerovec$ é necessário não haver correlação entre todos os erros 
$u_{it}$ $t=1, \dots, T$
e todos os regressores
$\xvec_{it}'$ $t=1, \dots, T$.
\textbf{FE.1} implica a condição acima.
Além disso, sob \textbf{FE.1}, $\betavechat^{FE}$ é não viciado.

%%%%%%%%%%%%%%%%%%%%%%%%%%%%%%%%%%%%%%%%%%%%%%%%%%%%%%
\subsection*{Formato Totalmente Matricial}

Empilhando os vetores $N$ vezes, vamos definir:

\vspace{-1 em}
\begin{align*}
	\underset{NT \times K}{\ddot{\Xmat}} = 
	\begin{bmatrix}
		\ddot{\Xmat}_{1} \\ \vdots \\ \ddot{\Xmat}_{N}
	\end{bmatrix}
	=
	\begin{bmatrix}
		\ddot{x}_{1,11} & \dots & \ddot{x}_{K, 11} \\
		\vdots          &       &  \vdots          \\
		\ddot{x}_{1,1T} & \dots & \ddot{x}_{K, 1T} \\
		\ddot{x}_{1,21} & \dots & \ddot{x}_{K, 21} \\
		\vdots          &       &  \vdots          \\
		\ddot{x}_{1,2T} & \dots & \ddot{x}_{K, 2T} \\
		\vdots          &       &  \vdots          \\
		\ddot{x}_{1,N1} & \dots & \ddot{x}_{K, N1} \\
		\vdots          &       &  \vdots          \\
		\ddot{x}_{1,NT} & \dots & \ddot{x}_{K, NT}
	\end{bmatrix}
	\quad
% Y vecs
	\underset{NT \times 1}{\ddot{\yvec}} = 
	\begin{bmatrix}
		\ddot{\yvec}_{1} \\ \vdots \\ \ddot{\yvec}_{N}
	\end{bmatrix}
	=
	\begin{bmatrix}
		\ddot{y}_{11} \\ \vdots \\ \ddot{y}_{1T} \\
		\ddot{y}_{21} \\ \vdots \\ \ddot{y}_{2T} \\
		\vdots \\
		\ddot{y}_{N1} \\ \vdots \\ \ddot{y}_{NT} \\
	\end{bmatrix}
	\quad
% u vecs
	\underset{NT \times 1}{\ddot{\uvec}} = 
	\begin{bmatrix}
		\ddot{\uvec}_{1} \\ \vdots \\ \ddot{\uvec}_{N}
	\end{bmatrix}
	=
	\begin{bmatrix}
		\ddot{u}_{11} \\ \vdots \\ \ddot{u}_{1T} \\
		\ddot{u}_{21} \\ \vdots \\ \ddot{u}_{2T} \\
		\vdots \\
		\ddot{u}_{N1} \\ \vdots \\ \ddot{u}_{NT} \\
	\end{bmatrix}
\end{align*}


%%%%%%%%%%%%%%%%%%%%%%%%%%%%%%%%%%%%%%%%%%%%%%%%%%%%%%
\vspace{1 em}
\noindent
A matriz $\Xmat$ é $NT \times K$, a matriz $\Mzero$ é $T \times T$ e a matriz $\Imat_{N}$ é $N \times N$.
A produto \textbf{Kronecker} de $\Imat_{N}$ por $\Mzero$,

\vspace{-1 em}
\begin{align*}
	\underset{N \times N}{\Imat_{N}},
	\otimes
	\underset{T \times T}{\Mzero}
\end{align*}

\noindent
é uma matriz $NT \times NT$.
Dessa forma, podemos definir:

\vspace{-1 em}
\begin{align*}
	\ddot{\yvec} = ( \Imat_{N} \otimes \Mzero ) \yvec,
	\\
	\ddot{\Xmat} = ( \Imat_{N} \otimes \Mzero ) \Xmat
	\\
	\ddot{\uvec} = ( \Imat_{N} \otimes \Mzero ) \uvec,
\end{align*}

%%%%%%%%%%%%%%%%%%%%%%%%%%%%%%%%%%%%%%%%%%%%%%%%%%%%%%
\vspace{1 em}
\noindent
E com isso, reescrevermos \eqref{fe:betahat:2} como:

\vspace{-1 em}
\begin{align} \label{fe:betahat:3}
	\betavechat^{FE} =
	\betavec +
	\left( \ddot{\Xmat}' \ddot{\Xmat} \right)^{-1}
	\ddot{\Xmat}' \ddot{\uvec}
\end{align}

\noindent
Ou ainda, como:

\vspace{-1 em}
\begin{align}
	\notag
	\betavechat^{FE} &=
	\betavec +
	\left[ \Xmat' ( \Imat_{N} \otimes \Mzero )( \Imat_{N} \otimes \Mzero ) \Xmat \right]^{-1}
	\left[ \Xmat' ( \Imat_{N} \otimes \Mzero )( \Imat_{N} \otimes \Mzero ) \uvec \right]
	\\
	\label{fe:betahat:4}
	\betavechat^{FE} &=
	\betavec +
	\left[ \Xmat' ( \Imat_{N} \otimes \Mzero ) \Xmat \right]^{-1}
	\left[ \Xmat' ( \Imat_{N} \otimes \Mzero ) \uvec \right]
\end{align}

\noindent
onde usamos as propriedades de simetria e idempotência da matriz $\Mzero$.

%%%%%%%%%%%%%%%%%%%%%%%%%%%%%%%%%%%%%%%%%%%%%%%%%%%%%%
\subsection{Matriz de Covariância Robusta}
\noindent
\citet[sec.10.5.2 -- Asymptotic Inference with Fixed Effects, p.269--272]{wool-2010} \\

%%%%%%%%%%%%%%%%%%%%%%%%%%%%%%%%%%%%%%%%%%%%%%%%%%%%%%
A matriz de covariância assintótica fica:

\vspace{-1 em}
\begin{align*}
	\E\left( \ddot{\Xmat}' \ddot{\Xmat} \right)^{-1}
	\E\left( \ddot{\Xmat}' \ddot{\uvec} \ddot{\uvec}' \ddot{\Xmat} \right)
	\E\left( \ddot{\Xmat}' \ddot{\Xmat} \right)^{-1}
\end{align*}

A qual pode ser estimada por

\vspace{-1 em}
\begin{align*}
	\left( \ddot{\Xmat}' \ddot{\Xmat} \right)^{-1}
	\left( \sum_{i=1}^{N} \ddot{\Xmat}'
		\widehat{\ddot{\uvec}} \widehat{\ddot{\uvec}}'
	\ddot{\Xmat} \right)
	\left( \ddot{\Xmat}' \ddot{\Xmat} \right)^{-1}
\end{align*}

%%%%%%%%%%%%%%%%%%%%%%%%%%%%%%%%%%%%%%%%%%%%%%%%%%%%%%
\noindent
Analisando
$\E\left( \ddot{\Xmat}_{i}' \ddot{\uvec}_{i} \ddot{\uvec}_{i}' \ddot{\Xmat}_{i} \right)$:

\vspace{-1 em}
\begin{align*}
	\E\left( \ddot{\Xmat}_{i}' \ddot{\uvec}_{i} \ddot{\uvec}_{i}' \ddot{\Xmat}_{i} \right)
	&=
	\E\left[
		(\Xmat_{i}' \Mzero')  (\Mzero \uvec_{i}) (\uvec_{i}' \Mzero')  (\Mzero \Xmat_{i})
	\right]
	\\ &=
	\E\left[
		(\Xmat_{i}' \Mzero') \uvec_{i} \uvec_{i}' (\Mzero \Xmat_{i})
	\right]
	\\ &=
	\E\left[
		\ddot{\Xmat}_{i}' \uvec_{i} \uvec_{i}'  \ddot{\Xmat}_{i}
	\right]
\end{align*}

\noindent
onde usamos as propriedades de simetria e idempotência da matriz $\Mzero$ e as definições de $\ddot{\Xmat}_{i}$ e $\ddot{\uvec}_{i}$.

%%%%%%%%%%%%%%%%%%%%%%%%%%%%%%%%%%%%%%%%%%%%%%%%%%%%%%
\noindent
Sob \textbf{FE.3}, 
$\E(\uvec_{i} \uvec_{i}' | \ddot{\Xmat}_{i}) = \sigma^2_{u} \Imat_{T}$,
temos:

\vspace{-1 em}
\begin{align*}
	\E\left( \ddot{\Xmat}_{i}' \uvec_{i} \uvec_{i}' \ddot{\Xmat}_{i} \right)
	=
	\E\left[
		\E\left( 
			\ddot{\Xmat}_{i}' \uvec_{i} \uvec_{i}'  \ddot{\Xmat}_{i}
		| \Xmat_{i}, c_{i} \right)
	\right]
	=
	\E\left[
		\ddot{\Xmat}_{i}'
		\E\left( 
			\uvec_{i} \uvec_{i}' 
		| \Xmat_{i}, c_{i} \right)
		\ddot{\Xmat}_{i}
	\right]
	=
	\sigma^{2}_{u} \Imat_{T} \E(\ddot{\Xmat}_{i}' \ddot{\Xmat}_{i})
\end{align*}

%%%%%%%%%%%%%%%%%%%%%%%%%%%%%%%%%%%%%%%%%%%%%%%%%%%%%%
Assim, a matriz de covariância fica:

\vspace{-1 em}
\begin{align*}
	\E\left( \ddot{\Xmat}' \ddot{\Xmat} \right)^{-1}
	\sigma^{2}_{u} \Imat_{T} \E(\ddot{\Xmat}_{i}' \ddot{\Xmat}_{i})
	\E\left( \ddot{\Xmat}' \ddot{\Xmat} \right)^{-1}
	=
	\boxed{
		\sigma^{2}_{u} 
	\E\left( \ddot{\Xmat}' \ddot{\Xmat} \right)^{-1} }
\end{align*}

%%%%%%%%%%%%%%%%%%%%%%%%%%%%%%%%%%%%%%%%%%%%%%%%%%%%%%
\subsubsection*{Estimando Elementos da Matriz de Covariância}
\noindent
Queremos estimar $\sigma^2_{u}$ por valores amostrais:

\vspace{-1 em} 
\begin{align*}
	\E( u_{it}^2) &= \sigma^2_{u}
	\quad \text{ e } \quad
	\E( \ddot{u}_{it}^2) = \sigma^{2}_{\ddot{u}}
\end{align*}

\vspace{-1 em} 
\begin{align*}
	\sigma^{2}_{\ddot{u}} &=
	\E[ ( u_{it} - \overline{u}_{i})^2] = 
	\E( u_{it}^2) + \E( \overline{u}_{i}^2) - 2 \E( u_{it} \overline{u}_{i})
\end{align*}

\noindent
utilizando $\overline{u}_{i} = T^{-1} \sum_{t=1}^{T} u_{it}$:

\vspace{-1 em} 
\begin{align*}
	\E( \ddot{u}_{it}^2) 
	&=
	\E( u_{it}^2) + T^{-1} \sum_{t=1}^{T} \E( u_{it}^2) - 2 \E( u_{it} T^{-1} \sum_{t=1}^{T} u_{it})
	= 
	\E( u_{it}^2) + T^{-1} \sum_{t=1}^{T} \E( u_{it}^2) - 2T^{-1} \sum_{t=1}^{T} \E( u^2_{it})
	\\
	&= 
	\sigma^2_{u} + \sigma^2_{u}/T - 2\sigma^2_{u}/T = \sigma^2_{u}(1 - 1/T)
	\implies
	\boxed{\sigma^2_{u} = \frac{T}{T-1} \sigma^2_{\ddot{u}} }
\end{align*}

\noindent
Utilizando estimadores amostrais para $\E(\ddot{u}_{it}^{2})$:

\vspace{-1 em} 
\begin{align*}
	\sigmahat_{u}^2
	&= 
	\frac{T}{T-1} \frac{1}{NT}
	\sum_{i=1}^{N} \sum_{t=1}^{T}  
	\widehat{\ddot{u}}_{it}^2
\end{align*}

\noindent
Ajustando os Graus de Liberdade (Cortando $T$s e subtraindo $K$ do número de regressores):

\vspace{-1 em} 
\begin{align}\label{fe:sighat}
	\sigmahat^2_{u} = \frac{1}{N(T-1) - K}
	\sum_{i=1}^{N} \sum_{t=1}^{T} \widehat{\ddot{u}}^2_{it}.
\end{align}

%%%%%%%%%%%%%%%%%%%%%%%%%%%%%%%%%%%%%%%%%%%%%%%%%%%%%%
\subsection*{Teste para Autocorrelação AR(1)}
\noindent
\citet[p.275]{wool-2010}

\vspace{-1 em} 
\begin{align*}
	\E( \ddot{u}_{it} \ddot{u}_{i,t-1} ) &=
	\E\left[ ( u_{i,t} - \overline{u}_{i} ) ( u_{i,t-1} - \overline{u}_{i} ) \right]
	\\
	&=
	\E( u_{i,t} u_{i,t-1} ) - 
	\E( u_{i,t} \overline{u}_{i} ) -
	\E( \overline{u}_{i} u_{i,t-1} ) +
	\E( \overline{u}^2_{i} )
	\\
	&=
	0 - T^{-1} \sigma^2_{u} - T^{-1} \sigma^2_{u} + T^{-1} \sigma^2_{u}
	\\
	\E( \ddot{u}_{it} \ddot{u}_{i,t-1} ) 
	&= -T^{-1} \sigma_{u}^2
\end{align*}

\vspace{-1 em} 
\begin{align*}
	\Corr( \ddot{u}_{it}, \ddot{u}_{i,t-1} ) &=
	\frac{\E( \ddot{u}_{it} \ddot{u}_{i,t-1} )}{\E( \ddot{u}_{it}^2 )} =
	\frac{-T^{-1} \sigma_{u}^2}{\frac{T-1}{T}\sigma^2_{u}} = 
	\frac{-1}{T-1}
\end{align*}

Vamos testar 

\vspace{-1 em} 
\begin{align*}
	H_{0}: \delta = \frac{-1}{T-1}
\end{align*}
(ausência de correlação em \red{$\uvec$}) na equação:

\vspace{-1 em} 
\begin{align*}
	\widehat{\ddot{u}}_{it} = \delta \widehat{\ddot{u}}_{it-1} + e_{it}
\end{align*}
para $t=2, \dots, T$.
Fazer teste $t$ robusto.

%%%%%%%%%%%%%%%%%%%%%%%%%%%%%%%%%%%%%%%%%%%%%%%%%%%%%%
\clearpage
\section{First Difference (FD, PD)}
\noindent
\citet[Sec.10.6 -- First Difference Methods, p.279]{wool-2010} \\

%%%%%%%%%%%%%%%%%%%%%%%%%%%%%%%%%%%%%%%%%%%%%%%%%%%%%%
\subsection{Modelo}

O modelo linear de \textbf{efeitos não observados}:

\vspace{-1 em}
\begin{align} \label{mod1:FD}
	y_{it} = \xvec_{it} \betavec + c_{i} + u_{it},
\end{align}

\noindent
para
$t = 1, \dots, T$ e $i = 1, \dots, N$.

O modelo contém explicitamente um componente não observado, $c_{i}$, que não varia no tempo.
Tratamos o componente não observado como parte do erro, não como parâmetro a ser estimado.
Aqui permitimos que $c_{i}$ seja correlacionado com $\xvec_{it}$.
Deste modo, \textbf{não} podemos ignorar a sua presença e estimar \eqref{mod1:FD} por POLS, visto que isso resultaria num estimador inconsistente devido a \textbf{endogeneidade}.

Assim, transformamos o modelo para eliminar $c_{i}$ e conseguirmos fazer uma estimação consistente de $\betavec$.
A trasnformação a ser feita é a primeira diferença.
Para tanto, seguimos os seguintes passos:

\begin{itemize}\itemsep0pt
	\item Reescrevemos \eqref{mod1:FD} defasado:

		\vspace{-1 em}
		\begin{align}  \label{mod2:FD}
			y_{it-1} = \xvec_{it-1} \betavec + c_{i} + u_{it-1}
		\end{align}

	\item Tiramos a diferença entre \eqref{mod2:FD} e \eqref{mod1:FD}:

		\vspace{-1 em}
		\begin{align}
			\nonumber
			y_{it} - y_{it-1} &=
			(\xvec_{it} - \xvec_{it-1}) \betavec +
			c_{i} - c_{i} +
			u_{it} - u_{it-1}
			\\
			\label{mod3:FD}
			\Delta y_{it} &=
			\Delta \xvec_{it} \betavec +
			\Delta u_{it}. 
		\end{align}

		\noindent
		para
		$t = 2, \dots, T$ e $i = 1, \dots, N$.
\end{itemize}

Reescrevendo \eqref{mod3:FD} no formato matricial empilhando $T$:

\vspace{-1 em}
\begin{align} \label{mod4:FD}
	\Delta \yvec_{i} = \Delta \Xmat_{i} \betavec + \mbs{e}_{i}
\end{align}

\noindent
com 
$\boxed{e_{it} = \Delta u_{it}}$.

\begin{itemize}\itemsep0pt
	\item
		$\Delta \yvec_{i}$ vetor $( T - 1 ) \times 1$ 
	\item
		$\Delta \Xmat_{i}$  matriz  $( T - 1 ) \times K$
	\item
		$\betavec$ vetor $K \times 1$
	\item
		$\mbs{e}_{i}$ vetor $(T - 1 ) \times 1$
\end{itemize}


%%%%%%%%%%%%%%%%%%%%%%%%%%%%%%%%%%%%%%%%%%%%%%%%%%%%%%
\subsection{Estimação POLS}

O estimador $\widehat{\betavec}^{FD}$ é o POLS da regressão no modelo \eqref{mod4:FD}, assim:

\vspace{-1 em}
\begin{align} \label{beta:pols:FD}
	\Aboxed{
		\betavec^{FD} =
		\left[ \sum_{i=1}^{N} \Delta \Xmat_{i}' \Delta \Xmat_{i} \right]^{-1}
		\left[ \sum_{i=1}^{N} \Delta \Xmat_{i}' \Delta \yvec_{i} \right]
	}
\end{align}

%%%%%%%%%%%%%%%%%%%%%%%%%%%%%%%%%%%%%%%%%%%%%%%%%%%%%%
\subsection{Hipóteses}

As Hipóteses que usamos para $\widehat{\betavec}^{FD}$ são:

\begin{description}
		\setlength\itemsep{.5 em}

	\item [FD.1:] Exogeneidade Estrita:
		$\E( u _{it} \, | \, \xvec_{i1}, \dots, \xvec_{iT}, c_{i}) = 0$, para $t=1, \dots, T$ e $i = 1, \dots, N$.

	\item  [FD.2:] Posto completo de $\E( \Delta \Xmat_{i}' \Delta \Xmat_{i} )$ (para inverter a matriz).
		$posto[ \E( \Delta \Xmat_{i} ' \Delta \Xmat_{i} ) ]  = K$.

	\item [FD.3:] Homoscedasticidade:
		$\E(\mbs{e}_{i} \mbs{e}_{i}' \,|\, \Xmat_{i}, c_{i}) = \sigma_{e}^{2} \Imat_{T-1}$.
\end{enumerate}

%%%%%%%%%%%%%%%%%%%%%%%%%%%%%%%%%%%%%%%%%%%%%%%%%%%%%%
\subsection{Valor Esperado}
Usando apenas \textbf{FD.1} e \textbf{FD.2}:

\vspace{-1 em}
\begin{align*}
	\E( \betavec^{FD} ) &=
	\betavec +
	\E\left[
		\left( \sum_{i=1}^{N} \Delta \Xmat_{i}' \Delta \Xmat_{i} \right)^{-1}
		\left( \sum_{i=1}^{N} \Delta \Xmat_{i}' \mbs{e}_{i} \right)
	\right]
	\\
	\Aboxed{
		\E( \betavec^{FD} ) &=
		\betavec +
		\E \left[ ( \Delta \Xmat' \Delta \Xmat )^{-1} ( \Delta \Xmat' \mbs{e} \right]
		}
	\end{align*}

%%%%%%%%%%%%%%%%%%%%%%%%%%%%%%%%%%%%%%%%%%%%%%%%%%%%%%
	\subsection{Variância}

	Usando apenas \textbf{FD.1} e \textbf{FD.2}:

	\vspace{-1 em}
	\begin{align*} 
		\Aboxed{
			\Var( \betavec^{FD} ) = 
			\E \left[
				( \Delta \Xmat' \Delta \Xmat )^{-1}
				( \Delta \Xmat' \mbs{e}  \mbs{e}' \Delta \Xmat )
				( \Delta \Xmat' \Delta \Xmat )^{-1} 
		\right]}
	\end{align*}

%%%%%%%%%%%%%%%%%%%%%%%%%%%%%%%%%%%%%%%%%%%%%%%%%%%%%%
	\subsection{Variância sob Homocedasticidade}

	Usando \textbf{FD.3}, temos

	\vspace{-1 em}
	\begin{align*} 
		\Var( \betavec^{FD} ) &= 
		\sigma_{e}^{2}
		\E \left[
			( \Delta \Xmat' \Delta \Xmat )^{-1}
			( \Delta \Xmat' \Delta \Xmat )
			( \Delta \Xmat' \Delta \Xmat )^{-1} 
		\right]
		\\
		\Aboxed{
			\Var( \betavec^{FD} ) &= 
			\sigma^2_{e}
			\E \left[
				( \Delta \Xmat' \Delta \Xmat )^{-1} 
		\right]}
	\end{align*}

	\noindent 
	com

	\vspace{-1 em}
	\begin{align*} 
		\sigma^2_{e} = 
		\left[ N ( T - 1 ) - K \right]^{-1}
		\left[  
			\sum_{i=1}^{N} 
			\sum_{t=1}^{T}
			\hat{e}_{it}^{2}
		\right],
	\end{align*}

	\noindent
	que é a média de todos $\hat{e}^{2}_{it}$ contando $K$ regressores.

%%%%%%%%%%%%%%%%%%%%%%%%%%%%%%%%%%%%%%%%%%%%%%%%%%%%%%
	\clearpage
	\section{System GLS (SGLS)}

	\noindent
	\citet[Sec.7.4 -- Consistency and Asymptotic Normality of Generalized Least Squares, p.153]{wool-2010} 

%%%%%%%%%%%%%%%%%%%%%%%%%%%%%%%%%%%%%%%%%%%%%%%%%%%%%%
	\subsection{Modelo Linear}

%%%%%%%%%%%%%%%%%%%%%%%%%%%%%%%%%%%%%%%%%%%%%%%%%%%%%%
	\subsection{Hipóteses}

	Para implementarmos o estimador de \textbf{GLS} precisamos das seguintes hipótese:

	\begin{enumerate}
		\item %1

			$\E(\Xmat_{i} \otimes \uvec_{i}) = 0$.

			Para SGLS ser consistente, precisamos que $\uvec_{i}$ não seja correlacionada com nenhum elemento de $\Xmat_{i}$.

		\item %2

			$\Omega$ é positiva definida (para ter inversa).
			$\E(\Xmat_{i}^{\prime} \Omega^{-1} \Xmat_{i})$ é \textbf{não} singular (para ter invesa).

			Onde, $\Omega$ é a seguinte matriz \textbf{simétrica}, positiva-definida:

			\vspace{-1.5 em}
			\begin{align*}
				\Omega = \E(\uvec_{i} \uvec_{i}^{\prime}).
			\end{align*}
	\end{enumerate}

%%%%%%%%%%%%%%%%%%%%%%%%%%%%%%%%%%%%%%%%%%%%%%%%%%%%%%
	\subsection{Estimação}

	Agora, transformamos o sistema de equações ao realizarmos a pré-multiplicação do sistema por $\Omega^{-1/2}$:

	\vspace{-1.5 em}
	\begin{align*}
		\Omega^{-1/2} \yvec_{i} 
		&=
		\Omega^{-1/2} \Xmat_{i} \betavec
		+
		\Omega^{-1/2} \uvec_{i}
		\\
		\yvec_{i}^{*}
		&=
		\Xmat_{i}^{*} \betavec
		+
		\uvec^{*}_{i}
	\end{align*}

	Estimando a equação acima por \textbf{SOLS}:

	\vspace{-1.5 em}
	\begin{align*}
		\beta^{SOLS}
		&=
		\left( \sum_{i=1} \Xmat_{i}^{*'} \Xmat_{i}^{*} \right)^{-1}
		\left( \sum_{i=1} \Xmat_{i}^{*'} \yvec_{i}^{*} \right)
		\\
		&=
		\left( \sum_{i=1} \Xmat_{i}^{'} \Omega^{-1/2} \Omega^{-1/2} \Xmat_{i} \right)^{-1}
		\left( \sum_{i=1} \Xmat_{i}^{'} \Omega^{-1/2} \Omega^{-1/2} \yvec_{i} \right)
		\\
		&=
		\left( \sum_{i=1} \Xmat_{i}^{'} \Omega^{-1} \Xmat_{i} \right)^{-1}
		\left( \sum_{i=1} \Xmat_{i}^{'} \Omega^{-1} \yvec_{i} \right)
	\end{align*}


%%%%%%%%%%%%%%%%%%%%%%%%%%%%%%%%%%%%%%%%%%%%%%%%%%%%%%
	\clearpage
	\section{GLS Factível}

	\noindent
	\citet[Sec.7.5 -- Feasible GLS, p.153]{wool-2010} 

%%%%%%%%%%%%%%%%%%%%%%%%%%%%%%%%%%%%%%%%%%%%%%%%%%%%%%
	\subsection{FSGLS: SGLS Factível}

	Para obtermos $\beta^{SGLS}$ precisamos conhecer $\Omega$, o que não ocorre na prática.
	Então, precisamos estimar $\Omega$ com um estimador consistente.
	Para tanto usamos um procedimento de dois passos:

	\begin{enumerate}
		\item  % Passo 1
			Estimar $\yvec_{i} = \Xmat_{i} \betavec + \uvec_{i}$ via \textbf{SOLS} e guardar o resíduo estimado $\widehat{\uvec_{i}}$.

		\item  %Passo 2
			Estimar $\Omega$ com o seguinte estimador $\widehat{\Omega}$:

			\vspace{-1.5 em}
			\begin{align*}
				\widehat{\Omega} 
				= 
				N^{-1} \sum_{i=1}^{N} \uvec_{i} \uvec_{i}'
			\end{align*}
	\end{enumerate}

	Com a estimativa $\widehat{\Omega}$ feita, podemos obter $\beta^{FSGLS}$ pela fórmula do $\beta^{SGLS}$:

	\vspace{-1.5 em}
	\begin{align*}
		\beta^{FGLS}
		= 
		\left[ 
			\sum_{i} \Xmat_{i}' \widehat{\Omega}^{-1} \Xmat_{i}
		\right]^{-1}
		\left[ 
			\sum_{i} \Xmat_{i}' \widehat{\Omega}^{-1} \yvec_{i}
		\right]
	\end{align*}

	Empilhando as $N$ observações:

	\vspace{-1.5 em}
	\begin{align*}
		\beta^{FGLS}
		= 
		\left[ \Xmat \left( \Imat_{N} \otimes \widehat{\Omega}^{-1} \right) \Xmat \right]^{-1}
		\left[ \Xmat' \left( \Imat_{N} \otimes \widehat{\Omega}^{-1} \right) \yvec \right]
	\end{align*}

	Reescrevendo a equação acima:

	\vspace{-1.5 em}
	\begin{align*}
		\beta^{FGLS}
		&= 
		\left[  \Xmat \left( \Imat_{N} \otimes \widehat{\Omega}^{-1} \right) \Xmat \right]^{-1}
		\left[  \Xmat' \left( \Imat_{N} \otimes \widehat{\Omega}^{-1} \right) (\Xmat \beta + u) \right]
		\\
		&= 
		\left[ \Xmat \left( \Imat_{N} \otimes \widehat{\Omega}^{-1} \right) \Xmat \right]^{-1}
		\left\{ 
			\left[ \Xmat' \left( \Imat_{N} \otimes \widehat{\Omega}^{-1} \right) \Xmat \beta \right]
% \\
% &
			\; +
			\left[ \Xmat' \left( \Imat_{N} \otimes \widehat{\Omega}^{-1} \right) u \right]
		\right\}
		\\
		&= 
		\beta +
		\left[ \Xmat \left( \Imat_{N} \otimes \widehat{\Omega}^{-1} \right) \Xmat \right]^{-1}
		\left[ \Xmat' \left( \Imat_{N} \otimes \widehat{\Omega}^{-1} \right) u \right]
	\end{align*}

%%%%%%%%%%%%%%%%%%%%%%%%%%%%%%%%%%%%%%%%%%%%%%%%%%%%%%
	\subsection{Valor Esperado}

	\vspace{-1 em}
	\begin{align*}
		\E(\beta^{FGLS})
		= 
		\beta +
		\left[ \Xmat \left( \Imat_{N} \otimes \widehat{\Omega}^{-1} \right) \Xmat \right]^{-1}
		\left[ \Xmat' \left( \Imat_{N} \otimes \widehat{\Omega}^{-1} \right) u \right]
	\end{align*}

	Concluímos que, se 
	$\widehat{\Omega} \xrightarrow{\enskip p \enskip} \Omega$,
	então,
	$\beta^{FSGLS} \xrightarrow{\enskip p \enskip} \beta$,

%%%%%%%%%%%%%%%%%%%%%%%%%%%%%%%%%%%%%%%%%%%%%%%%%%%%%%
	\subsection{Variância}

	\vspace{-1 em}
	\begin{align*}
		\Var(\beta^{FGLS})
		&= 
		\left[ \Xmat \left( \Imat_{N} \otimes \widehat{\Omega}^{-1} \right) \Xmat \right]^{-1}
		\left[ \Xmat' \left( \Imat_{N} \otimes \widehat{\Omega}^{-1} \right) u \right]
		\left\{ 
			\left[ \Xmat \left( \Imat_{N} \otimes \widehat{\Omega}^{-1} \right) \Xmat \right]^{-1}
			\left[ \Xmat' \left( \Imat_{N} \otimes \widehat{\Omega}^{-1} \right) u \right]
		\right\}^{\prime}
		\\
		&=
		\left[ \Xmat \left( \Imat_{N} \otimes \widehat{\Omega}^{-1} \right) \Xmat \right]^{-1}
		\left[
			\Xmat' \left( \Imat_{N} \otimes \widehat{\Omega}^{-1} \right) 
			u u'
			\left( \Imat_{N} \otimes \widehat{\Omega}^{-1} \right) \Xmat
		\right]
		\left[ \Xmat \left( \Imat_{N} \otimes \widehat{\Omega}^{-1} \right) \Xmat \right]^{-1}
	\end{align*}

	Tirando o valor Esperado e supondo que:

	\vspace{-1.5 em}
	\begin{align*}
		\E(\Xmat_{i} \Omega^{-1} u_{i} u_{i}' \Xmat_{i}) = \E(\Xmat_{i} \Omega^{-1})
	\end{align*}
	temos:

	\vspace{-1.5 em}
	\begin{align*}
		\E\left[ \Xmat' \left( \Imat_{N} \otimes \widehat{\Omega}^{-1} \right)
			u u'
		\left( \Imat_{N} \otimes \widehat{\Omega}^{-1} \right)' \Xmat \right]
		=
		\E(\Xmat' \Omega^{-1} \Xmat)
	\end{align*}
	e temos:

	\vspace{-1.5 em}
	\begin{align*}
		\Var(\beta^{FSGLS}) = \left[ \E(\Xmat' \Omega^{-1} \Xmat \right]^{-1}.
		\end{align*}

%%%%%%%%%%%%%%%%%%%%%%%%%%%%%%%%%%%%%%%%%%%%%%%%%%%%%%
		\clearpage
		\section{Random Effects (RE, EA)}
		\noindent
		\citet[Sec.10.4 -- Random Effects Methods, p.257]{wool-2010} \\

%%%%%%%%%%%%%%%%%%%%%%%%%%%%%%%%%%%%%%%%%%%%%%%%%%%%%%
		\subsection{Modelo}

		O modelo linear de \textbf{efeitos não observados}:

		\vspace{-1 em}
		\begin{align} \label{mod1:EA}
			y_{it} = \xvec_{it} \betavec + c_{i} + u_{it},
		\end{align}

		\noindent
		onde
		$t = 1, \dots, T$ e $i = 1, \dots, N$.

		O modelo contém explicitamente um componente não observado que não varia no tempo $c_{i}$.
		Abordamos esse componente como parte do erro, não como parâmetro a ser estimado.
		Para a análise de \textbf{Efeitos Aleatórios, (EA) ou (RE)}, supomos que os regressões $\xvec_{it}$ são \textbf{não correlacionados} com $c_{i}$, mas fazemos hipóteses mais restritas que o \textbf{POLS}; pois assim exploramos a presença de \textbf{correlação serial} do erro composto por GLS e garantimos a consitência do estimador de FGLS.

		Podemos reescrever \eqref{mod1:EA} como:

		\vspace{-1 em}
		\begin{align} \label{mod2:EA}
			y_{it} = \xvec_{it} \betavec + v_{it},
		\end{align}

		\noindent
		onde
		$t = 1, \dots, T$, $i = 1, \dots, N$ e $\boxed{v_{it} = c_{i} + u_{it}}$ é o erro composto.

		Agora, vamos empilhar os $t$'s e reescrever \eqref{mod2:EA} como:

		\vspace{-1 em}
		\begin{align} \label{mod3:EA}
			\yvec_{i} = \Xmat_{i} \betavec + \mbs{v}_{i},
		\end{align}

		\noindent
		onde
		$i = 1, \dots, N$ e $\boxed{\mbs{v}_{i} = c_{i} \onevec_{T} + \uvec_{i}}$.

%%%%%%%%%%%%%%%%%%%%%%%%%%%%%%%%%%%%%%%%%%%%%%%%%%%%%%
		\subsection{Hipóteses}

		As Hipóteses que usamos para $\mbs{\widehat{\beta}}^{RE}$ são:

		\begin{enumerate} \itemsep0pt
			\item  
				Usamos o modelo correto e $c_{i}$ não é endógeno.

				\begin{enumerate}[label =\alph*)]
				\item 
					$\E( u_{it} \, | \,  x_{i1}, \dots, x_{iT}, c_{i} ) = 0$,
					$i = 1, \dots, N$.
				\item        
					$\E( c_{it} \, | \, x_{i1}, \dots, x_{iT} ) = \E( c_{i} ) = 0$,
					$i = 1, \dots, N$.
			\end{enumerate}

		\item  Posto completo de $\E( \Xmat_{i}' \Omega^{-1} \Xmat_{i} )$.

			Definindo a matriz $T \times T$, $\boxed{\Omega \equiv \E(\mbs{v}_{i} \mbs{v}_{i}')}$, queremos que $\E( \Xmat_{i} \Omega^{-1} \Xmat_{i} )$ tenha posto completo (posto = $K$).
	\end{enumerate}

	A matriz $\Omega$ é simétrica $\Omega' = \Omega$ e positiva definida $\det(\Omega) > 0$.
	Assim podemos achar $\Omega^{1/2}$ e $\Omega^{-1/2}$ com $\Omega = \Omega^{1/2} \Omega^{1/2}$ e $\Omega^{-1} = \Omega^{-1/2} \Omega^{-1/2}$.

%%%%%%%%%%%%%%%%%%%%%%%%%%%%%%%%%%%%%%%%%%%%%%%%%%%%%%
	\subsection{Estimação}

	Premultiplicando \eqref{mod3:EA} port $\Omega^{-1/2}$ do dois lados, temos:

	\vspace{-1 em}
	\begin{align} 
		\notag
		\Omega^{-1/2}\yvec_{i} &= \Omega^{-1/2}\Xmat_{i} \betavec + \Omega^{-1/2}\mbs{v}_{i}
		\\
		\label{mod4:EA}
		\yvec_{i}^{*} &= \Xmat_{i}^{*} \betavec + \mbs{v}_{i}^{*},
	\end{align}

	Estimando o modelo acima por POLS:

	\vspace{-1 em}
	\begin{align} 
		\notag
		\betavec^{POLS} &= 
		\left( \sum_{i=1}^{N} \Xmat_{i}^{*}' \Xmat_{i}^{*} \right)^{-1}
		\left( \sum_{i=1}^{N} \Xmat_{i}^{*}' \yvec_{i}^{*} \right)
		\\ \notag
		&=
		\left( \sum_{i=1}^{N} \Xmat_{i}' \Omega^{-1} \Xmat_{i} \right)^{-1}
		\left( \sum_{i=1}^{N} \Xmat_{i}' \Omega^{-1} \yvec_{i} \right)
		\\ \label{beta:RE:1}
		&=
		\left( \Xmat' (\Imat_{N} \otimes \Omega^{-1}) \Xmat \right)^{-1}
		\left( \Xmat' (\Imat_{N} \otimes \Omega^{-1}) \yvec \right).
	\end{align}

	O problema, agora, é estimar $\Omega$.
	Supondo:
	\begin{itemize}\itemsep0pt
		\item $\E(u_{it}u_{it}) = \sigma_{u}^{2}$;
		\item $\E(u_{it}u_{is}) = 0$.
	\end{itemize}
	Como $\Omega = \E(\mbs{v}_{i} \mbs{v}_{i}') = \E[ ( c_{i} \onevec_{T} + \uvec_{i} ) ( c_{i} \onevec_{T} + \uvec_{i} )' ]$, temos que:

	\vspace{-1 em}
	\begin{align*} 
		\E(v_{it}v_{it}) &=
		\E( c_{i}^{2} + 2c_{i} u_{it} + u_{it}^{2}) 
		=
		\sigma_{c}^{2} + \sigma_{u}^{2}
		\\
		\E(v_{it}v_{is})	&=
		\E[ ( c_{i} + u_{it} ) ( c_{i} + u_{is} ) ]
		=
		\E( c_{i}^{2} + c_{i} u_{is} + u_{it} c_{i} + u_{it} u_{is} )
		=
		\sigma_{c}^{2}.
	\end{align*}

	Assim, 

	\vspace{-1 em}
	\begin{align*}
		\Omega 
		= 
		\E(\mbs{v}_{i} \mbs{v}_{i}') = \sigma^{2}_{u} \Imat_{T} + \sigma_{c}^{2} \onevec_{T} \onevec_{T}'
	\end{align*}

	\noindent
	onde
	$\sigma^{2}_{u} \Imat_{T}$ 
	é uma matriz diagonal, e 
	$\sigma_{c}^{2} \onevec_{T} \onevec_{T}'$ é uma matriz com todos os elementos iguais a $\sigma_{c}^{2}$.

	Agora, rodando POLS em \eqref{mod3:EA} e guardando os resíduos, temos:

	\vspace{-1 em}
	\begin{align*}
		\hat{v}_{it}^{POLS}
		= 
		\hat{y}_{it}^{POLS} - \xvec_{it} \mbs{\hat{\beta}}^{POLS}
	\end{align*}

	\noindent
	e conseguimos estimar $\sigma_{v}^{2}$ e $\sigma_{c}^{2}$ por estimadores amostrais:

	\begin{itemize}\itemsep0pt
		\item 
			como $\sigma_{v}^{2} = \E(v_{it}^{2})$:

			\vspace{-1.5 em}
			\begin{align*}
				\hat{\sigma}_{v}^{2} =
				(NT - K)^{-1} 
				\sum_{i=1}^{N}
				\sum_{t=1}^{T}
				\hat{v}_{it}^2
			\end{align*}
			\vspace{-1.5 em}

		\item 
			como $\sigma_{c}^{2} = \E(v_{it} v_{is})$:

			\vspace{-1.5 em}
			\begin{align*}
				\hat{\sigma}_{c}^{2} =
				\left[ N \frac{T ( T-1 )}{2} - K  \right]^{-1}
				\sum_{i=1}^{N}
				\sum_{t=1}^{T-1}
				\sum_{s=t+1}^{T}
				\hat{v}_{it} \hat{v}_{is}
			\end{align*}
			\vspace{-1.5 em}

		\item $N$ indivíduos;

		\item $T$ elementos da diagonal principal de $\Omega$

		\item $\frac{T ( T - 1)}{2}$ elementos da matriz triangular superior dos elementos fora da diagonal.

		\item $K$ regressores.
	\end{itemize}

	Agora que temos $\hat{\sigma}^2_{v}$ e $\hat{\sigma}^2_{c}$ podemos achar $\hat{\sigma}^{2}_{u}$ pela equação $\boxed{\hat{\sigma}_{u}^{2} = \hat{\sigma}_{v}^{2} - \hat{\sigma}_{c}^{2}}$.
	Dessa forma, achamos os $T^2$ elementos de $\widehat{\Omega}$, e podemos escrever:

	\vspace{-1 em}
	\begin{align*}
		\widehat{\Omega}
		= 
		\hat{\sigma}^{2}_{u} \Imat_{T} + \hat{\sigma}_{c}^{2} \onevec_{T} \onevec_{T}'
	\end{align*}

	Com $\widehat{\Omega}$ estimado, reescrevemos \eqref{beta:RE:1} como:

	\vspace{-1 em}
	\begin{align} \label{beta:RE:2}
		\betavec^{RE} = 
		\left[ \Xmat' (\Imat_{N} \otimes \widehat{\Omega}^{-1}) \Xmat \right]^{-1}
		\left[ \Xmat' (\Imat_{N} \otimes \widehat{\Omega}^{-1}) \yvec \right].
	\end{align}


%%%%%%%%%%%%%%%%%%%%%%%%%%%%%%%%%%%%%%%%%%%%%%%%%%%%%%
	\subsection{Valor Esperado}

	\vspace{-1 em}
	\begin{align*}
		\Aboxed{
			\E( \betavec^{RE} ) = 
			\betavec +
			\left[ \Xmat' (\Imat_{N} \otimes \widehat{\Omega}^{-1}) \Xmat \right]^{-1}
		\left[ \Xmat' (\Imat_{N} \otimes \widehat{\Omega}^{-1}) \mbs{v} \right] }.
	\end{align*}

%%%%%%%%%%%%%%%%%%%%%%%%%%%%%%%%%%%%%%%%%%%%%%%%%%%%%%
	\subsection{Variância}

	\vspace{-1 em}
	\begin{align*} 
		\Var( \betavec^{RE} ) = 
		E
		\left\{ 
			\left[ \Xmat' ( \Imat_{N} \otimes \widehat{\Omega}^{-1} ) \Xmat \right]^{-1}
			\left[
				\Xmat' ( \Imat_{N} \otimes \widehat{\Omega}^{-1} )
				\mbs{v} \mbs{v}'
				( \Imat_{N} \otimes \widehat{\Omega}^{-1} )' \Xmat
			\right]
			\left[ \Xmat' ( \Imat_{N} \otimes \widehat{\Omega}^{-1} ) \Xmat \right]
		\right\},
	\end{align*}

	\noindent
	como $\E( \mbs{v}_{i} \mbs{v}_{i}' ) =\Omega$,

	\vspace{-1 em}
	\begin{align*} 
		\Aboxed{
			\Var( \betavec^{RE} ) = 
			E
		\left[ \Xmat' ( \Imat_{N} \otimes \widehat{\Omega}^{-1} ) \Xmat \right] }.
	\end{align*}



%%%%%%%%%%%%%%%%%%%%%%%%%%%%%%%%%%%%%%%%%%%%%%%%%%%%%%
	\clearpage
	\section{Endogeneity and GMM}


%%%%%%%%%%%%%%%%%%%%%%%%%%%%%%%%%%%%%%%%%%%%%%%%%%%%%%
	\subsection{Modelo}

	No seguinte modelo \textit{cross-section}:

	\vspace{-1 em}
	\begin{align} \label{mod1}
		y_{i} = \beta_{0} + \beta_{1} x_{1i} + \beta_{2} x_{2i} + \err_{i}
		\; ; \quad i = 1, \dots, N.
	\end{align}

	\noindent
	A variável explicativa $x_{k}$ é dita \textbf{endógena} se ela for correlacionada com erro.
	Se $x_{k}$ for não correlacionada com o erro, então $x_{k}$ é dita \textbf{exógena}.

	Endogeneidade surge, normalmente, de três maneiras diferentes:

	\begin{enumerate}\itemsep0pt
		\item Variável Omitida;
		\item Simultaneidade;
		\item Erro de Medida.
	\end{enumerate}

	No modelo \eqref{mod1} vamos supor:

	\begin{itemize}\itemsep0pt
		\item $x_{1}$ é exógena.
		\item $x_{2}$ é endógena.
	\end{itemize}

%%%%%%%%%%%%%%%%%%%%%%%%%%%%%%%%%%%%%%%%%%%%%%%%%%%%%%
	\subsection{Hipóteses}

	Assim, precisamos encontrar um instrumento $z_{i}$ para $x_{2}$, uma vez que queremos estimar $\beta_{0}$, $\beta_{1}$ e $\beta_{2}$ de maneira consistente.
	Para $z_{i}$ ser um bom instrumento precisamos que $z$ tenha:

	\begin{enumerate}\itemsep0pt
		\item $Cov(z, \err) = 0$ $\implies$  $z$ é exógena em \eqref{mod1}.
		\item $Cov(z, x_{2}) \neq 0$ $\implies$  correlação com $x_{2}$ após controlar para outras vaariáveis.
	\end{enumerate}

%%%%%%%%%%%%%%%%%%%%%%%%%%%%%%%%%%%%%%%%%%%%%%%%%%%%%%
	\subsection{Estimação}

	Indo para o problema de dados de painel, temos:

	\vspace{-1 em}
	\begin{align} \label{mod2}
		\yvec_{i} = \Xmat_{i} \betavec + \uvec_{i}
		\; ; \quad i = 1, \dots, N.
	\end{align}

	\noindent
	onde 
	$\yvec_{i}$ é um vetor $T \times 1$,
	$\Xmat_{i}$ é uma matriz $T \times K$,
	$\betavec$ é o vetor de coeficientes $K \times 1$,
	$\uvec_{i}$ é o vetor de erros $T \times 1$.

	Se é verdade que há endogeneidade em \eqref{mod2}, então:

	\vspace{-1 em}
	\begin{align*}
		\E(\Xmat_{i}^{\prime} \uvec_{i}) \neq 0
	\end{align*}

	Definimos $Z_{i}$ como uma matriz $T \times L$ com $L \geq K$ de variáveis exógenas (incluindo o instrumento).
	Queremos acabar com a endogeneidade, ou seja:

	\vspace{-1 em}
	\begin{align*}
		\E(Z_{i}^{\prime} \uvec_{i}) = 0
	\end{align*}

	Supondo $L = K$ (apenas substituímos a variável endógena por um instrumento).

	\vspace{-1 em}
	\begin{align*}
		\E[ Z_{i}^{\prime} ( \yvec_{i} - \Xmat_{i} \betavec ) ] &= 0
		\\
		\E( Z_{i}^{\prime} \yvec_{i} ) - \E( Z_{i}^{\prime} \Xmat_{i} ) \betavec &= 0
		\\
		\E( Z_{i}^{\prime} \yvec_{i} ) &= \E( Z_{i}^{\prime} \Xmat_{i} ) \betavec
		\\
		\Aboxed{
			\betavec &=
			\left[ \E( Z_{i}^{\prime} \Xmat_{i} ) \right]^{-1}
			\left[ \E( Z_{i}^{\prime} \yvec_{i} ) \right]
		}
	\end{align*}

	Se Usarmos estimadores amostrais:

	\vspace{-1 em}
	\begin{align*}
		\mbs{\hat{\beta}} &=
		\left[ N^{-1} \sum_{i=1}^{N} Z_{i}^{\prime} \Xmat_{i} \right]^{-1}
		\left[ N^{-1} \sum_{i=1}^{N} Z_{i}^{\prime} \yvec_{i} \right]
		\\
		\Aboxed{
			\mbs{\hat{\beta}} &=
		( Z^{\prime} \Xmat )^{-1} ( Z^{\prime} \yvec ) }
	\end{align*}

	\vspace{1 em}
	Se $L > K$, vamos considerar:

	\vspace{-1 em}
	\begin{align*}
		\underset{\betavec}{\text{Min}} \;
		\E( Z_{i}\uvec_{i} )^2
	\end{align*}
	\noindent onde:

	\vspace{-1 em}
	\begin{align*}
		\E( Z_{i}\uvec_{i} )^2 
		&=
		\E[ ( Z_{i}\uvec_{i} )' ( Z_{i}\uvec_{i} ) ]
		=
		( Z' \yvec - Z' \Xmat \betavec )' ( Z' \yvec - Z' \Xmat \betavec )
		\\
		&=
		\yvec' ZZ' \yvec
		-
		\yvec' ZZ' \Xmat \betavec
		-
		\betavec' \Xmat' ZZ' \yvec
		+
		\betavec' \Xmat' ZZ' \Xmat \betavec
	\end{align*}

	Derivando em relação em $\betavec$ e igualando a zero:

	\vspace{-1 em}
	\begin{align*}
		-2 \yvec' ZZ' \Xmat + 2 \betavec'\Xmat' ZZ' \Xmat &= 0
		\\
		\betavec'\Xmat' ZZ' \Xmat &= \yvec' ZZ' \Xmat 
		\\
		\betavec' &= ( \yvec' ZZ' \Xmat ) ( \Xmat' ZZ' \Xmat )^{-1}
		\\
		\Aboxed{
		\betavec &= ( \Xmat' ZZ' \Xmat )^{-1} ( \Xmat' ZZ' \yvec ) }
	\end{align*}

	Um estimador mais eficiente pode ser encontrado fazendo:

	\vspace{-1 em}
	\begin{align*}
		\underset{\betavec}{\text{Min}} \;
		\E[ ( Z_{i}' \yvec - Z' \Xmat \betavec )' W ( Z_{i}' \yvec - Z' \Xmat \betavec ) ].
	\end{align*}

	\noindent
	Escolhendo $\widehat{W}$, a priori, temos:

	\vspace{-1 em}
	\begin{align*}
		\underset{\betavec}{\text{Min}} \;
		\left\{ 
			\yvec' Z \widehat{W} Z' \yvec
			-
			\yvec' Z \widehat{W} Z' \Xmat \betavec
			-
			\betavec' \Xmat'  Z \widehat{W} Z' \yvec
			+
			\betavec' \Xmat'  Z \widehat{W} Z' \Xmat \betavec
		\right\}
	\end{align*}

	Derivando em relação em $\betavec$ e igualando a zero:

	\vspace{-1 em}
	\begin{align*}
		-2 \yvec' Z \widehat{W} Z' \Xmat + 2 \betavec'\Xmat' Z \widehat{W} Z' \Xmat &= 0
		\\
		\betavec'\Xmat' Z \widehat{W} Z' \Xmat &= \yvec' Z \widehat{W} Z' \Xmat 
		\\
		\betavec' &= ( \yvec' Z \widehat{W} Z' \Xmat ) ( \Xmat' Z \widehat{W} Z' \Xmat )^{-1}
		\\
		\Aboxed{
		\betavec^{GMM} &= ( \Xmat' Z \widehat{W}' Z' \Xmat )^{-1} ( \Xmat' Z \widehat{W}' Z' \yvec ) }
	\end{align*}

%%%%%%%%%%%%%%%%%%%%%%%%%%%%%%%%%%%%%%%%%%%%%%%%%%%%%%
	\subsection{Valor Esperado} 

	\vspace{-1 em}
	\begin{align*}
		\Aboxed{
			\E( \betavec^{GMM} ) &=
			\betavec +
		\E[ ( \Xmat' Z \widehat{W}' Z' \Xmat )^{-1} ( \Xmat' Z \widehat{W}' Z' \uvec ) ] }.
	\end{align*}

%%%%%%%%%%%%%%%%%%%%%%%%%%%%%%%%%%%%%%%%%%%%%%%%%%%%%%
	\subsection{Variância} 

	\vspace{-1 em}
	\begin{align*}
		\Var( \betavec^{GMM} ) &=
		\E \left\{ 
			\left[ ( X' Z \widehat{W}' Z' X )^{-1} ( X' Z \widehat{W}' Z' \uvec ) \right]
			\left[ ( X' Z \widehat{W}' Z' X )^{-1} ( X' Z \widehat{W}' Z' \uvec ) \right]'
		\right\}
		\\ &=
		\E \left\{ 
			( X' Z \widehat{W}' Z' X )^{-1}
			X' Z \widehat{W}' Z' \uvec \uvec' Z \widehat{W} Z' X 
			( X' Z \widehat{W} Z' X )^{-1}
		\right\}.
	\end{align*}

	\noindent
	Definindo $\Delta = \E(Z' \uvec\uvec' Z)$ com $\Delta = W^{-1}$:

	\vspace{-1 em}
	\begin{align*}
		\Var( \betavec^{GMM} ) &=
		\E \left\{ 
			( X' Z \widehat{W}' Z' X )^{-1}
			X' Z \widehat{W}' W^{-1} \widehat{W} Z' X 
			( X' Z \widehat{W} Z' X )^{-1}
		\right\}
		\\ &=
		\E \left\{ 
			( X' Z \widehat{W}' Z' X )^{-1}
			( X' Z \widehat{W}' Z' X )
			( X' Z \widehat{W} Z' X )^{-1}
		\right\}.
		\\
		\Aboxed{
			\Var( \betavec^{GMM} ) &=
			\E \left[
				( X' Z \widehat{W} Z' X )^{-1}
		\right] }.
	\end{align*}

	\noindent
	Se tivéssemos definido $W = (Z'Z)^{-1}$, teríamos $\beta^{2SLS}$.


%%%%%%%%%%%%%%%%%%%%%%%%%%%%%%%%%%%%%%%%%%%%%%%%%%%%%%
	\clearpage
	\section{Exogeneidade Estrita e FDIV}


%%%%%%%%%%%%%%%%%%%%%%%%%%%%%%%%%%%%%%%%%%%%%%%%%%%%%%
	\subsection{Modelo}

	No seguinte modelo

	\vspace{-1 em}
	\begin{align*} 
		y_{it} = \xvec_{it} \betavec + u_{it},
	\end{align*}

	\noindent
	para
	$t = 1, \dots, T$ e $i = 1, \dots, N$.

	\begin{itemize}\itemsep0pt
		\item
			$y_{it}$ escalar;

		\item
			$\xvec_{it}$  vetor $1 \times K$;

		\item
			$\betavec$ vetor $K \times 1$;

		\item
			$u_{it}$ escalar.
	\end{itemize}

	\noindent
	$\{x_{it}\}$ é estritamente \textbf{exógeno} se valer:

	\vspace{-1 em}
	\begin{align*}
		\E( u_{it} \, | \, \xvec_{i1}, \dots, \xvec_{iT} ) = 0 \; , \qquad t = 1, \dots, T
	\end{align*}

	\noindent
	ou seja:

	\vspace{-1 em}
	\begin{align*}
		\E( y_{it} \, | \, \xvec_{i1}, \dots, \xvec_{iT} ) = \xvec_{it} \betavec 
		\; , \qquad t = 1, \dots, T
	\end{align*}

	\noindent
	o que é equivalente a hipótese de que utilizamos o modelo linear correto.

	Para o seguinte modelo:

	\vspace{-1.5 em}
	\begin{align*}
		y_{it} = \mbs{z}_{it} \mbs{\gamma} + \rho y_{i t - 1} + c_{i} + u_{it}.
		\; , \qquad t = 2, \dots, T
	\end{align*}

	\noindent
	é \textbf{impossível} termos exogeneidade estrita.
	Isso porque, nesse modelo, de efeitos não observados temos:

	\vspace{-1.5 em}
	\begin{align*}
		\E( y_{it} \, | \, \mbs{z}_{i1}, \dots, \mbs{z}_{iT}, y_{it-1}, c_{i}) \neq 0.
	\end{align*}

	\noindent
	Isso ocorre porque, $y_{it}$ é afetado por $y_{it-1}$ que contribui para $y_{it}$ com, pelo menos, $\rho c_{i}$.

	\begin{equation*}
		\left.
			\begin{aligned}
				y_{it} &= \mbs{z}_{it} \mbs{\gamma} + \rho y_{i t - 1} + c_{i} + u_{it}
				\\
				y_{it-1} &= \mbs{z}_{it-1} \mbs{\gamma} + \rho y_{i t - 2} + c_{i} + u_{it-1}
			\end{aligned}
		\right\} 
		\implies
		y_{it} = \mbs{z}_{it} \mbs{\gamma} +
		\rho (\mbs{z}_{it-1} \mbs{\gamma} + \rho y_{i t - 2} + c_{i} + u_{it-1})
		+ c_{i} + u_{it}.
	\end{equation*}

	Para eliminarmos este efeito, podemos tirar a primeira diferença do modelo:

	\vspace{-1 em}
	\begin{align}
		\nonumber
		y_{it} - y_{it-1} &= 
		(\mbs{z}_{it} - \mbs{z}_{it-1}) \mbs{\gamma} +
		\rho (y_{i t - 1} -  y_{i t - 2} ) +
		(c_{i} - c_{i}) + (u_{it} - u_{it-1})
		\\
		\label{mod1:FDIV}
		\Aboxed{
			\Delta y_{it} &= 
			\Delta \mbs{z}_{it} \mbs{\gamma} + \rho \Delta y_{i t - 1} + \Delta u_{it}
		\, , \qquad t=3, \dots, T}
	\end{align}

%%%%%%%%%%%%%%%%%%%%%%%%%%%%%%%%%%%%%%%%%%%%%%%%%%%%%%
	\subsection{Estimação}

	Não podemos estimar o modelo \eqref{mod1:FDIV} por POLS, uma vez que $Cov(\Delta y_{it-1}, \Delta u_{it} ) \neq 0$.
	Como saída, podemos estimar por P2SLS, usando instrumentos para $\Delta y_{it-1}$ (alguns intrumentos para $\Delta y_{it-1}$ são $y_{it-2}, y_{it-3}, \dots, y_{i1}$).

%%%%%%%%%%%%%%%%%%%%%%%%%%%%%%%%%%%%%%%%%%%%%%%%%%%%%%
	\subsubsection{P2SLS}

	\vspace{-1 em}
	\begin{align*}
		y_{it} = \xvec_{it}' \betavec + u_{it}
	\end{align*}

	\begin{itemize}\itemsep0pt
		\item $i = 1, \dots, N$
		\item $t = 1, \dots, T$
		\item $y_{it}$ escalar;
		\item $\xvec_{it}$  vetor $K \times 1$;
		\item $\betavec$ vetor $K \times 1$;
		\item $u_{it}$ escalar.
	\end{itemize}

	\vspace{-1 em}
	\begin{align*}
		\boxed{
		\betavec^{P2SLS} =  ( X' P_{Z} X )^{-1} ( X' P_{Z} \yvec ) }
	\end{align*}

	\noindent
	com

	\vspace{-1 em}
	\begin{align*}
		\boxed{P_{Z} = Z'(Z'Z)^{-1}Z }
	\end{align*}

	\noindent
	onde
	$P_{Z}$ é a matriz de projeção em $Z$.

%%%%%%%%%%%%%%%%%%%%%%%%%%%%%%%%%%%%%%%%%%%%%%%%%%%%%%
	\subsubsection{FDIV}

	\vspace{-1 em}
	\begin{align*}
		y_{it} &= \xvec_{it}' \betavec + c_{i} + u_{it}
		\; , \quad i = 1, \dots, N
		\; , \quad t = 1, \dots, T
		\\
		\Delta y_{it} &= \Delta \xvec_{it}' \betavec + \Delta u_{it}
		\; , \quad i = 1, \dots, N
		\; , \quad t = 2, \dots, T
	\end{align*}

	Vamos supor $\Delta x_{it}'$ tem variável endógena ($y_{it}$, no caso).
	$\mbs{w}_{it}$ é um vetor $1 \times L_{t}$ de instrumentos, onde $L_{t} \geq K$.
	Se os instrumentos forem diferentes:

	\vspace{-1 em}
	\begin{align*}
		W_{i} = diag( \mbs{w}_{i2}', \mbs{w}_{i3}', \dots, \mbs{w}_{iT}')
	\end{align*}

	\noindent
	onde $W_{i}$ é uma matriz $( T - 1 ) \times L$

	\vspace{-1 em}
	\begin{align*}
		L = L_{2} + L_{3} + \dots + L_{T}
	\end{align*}

%%%%%%%%%%%%%%%%%%%%%%%%%%%%%%%%%%%%%%%%%%%%%%%%%%%%%%
	\subsection{Hipóteses}

	\begin{description}
		\item[FDIV.1:] $\E( \mbs{w}_{it} \Delta u_{it}')$ para $i = 1, \dots, N$, $t = 2, \dots, T$.
		\item[FDIV.2:] $Posto\left[ \E( W_{i}' W_{i} ) \right] = L$
		\item[FDIV.3:] $Posto\left[ \E( W_{i}' \Delta X_{i} ) \right] = K$
	\end{description}

%%%%%%%%%%%%%%%%%%%%%%%%%%%%%%%%%%%%%%%%%%%%%%%%%%%%%%
	\subsection{Estimação FDIV}

	\vspace{-1 em}
	\begin{align*}
		\boxed{
			\betavec^{FDIV} =  
			\left(
				\Delta X' P_{W} \Delta X 
			\right)^{-1}
			\left(
				\Delta X' P_{W} \Delta \yvec
			\right)
		}
		\qquad
		\boxed{
		P_{W} = W(W'W)^{-1}W'}
	\end{align*}

%%%%%%%%%%%%%%%%%%%%%%%%%%%%%%%%%%%%%%%%%%%%%%%%%%%%%%
	\subsection{Valor Esperado}

	\vspace{-1 em}
	\begin{align*}
		\E( \betavec^{FDIV} ) =  
		\beta + 
		\left( \Delta X' P_{W} \Delta X \right)^{-1}
		\left( \Delta X' P_{W} \mbs{e} \right)
	\end{align*}

%%%%%%%%%%%%%%%%%%%%%%%%%%%%%%%%%%%%%%%%%%%%%%%%%%%%%%
	\subsection{Variância}

	\vspace{-1 em}
	\begin{align*}
		\Var( \betavec^{FDIV} ) &=
		\E\left\{  
			\left[ \E( \betavec^{FDIV} ) - \beta \right] 
			\left[ \E( \betavec^{FDIV} ) - \beta \right]'
		\right\}
		\\
		&=
		\E\left\{  
			\left[ \Delta X' P_{W} \Delta X \right]^{-1}
			\left[ \Delta X' P_{W} \mbs{e} \right]
			\left[ \Delta X' P_{W} \mbs{e} \right]'
			\left[ \Delta X' P_{W} \Delta X \right]^{-1}
		\right\}
		\\
		&=
		\E\left[
			\left( \Delta X' P_{W} \Delta X \right)^{-1}
			\left( \Delta X' P_{W} \mbs{e} \mbs{e}' P_{W} \Delta X \right)
			\left( \Delta X' P_{W} \Delta X \right)^{-1}
		\right]
	\end{align*}

	\noindent
	$e_{i} = \Delta u_{it}$.

%%%%%%%%%%%%%%%%%%%%%%%%%%%%%%%%%%%%%%%%%%%%%%%%%%%%%%
	\clearpage
	\section{Latent Variables, Probit and Logit}

%%%%%%%%%%%%%%%%%%%%%%%%%%%%%%%%%%%%%%%%%%%%%%%%%%%%%%
	\subsection{Modelo}

	Suponha $y^{*}$ não observável (\textbf{latente}) seguindo o seguinte modelo:

	\vspace{-1 em}
	\begin{align} \label{mod1:probit}
		y_{i}^{*} = \xvec_{i}' \betavec + \err_{i}.
	\end{align}

	\noindent
	Defina $y$ como:

	\vspace{-1 em}
	\begin{align*}
		y_{i} =
		\begin{cases}
			1 \, , \quad y^{*}_{i} \geq 0
			\\
			0 \, , \quad y^{*}_{i} < 0
		\end{cases}
	\end{align*}

	\noindent
	temos que:

	\vspace{-1 em}
	\begin{align*}
		P( y_{i} = 1 | \xvec ) &= p( \xvec )
		\\
		P( y_{i} = 0 | \xvec ) &= 1 - p( \xvec ).
	\end{align*}

	Além disso, pela definição de $y_{i}$, equação \eqref{mod1:probit}, temos:

	\vspace{-1 em}
	\begin{align*}
		P( y_{i} = 1 | \xvec ) &= P(y_{i}^{*} \geq 0 \, | \xvec )
		\\
		&= P( \xvec_{i}' \betavec + \err_{i} \geq 0 \, | \xvec )
		\\
		&= P( \err_{i} \geq - \xvec_{i}' \betavec  \, | \xvec ).
	\end{align*}

	\noindent
	Agora, supondo que $\err_{i}$ tem FDA, $G$, tal que $G'=g$ é simétrica ao redor de zero:

	\vspace{-1 em}
	\begin{align*}
		P( y_{i} = 1 | \xvec ) 
		&= 1 - P( \err_{i} < - \xvec_{i}' \betavec  \, | \xvec )
		\\
		&= 1 - G( - \xvec_{i}' \betavec  \, | \xvec )
		\\
		&= G( \xvec_{i}' \betavec ).
	\end{align*}

	Se $G(\cdot)$ for uma distribuição:

	\begin{description}
		\item [Normal Padrão:] $\hat{\betavec}$ é o estimador \textbf{probit}.
		\item [Logística:] $\hat{\betavec}$ é o estimador \textbf{logit}.
	\end{description}

	Supondo $\yvec_{i} \, | \, \xvec \sim Bernoulli(p(\xvec))$, sua fmp é dada por:

	\vspace{-1 em}
	\begin{align*}
		f( y_{i} \, | \, \xvec_{i} ; \betavec ) 
		&= 
		\left[ G(\xvec_{i}' \betavec )  \right]^{y_{i}}
		\left[ 1 - G(\xvec_{i}' \betavec )  \right]^{1 - y_{i}}
		\; , \quad y=0,1.
	\end{align*}

	Para estimarmos $\hat{\betavec}$ por máxima verossimilhança, temos de encontrar $\betavec \in B$, onde $B$ é o espaço paramétrico, tal que $\betavec$ maximize o valor da distribuição conjunta de $\yvec$, ou seja:

	\vspace{-1 em}
	\begin{align*}
		\underset{\betavec \in B}{\text{Max }} 
		\prod_{i=1}^{N}
		f( y_{i} \, | \, \xvec_{i} ; \betavec ).
	\end{align*}

	\noindent

	Tirando o logaritmo e dividindo tudo por $N$ (podemos fazer isso pois são transformações monotônicas e não alteram o lugar onde $\betavec$ ótimo irá parar):

	\vspace{-1 em}
	\begin{align*}
		\underset{\betavec \in B}{\text{Max }} 
		\left\{ 
			N^{-1} \sum_{i=1}^{N}
			\ln \left[ f( y_{i} \, | \, \xvec_{i} ; \betavec ) \right]
		\right\}.
	\end{align*}

	\noindent
	Podemos definir
	$\ell_{i}( \betavec ) = \ln[ f( y_{i} \, | \, \xvec_{i} ; \betavec ) ]$
	como sendo a verossimilhança condicional da observação $i$:

	\vspace{-1 em}
	\begin{align*}
		\underset{\betavec \in B}{\text{Max }} 
		\left\{ 
			N^{-1} \sum_{i=1}^{N} \ell_{i} (\betavec)
		\right\}.
	\end{align*}

	Dessa forma, podemos ver que o problema acima é a analogia amostral de:

	\vspace{-1 em}
	\begin{align*}
		\underset{\betavec \in B}{\text{Max }} 
		\E \left[ 
			\ell_{i} ( \betavec )
		\right].
	\end{align*}

	Definindo o \textit{vector score} da observação $i$:

	\vspace{-1 em}
	\begin{align*}
		s_{i} (\betavec) = 
		\left[ \nabla_{\betavec} \ell_{i} (\betavec) \right]'
		=
		\begin{bmatrix}
			\dfrac{\partial{\ell_{i} (\betavec)}}{\partial{\beta_{1}}},
			\dots,
			\dfrac{\partial{\ell_{i} (\betavec)}}{\partial{\beta_{K}}}
		\end{bmatrix}
	\end{align*}

	Definindo a \textbf{Matriz Hessiana} da observação $i$:

	\vspace{-1 em}
	\begin{align*}
		H_{i} (\betavec) = 
		\nabla_{\betavec} s_{i} (\betavec) = 
		\nabla_{\betavec}^2 \ell_{i} (\betavec)
	\end{align*}

	Tendo essas definições, o \textbf{Teorema do Valor Médio} (TVM) nos diz que no intervalo $[a, b]$, existe um número, $c$, tal que:

	\vspace{-1 em}
	\begin{align*}
		f'(c) = \frac{f(b) - f(a)}{b - a}.
	\end{align*}

%%%%%%%%%%%%%%%%%%%%%%%%%%%%%%%%%%%%%%%%%%%%%%%%%%%%%%
	\vspace{1 em}
	\begin{center}
		\red{FAZER DESENHO}
	\end{center}
	\vspace{1 em}
%%%%%%%%%%%%%%%%%%%%%%%%%%%%%%%%%%%%%%%%%%%%%%%%%%%%%%

	Trocando 
	$f(\cdot)$ por $s_{i}(\cdot)$, 
	$a$ por $\betavec_{0}$, 
	$b$ por $\widehat{\betavec}$ e
	$c$ por $\bar{\betavec}$,
	temos:

	\vspace{-1 em}
	\begin{align*}
		H_{i} ( \bar{\betavec} ) =
		\frac{s_{i}( \widehat{\betavec} ) - s_{i}( \betavec_{0} )}{\widehat{\betavec} - \betavec_{0}},
	\end{align*}

	\noindent
	tirando médias dos dois lados:

	\vspace{-1 em}
	\begin{align*}
		N^{-1} \sum_{i=1}^{N} 
		H_{i} ( \bar{\betavec} ) 
		=
		\frac{1}{\widehat{\betavec} - \betavec_{0}}
		N^{-1} \sum_{i=1}^{N} 
		\left[ 
			s_{i}( \widehat{\betavec} ) - s_{i}( \betavec_{0} )
		\right]
	\end{align*}

	Supondo que
	$\widehat{\betavec}$
	maximiza
	$\ell (\betavec \, | \, \yvec, \xvec)$,
	temos que:
	$N^{-1} \sum_{i=1}^{N} s_{i}(\widehat{\betavec}) = 0$.
	E podemos reescrever a equação anterior como:

	\vspace{-1 em}
	\begin{align*}
		\widehat{\betavec} - \betavec_{0}
		&=
		(-1)
		\left[ N^{-1} \sum_{i=1}^{N} H_{i} ( \bar{\betavec} ) \right]^{-1}
		N^{-1} \sum_{i=1}^{N} s_{i}( \betavec_{0} ) 
		\\
		\sqrt{N} ( \widehat{\betavec} - \betavec_{0} )
		&=
		\left[
			- N^{-1} \sum_{i=1}^{N} H_{i} ( \bar{\betavec} )
		\right]^{-1}
		\sqrt{N} \cdot N^{-1} \sum_{i=1}^{N} s_{i}( \betavec_{0} ) 
		\\
		\Aboxed{
			\sqrt{N} ( \widehat{\betavec} - \betavec_{0} )
			&=
			\left[
				- N^{-1} \sum_{i=1}^{N} H_{i} ( \bar{\betavec} )
			\right]^{-1}
		N^{-1/2} \sum_{i=1}^{N} s_{i}( \betavec_{0} ) }.
	\end{align*}

	\noindent
	Onde

	\vspace{-1 em}
	\begin{align*}
		\left[ 
			- N^{-1} \sum_{i=1}^{N}
		H_{i} ( \bar{\betavec} ) \right]^{-1}
		\xrightarrow{p}
		A_{0}^{-1} \, ,
		&&
		N^{-1/2} \sum_{i=1}^{N} s_{i}( \betavec_{0} ) 
		\xrightarrow{d}
		N( 0, B_{0} ).
	\end{align*}

	\noindent
	Assim, temos que:

	\vspace{-1 em}
	\begin{align*}
		\Aboxed{
			\sqrt{N} ( \widehat{\betavec} - \betavec_{0} )
			\to
		N ( 0, A_{0}^{-1} B_{0} A_{0}^{-1} )}.
	\end{align*}

	A forma mais simples de achar $\Var ( {\widehat{\betavec}} )$ é:

	\vspace{-1 em}
	\begin{align*}
		\Aboxed{
			\Var( \widehat{\betavec} )
			&=
		- \E[ H_{i} ( \widehat{\betavec} ) ]^{-1} }.
	\end{align*}

%%%%%%%%%%%%%%%%%%%%%%%%%%%%%%%%%%%%%%%%%%%%%%%%%%%%%%
	\clearpage
	\section{ATT, ATE, Propensity Score}


%%%%%%%%%%%%%%%%%%%%%%%%%%%%%%%%%%%%%%%%%%%%%%%%%%%%%%
	\subsection{Modelo}

	\begin{itemize}\itemsep0pt
		\item
			$y_{1}$ $\rightarrow$ variável de interesse com tratamento

		\item
			$y_{0}$ $\rightarrow$ variável de interesse sem tratamento
	\end{itemize}

	\vspace{-1 em}
	\begin{align*}
		w = 
		\begin{cases}
			1 & \text{se tratam}
			\\
			0 & \text{se não tratam}
		\end{cases}
	\end{align*}

	Idealmente, para isolarmos completamente o efeito de $w=1$, gostaríamos de pode calcular:

	\vspace{-1 em}
	\begin{align*}
		N^{-1} \sum_{i=1}^{N}
		\left( y_{i1} - y_{i0} \right).
	\end{align*}

	Ou seja, o efeito que o tratamento causa sobre um indivíduo com todo o resto permanecendo constante.
	Em outras palavras, queríamos que houvesse dois mundos paralelos observáveis onde seria possível observar o que acontece com $y_{i}$ com e sem tratamento.
	Infelizmente, para ccada indivíduo $i$, observamos apenas $y_{i1}$ ou $y_{i0}$, nunca ambos.

	Antes de continuarmos, faremos as seguintes definições:

	\begin{description}
		\item[ATE:]  $\E( y_{1} - y_{0} )$
		\item[ATT:]  $\E( y_{1} - y_{0} \, | \, w = 1 )$ (ATE no tratado).
	\end{description}

%%%%%%%%%%%%%%%%%%%%%%%%%%%%%%%%%%%%%%%%%%%%%%%%%%%%%%
	\subsection{ATE e ATT condicional a variáveis dependentes}

	\vspace{-1 em}
	\begin{align*}
		ATE( \xvec ) &= \E( y_1 - y_0 \, | \xvec)
		\\
		ATT( \xvec ) &= \E( y_1 - y_0 \, | \xvec, w = 1)
	\end{align*}

	\noindent
	\underline{OBS:}

	\vspace{-1 em}
	\begin{align*}
		\E( y_1 - y_0 ) &= \E \left[ \E( y_1 - y_0 \, | w) \right]
		\\
		\E( y_1 - y_0 \, | w ) &=
		\E( y_1 - y_0 \, | w = 0 ) \cdot P(w=0)
		+
		\E( y_1 - y_0 \, | w = 1 ) \cdot P(w=1).
	\end{align*}

%%%%%%%%%%%%%%%%%%%%%%%%%%%%%%%%%%%%%%%%%%%%%%%%%%%%%%
	\subsection{Métodos Assumindo Ignorabilidade do Tratamento}

	\begin{description}
		\item[ATE.1:] Ignorabilidade. 
			\\
			$w$ e $(y_{1}, y_{0})$ são independentes condicionais a $\xvec$.

		\item[ATE.1':] Ignorabilidade da Média. 

			\vspace{-.75 em}
			\begin{enumerate}[label =\alph*)] \itemsep0pt
			\item $\E( y_{0} \, | \, w, \xvec ) = \E( y_{0} \, | \, \xvec )$
			\item $\E( y_{1} \, | \, w, \xvec ) = \E( y_{1} \, | \, \xvec )$
		\end{enumerate}

\end{description}

Vamos definir

\vspace{-1 em}
\begin{align*}
	\E( y_{0} \, | \, \xvec ) &= \mu_{0}( \xvec )
	\\
	\E( y_{1} \, | \, \xvec ) &= \mu_{1}( \xvec ).
\end{align*}

Sob \textbf{ATE.1} e \textbf{ATE.1'}:

\vspace{-1 em}
\begin{align*}
	ATE( \xvec ) &= \E( y_{1} - y_{0} \, | \xvec ) = \mu_{1}(\xvec) - \mu_{0}(\xvec) 
	\\
	ATT( \xvec ) &= \E( y_{1} - y_{0} \, | \xvec, w=1 ) = \mu_{1}(\xvec) - \mu_{0}(\xvec) 
\end{align*}

\begin{description}
	\item[ATE.2:] \textit{Overlap} \\
		Para todo $\xvec$, $P(w=1 \, | \, \xvec ) \in ( 0, 1 )$, 
		$p(\xvec) = p(w=1 | \xvec)$.
\end{description}

$p(\xvec)$ é o \textit{Propensity Score}, ele representa a probabilidade de $y_{i}$ ser tratado dado o valor das covariáveis $\xvec$.
Essa hipótese é importante visto que podemos expressar o $ATE$ em função de $p(\xvec)$.

\vspace{1 em}
Para o $ATT$ vamos supor:

\begin{description}
	\item[ATT.1':] 
		$\E( y_{0} \, | \xvec, w ) = \E( y_{0} \, | \, \xvec )$

	\item[ATT.2:] \textit{Overlap:} Para todo $\xvec$, $P(w=1 | \xvec ) < 1$.
\end{description}

%%%%%%%%%%%%%%%%%%%%%%%%%%%%%%%%%%%%%%%%%%%%%%%%%%%%%%
\subsection{Propensity Score}

Como foi dito anteriormente, apenas observamos ou $y_{1}$ ou $y_{0}$ para a mesma pessoa, mas não ambos.
Mais precisamente, junto com $w$, o resultado observado é:

\vspace{-1 em}
\begin{align*}
	y = wy_{1} + (1 - w) y_{0}
\end{align*}

\noindent
como  $w$ é binário, $w^2 = w$, assim, temos:

\vspace{-1 em}
\begin{align*}
	w y &= w^{2} y_{1} + (w - w^{2}) y_{0}
	\implies
	\boxed{w y = w y_{1} }
	\\
	( 1 - w ) y &= (w - w^{2}) y_{1} + ( w^{2} - 2w + 1 ) y_{0}
	\implies
	\boxed{( 1 - w ) y = (1 - w) y_{0}}.
\end{align*}

Fazemos isso para tentar isolar $\mu_{0}(\xvec)$ e $\mu_{1}(\xvec)$:

%%%%%%%%%%%%%%%%%%%%%%%%%%%%%%%%%%%%%%%%%%%%%%%%%%%%%%
\paragraph{$\mu_{1}( \xvec )$}

\begin{align*}
	\E( w y | \xvec ) &= \E\left[  \E \left( w y_{1} | \xvec, w  \right) | \xvec \right]
	\\ &=
	\E \left[ w \mu_{1}(\xvec) | \xvec \right]
	\\ &=
	\mu_{1}(\xvec) \E(w | \xvec ).
\end{align*}

\noindent
Como $w$ é binaria: $\E(w| \xvec) = P(w=1 | \xvec) = p(\xvec)$.
Assim:

\vspace{-1 em}
\begin{align*}
	\E( w y | \xvec ) &= \mu_{1}(\xvec) p(\xvec)
	\\
	\Aboxed{ \mu_{1}(\xvec) &= \frac{\E(w y | \xvec)}{p(\xvec)} }
\end{align*}

%%%%%%%%%%%%%%%%%%%%%%%%%%%%%%%%%%%%%%%%%%%%%%%%%%%%%%
\paragraph{$\mu_{0}( \xvec )$}

\vspace{-1 em}
\begin{align*}
	\E[ (1-w) y | \xvec ] &= \E\left[  \E \left( (1 - w) y_{0} | \xvec, w  \right) | \xvec \right]
	\\ &=
	\E \left[ (1 - w) \mu_{0}(\xvec) | \xvec \right]
	\\ &=
	\mu_{0}(\xvec) \E(w | \xvec )
	\\ 
	\E[ (1-w) y | \xvec ] 
	&=
	\mu_{0}(\xvec) [1 - p(\xvec)] \implies
	\\ 
	\Aboxed{
		\mu_{0}(\xvec)
		&=
	\frac{\E[ (1-w) y | \xvec ] }{1 - p( \xvec ) } }
\end{align*}

%%%%%%%%%%%%%%%%%%%%%%%%%%%%%%%%%%%%%%%%%%%%%%%%%%%%%%
\paragraph{ATE:}

\begin{align*}
	\mu_{1}(\xvec) - \mu_{0}(\xvec) =
	\E\left[ 
		\frac{[w - p(\xvec)] y}{p(\xvec) [1 - p(\xvec)]}
		| \xvec
	\right]
\end{align*}

\begin{align*}
	\Aboxed{
		\widehat{ATE} =
		N^{-1} \sum_{i=1}^{N}
		\frac{[ w_{i} - p(\xvec_{i} ) ] y_{i} }{ p( \xvec_{i} ) [1 - p( \xvec_{i} ) ] }
	}
\end{align*}

%%%%%%%%%%%%%%%%%%%%%%%%%%%%%%%%%%%%%%%%%%%%%%%%%%%%%%
\paragraph{ATT:}

\begin{align*}
	\E( y_{1} | \xvec, w=1) - \E(y_{0} | \xvec) =
	\frac{1}{\hat{P}(w=1)}
	\E\left[ 
		\frac{[w - \hat{p}(\xvec)] y}{[ 1 - \hat{p}(\xvec) ]}
		| \xvec
	\right]
\end{align*}

\vspace{-1 em}
\begin{align*}
	\hat{P} (w = 1) = N^{-1} \sum_{i=1}^{N} w_{i}
\end{align*}

\vspace{-1.5 em}
\begin{align*}
	\widehat{ATT} &=
	\frac{N}{\sum_{i=1}^{N} w_{i} }
	N^{-1} \sum_{i=1}^{N}
	\frac{[ w_{i} - \hat{p}(\xvec_{i} ) ] y_{i} }{[ 1 - \hat{p}( \xvec_{i} ) ]}
	\\
	\Aboxed{
		\widehat{ATT} &=
		\frac{1}{\sum_{i=1}^{N} w_{i} }
		\sum_{i=1}^{N}
		\frac{[ w_{i} - \hat{p}(\xvec_{i} ) ] y_{i} }{[1 - \hat{p}( \xvec_{i} ) ]}
	}
\end{align*}

%%%%%%%%%%%%%%%%%%%%%%%%%%%%%%%%%%%%%%%%%%%%%%%%%%%%%%
%%%%%%%%%%%%%%%%%%%%%%%%%%%%%%%%%%%%%%%%%%%%%%%%%%%%%%
\clearpage
\section{Álgebra Linear}

%%%%%%%%%%%%%%%%%%%%%%%%%%%%%%%%%%%%%%%%%%%%%%%%%%%%%%
\subsection{Vetores}


%%%%%%%%%%%%%%%%%%%%%%%%%%%%%%%%%%%%%%%%%%%%%%%%%%%%%%
\subsection{Operações com Vetores}

Vamos definir os vetores $\xvec$ e $\yvec$ com dimensão $1 \times N$:

\begin{align*}
	\xvec = 
	\begin{bmatrix}
		x_{1} \\ \vdots \\ x_{N}	
	\end{bmatrix}
	\quad
	\yvec = 
	\begin{bmatrix}
		y_{1} \\ \vdots \\ y_{N}	
	\end{bmatrix}
\end{align*}

%%%%%%%%%%%%%%%%%%%%%%%%%%%%%%%%%%%%%%%%%%%%%%%%%%%%%%
\paragraph{Multiplicação por escalar}

%%%%%%%%%%%%%%%%%%%%%%%%%%%%%%%%%%%%%%%%%%%%%%%%%%%%%%
\paragraph{Soma}

%%%%%%%%%%%%%%%%%%%%%%%%%%%%%%%%%%%%%%%%%%%%%%%%%%%%%%
\paragraph{Subtração}

%%%%%%%%%%%%%%%%%%%%%%%%%%%%%%%%%%%%%%%%%%%%%%%%%%%%%%
\paragraph{Multiplicação de vetores}

%%%%%%%%%%%%%%%%%%%%%%%%%%%%%%%%%%%%%%%%%%%%%%%%%%%%%%
\paragraph{Poduto Interno (Produto Escalar, Dot Product, Inner Product)}

\begin{align}\label{innerp}
	\xvec' \yvec = \xvec \cdot \yvec = \innerp{\xvec}{\yvec} = \sum_{i=1}^{N} x_{i} y_{i} = x_{1}y_{1} + \dots + x_{N}y_{N}.
\end{align}

Podemos utilizar a equação \eqref{innerp} para denotarmos a soma dos elementos de um vetor.
Para tanto, definimos o vetor $\onevec_{N}$ como sendo o vetor cujos elementos são 1 e tem dimensão $N \times 1$.
\cite[p. 977, A.2.7]{greene-7ed}

% SOMA
\begin{align*}
	x_{1} + \dots + x_{N} =
	\sum_{i=1}^{N} x_{i} =
	\xvec' \onevec_{N} = 
	\onevec_{N}' \xvec = 
	(\xvec' \onevec_{N})'
\end{align*}

Com a definição do vetor $\onevec_{N}$, também podemos escrever:
$\onevec_{N}' \onevec_{N} = N$.

%%%%%%%%%%%%%%%%%%%%%%%%%%%%%%%%%%%%%%%%%%%%%%%%%%%%%%
% Média Aritmética
Usando a definição de \textbf{média aritmética}, também podemos representá-la da seguinte forma:

\begin{align*}
	\frac{x_{1} + \dots + x_{N}}{N} =
	\overline{x} =
	N^{-1} \sum_{i=1}^{N} x_{i} =
	N^{-1} \xvec' \onevec_{N}
\end{align*}

%%%%%%%%%%%%%%%%%%%%%%%%%%%%%%%%%%%%%%%%%%%%%%%%%%%%%%
% Média Ponderada
Usando a definição de \textbf{média ponderada}, também podemos representá-la da seguinte forma:

\begin{align*}
	w_{1}x_{1} + \dots + w_{N}x_{N} =
	\sum_{i=1}^{N} w_{i} x_{i} =
	\wvec'\xvec
\end{align*}
onde $\wvec'\onevec=1$

%%%%%%%%%%%%%%%%%%%%%%%%%%%%%%%%%%%%%%%%%%%%%%%%%%%%%%
\paragraph{Poduto Externo (Outer Product)}

\begin{align*}
	\onevec_{N} \onevec_{N}' =
	\begin{bmatrix}
		1 & \dots & 1	 \\
		\vdots & \ddots & \vdots \\
		1 & \dots & 1	
	\end{bmatrix}_{N \times N}
	\quad
	\onevec_{N} \xvec' =
	\begin{bmatrix}
		x_{1} & \dots & x_{N} \\
		\vdots & \ddots & \vdots \\
		x_{1} & \dots & x_{N}	
	\end{bmatrix}_{N \times N}
	\qquad
	\xvec \onevec_{N}' =
	\begin{bmatrix}
		x_{1} & \dots & x_{1} \\
		\vdots & \ddots & \vdots \\
		x_{N} & \dots & x_{N}	
	\end{bmatrix}_{N \times N}
\end{align*}

\begin{align}\label{eq:xx:outer}
	\xvec \xvec' =
	\begin{bmatrix}
		x_{1} \\ \vdots \\ x_{N}
	\end{bmatrix}
	\begin{bmatrix}
		x_{1} & \dots & x_{N}
	\end{bmatrix}
	=
	\begin{bmatrix}
		x_{1}^{2}  & x_{1} x_{2} & \dots  & x_{1}x_{N} \\
		x_{2}x_{1} & x_{2}^2     & \dots  & x_{2}x_{N} \\
		\vdots     & \vdots      & \ddots & \vdots \\
		x_{N}x_{1} & x_{N} x_{2} & \dots  & x_{N}^{2}
	\end{bmatrix}
\end{align}

%%%%%%%%%%%%%%%%%%%%%%%%%%%%%%%%%%%%%%%%%%%%%%%%%%%%%%
\paragraph{Distância e Ângulo}

O coseno do ângulo, $\alpha$, entre dois vetores é:
\begin{align*}
	\cos(\alpha) = \frac{\uvec'\vvec}{\|\uvec \|\cdot\|\vvec \|}
\end{align*}

Dois vetores são \textbf{ortogonais} se $\uvec'\vvec = 0$.

%%%%%%%%%%%%%%%%%%%%%%%%%%%%%%%%%%%%%%%%%%%%%%%%%%%%%%
\paragraph{Projeções Ortogonal}

\vspace{-1 em}
\begin{align*}
	\proj_{\uvec}(\yvec) =
	\yvechat =
	\frac{\yvec' \uvec}{\uvec'\uvec} \uvec
\end{align*}
é a \textbf{projeção ortogonal} de $\yvec$ em $\uvec$.

\vspace{-1 em}
\begin{align*}
	\zvec = \yvec - \yvechat = \yvec - \frac{\yvec' \uvec}{\uvec'\uvec} \uvec
\end{align*}
é a \textbf{componente de $\yvec$ ortogonal a $\uvec$} \textbf{(Rejeição)}.

Por construção, temos:
\vspace{-1 em}
\begin{align*}
	\zvec + \yvechat = \yvec.
\end{align*}

%%%%%%%%%%%%%%%%%%%%%%%%%%%%%%%%%%%%%%%%%%%%%%%%%%%%%%
\red{[FIGURA AQUI]}
%%%%%%%%%%%%%%%%%%%%%%%%%%%%%%%%%%%%%%%%%%%%%%%%%%%%%%

%%%%%%%%%%%%%%%%%%%%%%%%%%%%%%%%%%%%%%%%%%%%%%%%%%%%%%
\paragraph{Outras Notações Projeções Ortogonal}
Dado que

\vspace{-1 em}
\begin{align*}
	\proj_{\uvec}(\yvec) =
	\frac{\yvec' \uvec}{\uvec'\uvec} \uvec
\end{align*}
é a \textbf{projeção ortogonal} de $\yvec$ em $\uvec$.
Usando $\yvec'\uvec = \uvec'\yvec$, temos

\vspace{-1 em}
\begin{align*}
	\proj_{\uvec}(\yvec) =
	\frac{\uvec \uvec'}{\uvec'\uvec} \yvec
\end{align*}

\noindent
Tirando o $\yvec$ da equação, obtemos o \red{operador de projeção} \red{(Matriz de projeção em $\uvec$?)}:

\vspace{-1 em}
\begin{align*}
	\frac{\uvec \uvec'}{\uvec'\uvec} 
	=
	\uvec(\uvec'\uvec)^{-1} \uvec'
	=
	(\uvec'\uvec)^{-1} \uvec \uvec'
	=
	(\|\uvec \|^2)^{-1} \uvec \uvec'
\end{align*}

Agora, podemos definir o \red{operador rejeição} como:

\vspace{-1 em}
\begin{align*}
	\Imat_{N} - \frac{\uvec \uvec'}{\uvec'\uvec} 
\end{align*}

Usando o caso especial onde $\uvec$ é um vetor de uns ($\uvec = \onevec_{N}$), temos a \textbf{projeção} no eixo de 45 graus:

\vspace{-1 em}
\begin{align*}
	( \onevec_{N}' \onevec_{N} )^{-1} \onevec_{N} \onevec_{N}'
	=
	N^{-1} 
	\begin{bmatrix}
		1      & \dots  & 1	 \\
		\vdots & \ddots & \vdots \\
		1      & \dots  & 1	
	\end{bmatrix}_{N \times N}
	=
	\begin{bmatrix}
		1/N    & \dots  & 1/N	 \\
		\vdots & \ddots & \vdots \\
		1/N    & \dots  & 1/N	
	\end{bmatrix}_{N \times N}
\end{align*}

A \textbf{rejeição} no eixo de 45 graus:

\vspace{-1 em}
\begin{align*}
	\Imat_{N} - \frac{\onevec_{N} \onevec_{N}'}{\onevec_{N}' \onevec_{N}} 
	=
	\begin{bmatrix}
		1      & 0 &\dots  & 0	 \\
		0      & 1 &\dots  & 0	 \\
		\vdots & \vdots & \ddots & \vdots\\
		0      & 0 & \dots  & 1	
	\end{bmatrix}
	-
	\begin{bmatrix}
		1/N    & 1/N    & \dots  & 1/N	 \\
		1/N    & 1/N    & \dots  & 1/N	 \\
		\vdots & \vdots & \ddots & \vdots \\
		1/N    & 1/N    & \dots  & 1/N	
	\end{bmatrix}
	=
	\begin{bmatrix}
		(N-1)/N & 1/N     & \dots  & 1/N    \\
		1/N     & (N-1)/N & \dots  & 1/N    \\
		\vdots  & \vdots  & \ddots & \vdots \\
		1/N     & 1/N     & \dots  & (N-1)/N	
	\end{bmatrix}
\end{align*}

%%%%%%%%%%%%%%%%%%%%%%%%%%%%%%%%%%%%%%%%%%%%%%%%%%%%%%
\paragraph{Centering Matrix} \cite[p. 978, A.28]{greene-7ed}

\vspace{-1 em}
\begin{align*}
	\Mzero &= 
	\Imat_{N} - \onevec_{N} ( \onevec_{N}' \onevec_{N} )^{-1} \onevec_{N}'
	= 
	\Imat_{N} - N^{-1} \onevec_{N} \onevec_{N}' 
\end{align*}

A Matriz $\Mzero$ é \textbf{idempotente} e \textbf{simétrica}.

\begin{description}\itemsep0pt
	\item [Idempotência:] $AA = A$
	\item [Simetria:] $A'=A$
\end{description}

\vspace{-1 em}
\begin{align*}
	\Mzero \xvec &= 
	( \Imat_{N} - N^{-1} \onevec_{N} \onevec_{N}' ) \xvec 
	= 
	\xvec - N^{-1} \onevec_{N} (\onevec_{N}' \xvec) 
	= 
	\xvec - \onevec_{N} \overline{x}
\end{align*}

\noindent
onde podemos denotar

\vspace{-1 em}
\begin{align*}
	\overline{\xvec}
	=
	\onevec_{N} \overline{x}
	=
	\begin{bmatrix}
		\overline{\overline{x}} \\ \vdots \\ \overline{x}
	\end{bmatrix}
\end{align*}

\vspace{-1 em}
\begin{align*}
	\Mzero \xvec 
	= 
	\xvec - \overline{\xvec} 
\end{align*}

\vspace{-1 em}
\begin{align*}
	\Mzero \onevec &= 
	( \Imat_{N} - N^{-1} \onevec_{N} \onevec_{N}' ) \onevec_{N}
	= 
	\onevec_{N} - N^{-1} \onevec_{N} (\onevec_{N}' \onevec_{N}) 
	=
	\onevec_{N} - \onevec_{N} 
	=
	\zerovec_{N} 
\end{align*}



%%%%%%%%%%%%%%%%%%%%%%%%%%%%%%%%%%%%%%%%%%%%%%%%%%%%%%
\subsection{Operações com Matrizes}

\begin{description}
	\item[Multiplicação por escalar] 
	\item[Soma] 
	\item[Subtração] 
\end{description}

%%%%%%%%%%%%%%%%%%%%%%%%%%%%%%%%%%%%%%%%%%%%%%%%%%%%%%
\paragraph{Multiplicação de Matriz}

\begin{align*}
	A_{2 \times 2} =
	\begin{bmatrix}
		a_{11} & a_{12} \\
		a_{21} & a_{22}		
	\end{bmatrix}
	\quad
	B_{2 \times 3} =
	\begin{bmatrix}
		b_{11} & b_{12} & b_{13} \\
		b_{21} & b_{22} & b_{23}		
	\end{bmatrix}
\end{align*}

\begin{align*}
	[AB]_{2 \times 3} &=
	\begin{bmatrix}
		a_{11} \\ a_{21}
	\end{bmatrix}
	\begin{bmatrix}
		b_{11} & b_{12} & b_{13}
	\end{bmatrix}
	+
	\begin{bmatrix}
		a_{12} \\ a_{22}
	\end{bmatrix}
	\begin{bmatrix}
		b_{21} & b_{22} & b_{23}
	\end{bmatrix}
	\implies
	AB = \sum_{i=1}^{2} a_{i}b_{i}
\end{align*}

\noindent
onde
$a_{i}$ é a $i$-ésima \textbf{coluna} da matriz $A$.
$b_{i}$ é a $i$-ésima \textbf{linha} da matriz $B$.
%%%%%%%%%%%%%%%%%%%%%%%%%%%%%%%%%%%%%%%%%%%%%%%%%%%%%%

%%%%%%%%%%%%%%%%%%%%%%%%%%%%%%%%%%%%%%%%%%%%%%%%%%%%%%
\subsubsection{Projeções}

\vspace{-2 em}
\begin{align*}
	\Pmat_{\Xmat} &= \Xmat (\Xmat' \Xmat)^{-1} \Xmat'
	\\
	\Mmat_{\Xmat} &= (\Imat - \Xmat (\Xmat' \Xmat)^{-1} \Xmat')
\end{align*}

Usando o modelo

\vspace{-1 em}
\begin{align*}
	\yvec = \Xmat \betavec + \uvec
\end{align*}

E definindo $\betavechat = (\Xmat' \Xmat)^{-1} \Xmat' \yvec$, temos

\vspace{-1 em}
\begin{align*}
	\mbs{P}_{\Xmat}\yvec  &= \Xmat (\Xmat' \Xmat)^{-1} \Xmat' \yvec = \Xmat \betavechat = \widehat{\yvec}
	\\
	\mbs{M}_{\Xmat}\yvec  &= \Imat \yvec  - \Xmat (\Xmat' \Xmat)^{-1} \Xmat' \yvec 
	=
	\yvec  - \Xmat \betavechat 
	=
	\yvec - \widehat{\yvec} = \uvechat
\end{align*}


%%%%%%%%%%%%%%%%%%%%%%%%%%%%%%%%%%%%%%%%%%%%%%%%%%%%%%
%%%%%%%%%%%%%%%%%%%%%%%%%%%%%%%%%%%%%%%%%%%%%%%%%%%%%%
%%%%%%%%%%%%%%%%%%%%%%%%%%%%%%%%%%%%%%%%%%%%%%%%%%%%%%
\clearpage
\section{Conceitos Básicos de Convergência Estatística} \label{app:est}

%%%%%%%%%%%%%%%%%%%%%%%%%%%%%%%%%%%%%%%%%%%%%%%%%%%%%%
\begin{defn}[\textbf{Convergência em Probabilidade}]
	\citet[Def 3.3, p.36 ]{wool-2010}

	Uma sequência de variáveis aleatórias:
	$\{ X_{N} \}_{N \geq 1}$ 
	\textbf{converge em probabilidade} para uma variável aleatória $X$ se, dado $\err > 0$, 

	\vspace{-1 em}
	\begin{align*}
		P(| X_{N} - X | > \err ) \to 0,
	\end{align*}

	\noindent
	quando $N \to + \infty$.
	E denotamos

	\vspace{-1 em}
	\begin{align*}
		\plim X_{N} = X,
		\quad \text{ou}	\quad
		X_{N} \arrowp X,
		\quad \text{ou}	\quad
		X_{N} - X \arrowp 0.
	\end{align*}
\end{defn}

%%%%%%%%%%%%%%%%%%%%%%%%%%%%%%%%%%%%%%%%%%%%%%%%%%%%%%
\begin{defn}[\textbf{Estimador Consistente}]
	\citet[Def 3.8, p.40 ]{wool-2010}

	Seja $\seq{\thetabold_{N}: N=1, 2, \dots}$ uma sequência de estimadores do vetor $\thetabold \in \Thetabold$ com dimensão $P \times 1$, onde $N$ indexa o tamanho da amostra.
	Se

	\vspace{-1 em}
	\begin{align} \label{consist}
		\widehat{\thetabold} \overset{p}{\longrightarrow} \thetabold
	\end{align}
	Para qualquer valor de $\thetabold$, então dizemos que $\thetabold_{N}$ é um estimador consistente de $\thetabold$.
\end{defn}
%%%%%%%%%%%%%%%%%%%%%%%%%%%%%%%%%%%%%%%%%%%%%%%%%%%%%%

%%%%%%%%%%%%%%%%%%%%%%%%%%%%%%%%%%%%%%%%%%%%%%%%%%%%%%
\begin{teo}[\textbf{LGN -- Lei dos Grandes Números}] \label{teo:lgn}
	\red{Achar referência}
% \citet[Teo 3.1, p.39 ]{wool-2010}

	Seja $\seq{X_{i}}_{i \geq 1}$ uma sequência de variáveis aleatórias $iid$ com $\E(X_{i}) = \mu$.
	Então, 

	\vspace{-1 em}
	\begin{align*}
		N^{-1} \sum_{i=1}^{N} X_{i} \arrowp \mu.
	\end{align*}
\end{teo}
%%%%%%%%%%%%%%%%%%%%%%%%%%%%%%%%%%%%%%%%%%%%%%%%%%%%%%

%%%%%%%%%%%%%%%%%%%%%%%%%%%%%%%%%%%%%%%%%%%%%%%%%%%%%%
\begin{teo}[\textbf{LGN -- Caso Matricial}] \label{teo:lgn:mat:1}
	\red{Achar referência. O Teorema \eqref{teo:lgn:mat}, abaixo, é diferente.}

	Seja $\seq{\xvec_{i}}_{i=1}^{N}$, uma sequência $iid$ de vetores aleatórios $K \times 1$ com $\E(\xvec_{i} \xvec_{i}') = Q_{K \times K}$ finita.
	Então, 

	\vspace{-1 em}
	\begin{align*}
		N^{-1} \sum_{i=1}^{N}
		\xvec_{i} \xvec_{i}'
		\arrowp Q.
	\end{align*}

	Se $Q$ for positiva definida, $Q$ terá inversa.
\end{teo}
%%%%%%%%%%%%%%%%%%%%%%%%%%%%%%%%%%%%%%%%%%%%%%%%%%%%%%

%%%%%%%%%%%%%%%%%%%%%%%%%%%%%%%%%%%%%%%%%%%%%%%%%%%%%%
\begin{teo}[\textbf{LGNF -- WLLN}] \label{teo:lgn:mat}
	\citet[Teo 3.1, p.39 ]{wool-2010}

	Seja $\seq{\wvec_{i} : i=1,2, \dots}$, uma sequência $iid$ de vetores aleatórios $G \times 1$ com
	$\E(|w_{ig}|) < \infty$ para $g = 1, \dots, G$.
	Então, a seguência satisfaz a \textbf{Lei dos Grandes Números Fraca (WLLN)}:

	\vspace{-1 em}
	\begin{align*}
		N^{-1} \sum_{i=1}^{N}
		\wvec_{i} \arrowp \muvec_{w},
	\end{align*}
	onde $\muvec_{w} \equiv \E(\wvec_{i})$.
\end{teo}
%%%%%%%%%%%%%%%%%%%%%%%%%%%%%%%%%%%%%%%%%%%%%%%%%%%%%%

%%%%%%%%%%%%%%%%%%%%%%%%%%%%%%%%%%%%%%%%%%%%%%%%%%%%%%
\begin{defn}[$\litop$]
\item
	\citet[Def 3.4, p.36 ]{wool-2010}\\
	\citet[Lemma 3.2, p.36 ]{wool-2010}

	\vspace{-1 em}
	\begin{align*}
		X_{n} = \litop(1) & \implies X_{n} \arrowp 0
		\\
		X_{n} = \litop(Y_{n}) & \implies
		\frac{X_{n}}{Y_{n}} = \litop(1) \implies
		\frac{X_{n}}{Y_{n}} \arrowp 0
		\\
		X_{n} = W_{n} + \litop(1) & \implies
		(X_{n} - W_{n}) = \litop(1) \implies
		(X_{n} - W_{n}) \arrowp 0
	\end{align*}
\end{defn}
%%%%%%%%%%%%%%%%%%%%%%%%%%%%%%%%%%%%%%%%%%%%%%%%%%%%%%

%%%%%%%%%%%%%%%%%%%%%%%%%%%%%%%%%%%%%%%%%%%%%%%%%%%%%%
\begin{defn}[\textbf{Limitação em Probabilidade:} $\bigOp$]
	\citet[Def 3.3 (3), p.36 ]{wool-2010}

	Dizemos que $X_{n}$ é \textbf{limitado em probabilidade} e denotado por 
	$X_{n} = \bigOp(1)$,
	se existe $M$ maior que zero, tal que para todo $\err$ maior que zero, $P( | X_{n} | > 0) < \err$.

	\vspace{-1 em}
	\begin{align*}
		X_{n} = \bigOp(1) \implies \exists M > 0 \, ; \;
		\forall \err > 0 \, , \;
		P( | X_{n} | > 0) < \err.
	\end{align*}
\end{defn}
%%%%%%%%%%%%%%%%%%%%%%%%%%%%%%%%%%%%%%%%%%%%%%%%%%%%%%

%%%%%%%%%%%%%%%%%%%%%%%%%%%%%%%%%%%%%%%%%%%%%%%%%%%%%%
\begin{defn}
	\citet[Def 3.4, p.36]{wool-2010}

	Dizemos que
	$X_{n} = \bigOp(Y_{n})$ se existe $M$ maior que zero, tal que para todo $\err$ maior que zero, $P( | X_{n}/Y_{n} | > 0) < \err$.

	\vspace{-1 em}
	\begin{align*}
		X_{n} = \bigOp(Y_{n}) \implies \exists M > 0 \, ; \;
		\forall \err > 0 \, , \;
		P( | X_{n}/Y_{n} | > 0) < \err.
	\end{align*}
\end{defn}
%%%%%%%%%%%%%%%%%%%%%%%%%%%%%%%%%%%%%%%%%%%%%%%%%%%%%%

%%%%%%%%%%%%%%%%%%%%%%%%%%%%%%%%%%%%%%%%%%%%%%%%%%%%%%
\begin{lem}
	\citet[Lemma 3.2, p.36]{wool-2010}

	Se
	$X_{n} = \bigOp(1)$ e $Y_{n} = \litop(1)$, então

	\vspace{-1 em}
	\begin{align*}
		X_{n} Y_{n} = \bigOp(1) \litop(1) = \litop(1).
	\end{align*}
\end{lem}
%%%%%%%%%%%%%%%%%%%%%%%%%%%%%%%%%%%%%%%%%%%%%%%%%%%%%%

%%%%%%%%%%%%%%%%%%%%%%%%%%%%%%%%%%%%%%%%%%%%%%%%%%%%%%
\begin{lem}[\textbf{Equivalência Assintótica}] \label{lem:equiv:assin}
	\citet[Lemma 3.7, p.39]{wool-2010}

	Seja
	$\seq{\xvec_{n}}$ e $\seq{\mbs{z}_{n}}$
	sequências de vetores aleatórios $K \times 1$.
	Se $\mbs{z}_{n} \arrowd \mbs{z}$ e 
	$\xvec_{n} - \mbs{z}_{n} \arrowp \zerovec_{K}$.
	Então, 

	\vspace{-1 em}
	\begin{align*}
		\xvec_{n} \arrowd \mbs{z}.
	\end{align*}

\end{lem}
%%%%%%%%%%%%%%%%%%%%%%%%%%%%%%%%%%%%%%%%%%%%%%%%%%%%%%

%%%%%%%%%%%%%%%%%%%%%%%%%%%%%%%%%%%%%%%%%%%%%%%%%%%%%%
\begin{defn}[\textbf{Convergência em Distribuição}]
	\citet[Def 3.6, p.38]{wool-2010}

	Seja
	$\seq{X_{n}}_{n \geq 1}$  uma sequência de variáveis aleatórias e $X$ uma variável aleatória com $F_{n}$ e $F$ suas respectivas FDAs, então

	\vspace{-1 em}
	\begin{align*}
		X_{n} \arrowd X, \text{ se } F_{n}(X) \to F(X)
	\end{align*}
	\noindent
	para todo $X$ onde $F$ é contínuo.
\end{defn}
%%%%%%%%%%%%%%%%%%%%%%%%%%%%%%%%%%%%%%%%%%%%%%%%%%%%%%

%%%%%%%%%%%%%%%%%%%%%%%%%%%%%%%%%%%%%%%%%%%%%%%%%%%%%%
\begin{lem}[\textbf{Convergência em Distribuição e Limitação em Probabilidade}]
	\citet[Lemma 3.5, p.39]{wool-2010}

	Se $X_{n} \arrowd X$, $X$ um variável aleatória qualquer; então $X_{n} = \bigOp(1)$.
\end{lem}
%%%%%%%%%%%%%%%%%%%%%%%%%%%%%%%%%%%%%%%%%%%%%%%%%%%%%%

%%%%%%%%%%%%%%%%%%%%%%%%%%%%%%%%%%%%%%%%%%%%%%%%%%%%%%
\begin{defn}[\textbf{TCL -- Teorema Central do Limite}] \label{def:tcl}
	\red{Achar Referência}
% \citet[Lemma 3.5, p.XX ]{wool-2010}

	Seja $\seq{X_{n}}_{n = 1}^{N}$ $iid$ com $\E(X_{n}) = \mu$ e $\Var(X_{n}) = \sigma^{2} < + \infty$.
	Então, para $S_{N} = \sum_{n=1}^{N} X_{n}$:

	\begin{align*}
		\frac{S_{N} - N \mu }{ \sqrt{N} \sigma}
		=
		\frac{N ( \overline{X} - \mu ) }{ \sqrt{N} \sigma}
		=
		\frac{N ( \overline{X} - \mu ) }{ \sqrt{N} \sigma}
		=
		\boxed{
			\frac{\sqrt{N} ( \overline{X} - \mu ) }{\sigma}
		\arrowd Z \sim N(0,1)}.
	\end{align*}
\end{defn}
%%%%%%%%%%%%%%%%%%%%%%%%%%%%%%%%%%%%%%%%%%%%%%%%%%%%%%

%%%%%%%%%%%%%%%%%%%%%%%%%%%%%%%%%%%%%%%%%%%%%%%%%%%%%%
\begin{teo}[\textbf{TCL -- Lindeberg-Levy}] \label{def:tcl:vec}
	\citet[Teo 3.2, p.40]{wool-2010}

	Seja $\seq{\wvec_{i}:i=1,2,\dots}$ uma sequência $iid$ de vetores aleatórios $G \times 1$ com
	$\E( w_{ig}^{2} ) < \infty$ para $g= 1, \dots G$ 
	e
	$\E( \mbs{w}_{i} ) = \zerovec$.
	Então, $\seq{\wvec_{i}:i=1,2,\dots}$ satisfaz o \textbf{Teorema Central do Limite (CLT)}; 
	qual seja:

	\vspace{-1 em}
	\begin{align*}
		N^{-1/2} \sum_{i=1}^{N} \mbs{w}_{i} \arrowd \nor(\zerovec, \Bmat),
	\end{align*}

	\noindent
	onde, $\Bmat = \Var(\mbs{w}_{i}) = \E( \mbs{w}_{i} \mbs{w}_{i}')$ é necessariamente positiva semidefinida.
	Para nossos propósitos, $\Bmat$ será sempre positiva definida.
\end{teo}
%%%%%%%%%%%%%%%%%%%%%%%%%%%%%%%%%%%%%%%%%%%%%%%%%%%%%%

%%%%%%%%%%%%%%%%%%%%%%%%%%%%%%%%%%%%%%%%%%%%%%%%%%%%%%
\begin{cor}[]
	\citet[Cor 3.2, p.39]{wool-2010}

	Seja $\seq{\zvec_{N}}$ uma sequência de vetores $K \times 1$ aleatórios tal que
	$\zvec_{N} \arrowd \nor(\zerovec, \Vmat)$, então:

	\begin{enumerate}
		\item 
			Para qualquer matriz $\Amat$ de dimensão $K \times M$ \textbf{não} estocástica, temos:
			\begin{align*}
				\Amat' \zvec_{N} \arrowd \nor(\zerovec, \Amat'\Vmat\Amat).
			\end{align*}

		\item $\zvec_{N}'\Vmat^{-1}\zvec_{N} \arrowd \chisq_{K}$ (ou $\zvec_{N}'\Vmat^{-1}\zvec_{N} \asim \chisq_{K}$).
	\end{enumerate}


\end{cor}
%%%%%%%%%%%%%%%%%%%%%%%%%%%%%%%%%%%%%%%%%%%%%%%%%%%%%%

%%%%%%%%%%%%%%%%%%%%%%%%%%%%%%%%%%%%%%%%%%%%%%%%%%%%%%
\begin{defn}[\textbf{raiz de $N$ assintoticamente normalmente distribuido}]
	\citet[Def 3.9, p.40]{wool-2010}

	Seja $\seq{\thetaboldhat_{N}: N=1,2,\dots}$ uma sequência de estimadores do vetor $\thetabold \in \Thetabold$ com dimensão $P \times 1$.
	Suponha que

	\vspace{-1 em} 
	\begin{align}\label{nroot:conv}
		\sqrt{N}(\thetaboldhat_{N} - \thetabold) 
		\arrowd 
		\nor(\zerovec, \Vmat)
	\end{align}
	onde $\Vmat$ é uma matriz $P \times P$ positiva semidefinida.
	Então dizermos que $\thetaboldhat_{N}$ é $\sqrt{N}$-asintoticamente normalmente distribuido e
	$\Vmat$ é a \textbf{variância assintótica} de 
	$\sqrt{N}(\thetaboldhat_{N} - \thetabold)$,
	denotada $\Avar \sqrt{N}(\thetaboldhat_{N} - \thetabold) = \Vmat$
\end{defn}
%%%%%%%%%%%%%%%%%%%%%%%%%%%%%%%%%%%%%%%%%%%%%%%%%%%%%%

%%%%%%%%%%%%%%%%%%%%%%%%%%%%%%%%%%%%%%%%%%%%%%%%%%%%%%
\begin{remark}
	Apesar de $\Vmat/N = \Var(\thetaboldhat_{N})$ ser verdade apenas em casos especiais, e raramente $\thetaboldhat_{N}$ ter uma distribuição exatamete normal, tratamos $\thetaboldhat_{N}$ como se 

	\vspace{-1 em}
	\begin{align}\label{eq:3.4}
		\thetaboldhat_{N} \sim \nor(\thetabold, \Vmat/N).
	\end{align}
	sempre que a equação \eqref{nroot:conv} for verdade.
	Por essa razão, $\Vmat/N$ é chamado de \textbf{variância assintótica} de $\thetaboldhat_{N}$, e escrevemos:

	\vspace{-1 em}
	\begin{align}\label{eq:3.5}
		\Avar(\thetaboldhat_{N}) = \Vmat/N.
	\end{align}

\end{remark}
%%%%%%%%%%%%%%%%%%%%%%%%%%%%%%%%%%%%%%%%%%%%%%%%%%%%%%

%%%%%%%%%%%%%%%%%%%%%%%%%%%%%%%%%%%%%%%%%%%%%%%%%%%%%%
\hline
\vspace{1 ex}
%%%%%%%%%%%%%%%%%%%%%%%%%%%%%%%%%%%%%%%%%%%%%%%%%%%%%%
\red{Abaixo, temos as definições necessárias para mostrar como verificar se um estimador é consistente por EQM.
	Quero saber se mantemos essa parte.
Se sim, precisamos de referências.}

%%%%%%%%%%%%%%%%%%%%%%%%%%%%%%%%%%%%%%%%%%%%%%%%%%%%%%
\begin{defn}[\textbf{Desigualdade de Markov}]
	\red{Achar referência}

	Seja
	$\{ X_{n} \}_{n \geq 1}$ 
	uma sequência de variáveis aleatórias com
	$E|X_{n}|^{K} < +\infty$, $K>0$. 
	Então, dado $\err >0$

	\vspace{-1 em}
	\begin{align*}
		P(| X_{n} | > \err ) \leq \frac{E|X_{n}|^{K}}{\err^{K}}
	\end{align*}
\end{defn}
%%%%%%%%%%%%%%%%%%%%%%%%%%%%%%%%%%%%%%%%%%%%%%%%%%%%%%

%%%%%%%%%%%%%%%%%%%%%%%%%%%%%%%%%%%%%%%%%%%%%%%%%%%%%%
\begin{defn} %[\textbf{}]
	\red{Achar referência}

	\begin{align*}
		0 \leq P(| \thetahat - \theta | > \err ) \leq \frac{E|X_{n}|^{2}}{\err^{2}}
	\end{align*}
\end{defn}
%%%%%%%%%%%%%%%%%%%%%%%%%%%%%%%%%%%%%%%%%%%%%%%%%%%%%%

%%%%%%%%%%%%%%%%%%%%%%%%%%%%%%%%%%%%%%%%%%%%%%%%%%%%%%
\begin{defn}[\textbf{Erro Quadrático Médio}]
	\red{Achar referência}

	\begin{align*}
		EQM(\thetahat) 
		=
		\E\left[ \left( \thetahat - \theta \right)^2 \right] 
		=
		\left[ Bias(\thetahat)^{2} + \Var(\thetahat) \right]
	\end{align*}
\end{defn}
%%%%%%%%%%%%%%%%%%%%%%%%%%%%%%%%%%%%%%%%%%%%%%%%%%%%%%

%%%%%%%%%%%%%%%%%%%%%%%%%%%%%%%%%%%%%%%%%%%%%%%%%%%%%%
\begin{remark}
	\red{Achar referência?}

	Então, se $Bias(\thetahat) \to 0$ e $\Var(\thetahat) \to 0$, temos que $EQM(\thetahat) \to 0$.
	Pelo \textbf{Teorema do Sanduíche}, $P(|\thetahat - \theta| > \err) \to 0$; logo, $\thetahat \arrowp \theta$.
\end{remark}
%%%%%%%%%%%%%%%%%%%%%%%%%%%%%%%%%%%%%%%%%%%%%%%%%%%%%%

%%%%%%%%%%%%%%%%%%%%%%%%%%%%%%%%%%%%%%%%%%%%%%%%%%%%%%
\clearpage

\cite{bryant84}
\cite{herr80}
\cite{matman}
\cite{ruud-2000}
\cite{dm04}
\cite{faraway15}
\cite{vmls18}

%%%%%%%%%%%%%%%%%%%%%%%%%%%%%%%%%%%%%%%%%%%%%%%%%%%%%%
\clearpage
\thispagestyle{empty}

\begin{landscape}

\begin{table}[!ht]
\centering
% \scriptsize
% \footnotesize
\caption{Vector Space Ideas Expressed in Different Ways}
\vspace{-1 em} 
\label{tab:sth}
\begin{tabular}{@{\extracolsep{1 ex}} p{4.25cm} p{4cm} p{4.25cm} p{5cm} p{4cm}}
% ---------------------------------------- 
\hline\hline
% ---------------------------------------- 
& \multicolumn{2}{c}{Geometric Expression}  \\ \cline{2-3}
Vector Space Idea & Coordinate Free Form  & Analytic Form & Statistical Expression & Probabilistic Expression 
% ---------------------------------------- 
\\ \hline	
% ---------------------------------------- 

Vector $z$ &
Directed Line Segment \newline from $0$ to $z$ &
$z = (z_{1}, \dots, z_{n})$ &
Data $z_{1}, \dots, z_{n}$ &
Random Variable $z$
\\[4 ex]

Vector $z - y$ &
Directed Line Segment \newline from $y$ to $z$ &
$z - y =$ \newline $(z_{1} - y_{1}, \dots, z_{n} - y_{n})$ &
Differences  \newline $z_{1} - y_{1}, \dots, z_{n} - y_{n}$
& Random variable \newline $z - y$ 
\\[2 ex]

Inner Product \newline of $x$ and $z$, $\innerp{x, z}$ &
Measure of \newline Nonorthogonality &
$x_{1} z_{1} + \dots + x_{n} z_{n}$ &
Sums of cross \newline products &
$\E(xz)$
\\ [4 ex]

$\abs{z}^2 = \innerp{z, z}$ &
Squared distance, \newline from $0$ to $z$ &
$z_{1}^{2} + \dots + z_{n}^{2}$ &
Sum of Squares of $z$ &
$\E(z^2)$
\\[4 ex]

$\abs{z - y}^2$ &
Squared distance, \newline from $y$ to $z$ &
$(z_{1} - y_{1})^2 + \dots + (z_{n} - y_{n})^2$ &
Sum of Squared \newline differences &
$\E(z-y)^{2}$
\\[4 ex]

$\innerp{x, z} = 0$ &
\multicolumn{2}{c}{Orthogonality: $x, z$ perpendicular} &
&
$\E(xz) = 0$
\\[2 ex]

All vectors of the form \newline $ax + by + cz + \dots$ &
\multicolumn{2}{c}{Plane L(x, y, z, \dots) through $0$ determined by vectors $\xvec, \yvec, \zvec, \dots$} &
All models of the form \newline $ax + by + cz + \dots$ &
All square-integrable \newline functions of $x, y, z, \dots$
\\[2 ex]

A vector $P$ in $L(x, y, \dots)$ \newline such that $\innerp{x - P, Q} = 0$ \newline for every $Q$ in $L$ & 
\multicolumn{2}{c}{Projection $P_{L}(z)$ of $z$ onto $L(x, y, \dots)$} &
The best-fitting model \newline of the form \newline $ax + by + c\ + \dots$ &
$\E(z | x, y, \dots)$
\\[8 ex]

``45-degree'' vector $J$ &
&
$J = (1, 1, \dots, 1)$ &
Constant $(1, 1, \dots, 1)$ &
Constant $1$
\\[2 ex]

$\abs{x - P_{J}(x)}^{2}$ &
&
$\sum (x_{i} - \bar{x})^{2}$ &
$(n-1) x$ sample \newline variance of $x$ &
Variance of $x$
\\[1 ex]

$\dfrac{ \innerp{ x - P_{J}(x), z - P_{J}(z) } }{\abs{x - P_{J}(x)} \cdot \abs{z - P_{J}(z)}}$ &
\multicolumn{2}{c}{cosine of angle between $x - P_{J}(x)$ and $z-P_{J}(z)$} &
sample correlation \newline of $x$ and $z$ &
Correlation \newline of $x$ and $z$
\\[1 ex]

$\innerp{x-P_{J}(x), z-P_{J}(z)} = 0$ & 
\multicolumn{2}{c}{$x - P_{J}(x)$ and $z-P_{J}(z)$ are perpendicular} &
$x$ and $z$ are \newline uncorrelated &
$x$ and $z$ are \newline uncorrelated
\\

% ---------------------------------------- 
\hline	\hline
% ---------------------------------------- 
\end{tabular}
\vspace{-1 ex} 
\begin{tablenotes} 
\textbf{Source:} \citet[Table 1, p.43]{bryant84}
\end{tablenotes} 
\end{table}

\end{landscape}
%%%%%%%%%%%%%%%%%%%%%%%%%%%%%%%%%%%%%%%%%%%%%%%%%%%%%%

%%%%%%%%%%%%%%%%%%%%%%%%%%%%%%%%%%%%%%%%%%%%%%%%%%%%%%
\clearpage

%%%%%%%%%%%%%%%%%%%%%%%%%%%%%%%%%%%%%%%%%%%%%%%%%%%%%%
\clearpage

%%%%%%%%%%%%%%%%%%%%%%%%%%%%%%%%%%%%%%%%%%%%%%%%%%%%%%
\clearpage
\begin{figure}[!ht]
\caption{Vector Addition}
\vspace{-1 ex}
\centering
\begin{tikzpicture}[scale=.75]
\draw [--]  (-1,-1) -- (9,-1)  -- (9,6) -- (-1,6) node[below right] {L} -- cycle ;;
% origin
\fill (0,0) circle[radius=2pt] node[below]{$0$};

\draw [->, thick] (0,0) -- (5,0) node[right] {$x$};
\draw [->, thick] (0,0) -- (2,5) node[above] {$z$};
\draw [->, thick] (0,0) -- (7,5) node[above left] {$x + z$};

\draw [--, thick, dashed] (5,0) -- (7,5);
\draw [--, thick, dashed] (2,5) -- (7,5);
\end{tikzpicture}
\end{figure}

%%%%%%%%%%%%%%%%%%%%%%%%%%%%%%%%%%%%%%%%%%%%%%%%%%%%%%
\begin{figure}[!ht]
\centering
\caption{Vector Subtraction}
\vspace{-1 ex}
\begin{tikzpicture}[scale=.85]
\draw [--]  (-3,-4) -- (5,-4)  -- (5,4) -- (-3,4) node[below right] {L} -- cycle ;;
% origin
\fill (0,0) circle[radius=2pt] node[below]{$0$};

\draw [->, thick] (0,0) -- (3,0) node[right] {$x$};
\draw [->, thick] (0,0) -- (2,3) node[left] {$z$};
\draw [->, thick] (3,0) -- (2,3) node[below right] {$z-x$};
\draw [->, thick] (0,0) -- (-2,-3) node[right] {$-z$};
\end{tikzpicture}
\end{figure}

%%%%%%%%%%%%%%%%%%%%%%%%%%%%%%%%%%%%%%%%%%%%%%%%%%%%%%
\begin{figure}[!ht]
\centering
\caption{Scalar Multiplication and Subspaces}
\vspace{-1 ex}
\begin{tikzpicture}[scale=0.6]
\draw [--]  (-5,-5) -- (15,-5) node[above left] {$L = L(x, z)$}  -- (15,9) -- (-5,9) -- cycle;

% \draw [->, thick] (0,0) node[below] {$0$} -- (3,0) node[below] {$x$};
% origin

% x
\draw [->, thick] (0,0) -- (3,0) node[below] {$x$};
% -x/2
\draw [->, thick] (0,0) -- (-3/2,0) node[below] {$-x/2$};
% L(x)
\draw [--, thick, dashed] (-4,0) -- (14,0) node[below left] {$L(x)$};
% 3x
\draw [->, thick, dashed] (0,0) -- (9,0) node[below] {};

% z
\draw [->, thick] (0,0) -- (1,2) node[left] {$z$};
% 3z
\draw [->, thick] (0,0) -- (3,6) node[left] {$3z$};
% L(z)
\draw [--, thick, dashed] (-2,-4) node[right] {$L(z)$} -- (4,8);

% 3z'
\draw [--, thick, dotted] (3,6) -- (12,6);
\draw [->, thick, densely dotted] (9,0) -- (12,6);

% y
\draw [->, thick, densely dotted] (0,0) -- (10,5) node[above] {$y$} -- (12,6) ;

% origin
\fill (0,0) circle[radius=4pt] node[below]{$0$};
\end{tikzpicture}
\end{figure}

%%%%%%%%%%%%%%%%%%%%%%%%%%%%%%%%%%%%%%%%%%%%%%%%%%%%%%
\begin{figure}[!ht]
\centering
\caption{The Law of Cosines 1}
\vspace{-1 ex}
\begin{tikzpicture}[scale=1]
% origin
\fill (0,0) circle[radius=2pt] node[below]{$0$};

% x
\draw [->, thick] (0,0) -- (5,0) node[below] (x) {$x$};
\draw [--, dashed] (-1,0) -- (7,0);
% z
\draw [->, thick] (0,0) -- (-2,3) node[left] {$z$};
% x+z
\draw [--, densely dotted] (5,0) -- (3,3);
\draw [--, densely dotted] (-2,3) -- (3,3);
\draw [->, thick] (0,0) -- (3,3) node[left] {$x+z$};

\draw [->] (5.5,0) arc (0:135:10pt);
\draw [->] (.5,0) arc (0:135:10pt);

\draw (5.1, 0.5) node[right] {$\theta$};
\draw (0.1, 0.5) node[right] {$\theta$};

\end{tikzpicture}
\end{figure}

%%%%%%%%%%%%%%%%%%%%%%%%%%%%%%%%%%%%%%%%%%%%%%%%%%%%%%
\begin{figure}[!ht]
\centering
\caption{The Law of Cosines 2}
\vspace{-1 ex}
\begin{tikzpicture}[scale=1]
% coordinates
\coordinate (O) at (0,0);
\coordinate (x) at (5,0);
\coordinate (z) at (-2,3);
\coordinate (x+z) at (3,3);

% draw vectors
\draw [->, thick] (0)--(x);
\draw [->, thick] (0)--(z);
\draw [->, thick] (0)--(x+z);
\fill (O) circle[radius=2pt];

% labels
\draw (x) node[below] {$x$};
\draw (z) node[left] {$z$};
\draw (x+z) node[left] {$x+z$};
\draw (O) node[below] {$0$};

\draw [--, dashed] (-1,0) -- (7,0);
\draw [--, densely dotted] (5,0) -- (3,3);
\draw [--, densely dotted] (-2,3) -- (3,3);

\draw [->] (0.5,0) arc (0:135:10pt);
\draw [->] (5.5,0) arc (0:135:10pt);

\draw (5.1, 0.5) node[right] {$\theta$};
\draw (0.1, 0.5) node[right] {$\theta$};

\end{tikzpicture}
\end{figure}

%%%%%%%%%%%%%%%%%%%%%%%%%%%%%%%%%%%%%%%%%%%%%%%%%%%%%%
\begin{figure}[!ht]
\centering
\caption{Projection}
\vspace{-1 ex}
\begin{tikzpicture}
% coordinates
\coordinate (o) at (0,0);

\coordinate (a) at (-2,-2);
\coordinate (b) at (8,-2);
\coordinate (c) at (10,2);
\coordinate (d) at (0,2);

\coordinate (y) at (6,4);
\coordinate (yhat) at (6,0);

% draw FIG
\draw [--, thick] (a)--(b)--(c)--(d)--(a) ;
\draw [->, thick] (o)--(y) ;
\draw [->, thick] (o)--(yhat) ;
\draw [->, thick, dashed] (yhat)--(y) ;

% right angle
\draw [--] (5.5,0) -- (5.5,.5) -- (6,.5);

% labels
\draw (5.5, 3.5) node[above left] {$\yvec$};
\draw (5.75, 0) node[below] {$\yvechat = \Xmat \betavechat$};
\draw (6, 3) node[right] {$\uvechat$};
\draw (1, -2) node[above] {Space spanned by $\Xmat$};

\end{tikzpicture}
\newline
\citet[p.15]{faraway15}
\end{figure}

%%%%%%%%%%%%%%%%%%%%%%%%%%%%%%%%%%%%%%%%%%%%%%%%%%%%%%
\begin{figure}[!ht]
\centering
\caption{Least Squares Estimate}
\vspace{-1 ex}
\begin{tikzpicture}
% coordinates
\coordinate (o) at (0,0);

\coordinate (a) at (-1,-1);
\coordinate (b) at (9,-1);
\coordinate (c) at (11,4);
\coordinate (d) at (1,4);

\coordinate (y) at (7,6);
\coordinate (yhat) at (7,1);

% draw FIG
\draw [--, thick] (a)--(b)--(c)--(d)--(a) ;

\draw [->, thick] (a)--(y) ;
\draw [->, thick] (a)--(yhat) ;
\draw [--, thick, dashed] (yhat)--(y) ;

% labels
\draw (6.5, 5.75) node[left] {$\yvec$};
\draw (8, 1.5) node[below] {$\widehat{\yvec} = \Xmat \betavechat$};
\draw (8.7, -1) node[above] {$[\Xmat]$};

\end{tikzpicture}
\newline
\citet[p.44]{herr80}
\end{figure}

%%%%%%%%%%%%%%%%%%%%%%%%%%%%%%%%%%%%%%%%%%%%%%%%%%%%%%
\begin{figure}[!ht]
\centering
\caption{Square}
\vspace{-1 ex}
\begin{tikzpicture}
% coordinates
\coordinate (o) at (0,0);
\coordinate (a) at (-5,-5);
\coordinate (b) at (5,-5);
\coordinate (c) at (5,5);
\coordinate (d) at (-5,5);

% draw FIG
\draw [--, thick] (a)--(b)--(c)--(d)--(a) ;

\end{tikzpicture}
\end{figure}


%%%%%%%%%%%%%%%%%%%%%%%%%%%%%%%%%%%%%%%%%%%%%%%%%%%%%%
\clearpage
% axis
\begin{tikzpicture}
	\draw [->, very thick] (0,0) -- (5,0) node[right] {$x$};
	\draw [->, very thick] (0,0) -- (0,5) node[above] {$y$};
\end{tikzpicture}




%%%%%%%%%%%%%%%%%%%%%%%%%%%%%%%%%%%%%%%%%%%%%%%%%%%%%%

%%%%%%%%%%%%%%%%%%%%%%%%%%%%%%%%%%%%%%%%%%%%%%%%%%%%%%
%   Bibliografia
\clearpage
\bibliographystyle{authordate3} % bibliography style
\renewcommand\bibname{REFERENCES} 
\bibliography{./refs.bib}

%%%%%%%%%%%%%%%%%%%%%%%%%%%%%%%%%%%%%%%%%%%%%%%%%%%%%%%
\end{document}


